\documentclass{article}
\usepackage{amsmath}
\usepackage{amssymb}
\usepackage{amsthm}
\usepackage{mathtools}
\usepackage{physics}
\usepackage{geometry}
\usepackage{siunitx}
\usepackage{bm} % Bold math symbols
\usepackage{commath} % Differential operators, etc.
\usepackage{mathrsfs} % Script math fonts

\geometry{a4paper, margin=1in}
\usepackage{parskip} % For zero paragraph indentation

\begin{document}
\begin{math}
	\vec{a}=(1;-1;0;2;-3;5)\\
	\vec{b}=(\frac{1}{2};\frac{-1}{4};1;-1;\frac{2}{3};\frac{-1}{3})\\
	\vec{c}=(1;2;3;-0.5;-1;2)\\
	\vec{x}=?
\end{math}
\begin{align*}
	[(2\vec{a}-3\vec{b})-\vec{c}]\cdot(-2)+3\vec{x}&=
	2(\vec{x}-3\vec{c})-(-3)(\vec{a}+3\vec{b})\\
	(-4\vec{a})+6\vec{b}+2\vec{c}+3\vec{x}&=
	2\vec{x}-6\vec{c}+3\vec{a}+9\vec{b}
	\hspace{2em}/-2\vec{x}-2\vec{c}-6\vec{b}+4\vec{a}\\
	\vec{x}&=7\vec{a}+3\vec{b}-8\vec{c}\\
	\vec{x}&=7(1;-1;0;2;-3;5)
	+3(\frac{1}{2};\frac{-1}{4};1;-1;\frac{2}{3};\frac{-1}{3})
	-8(1;2;3;-0.5;-1;2)\\
	\vec{x}&=(7;-7;0;14;-21;35)+
	(1.5;-0.75;3;-3;2;-1)+
	(-8;-16;-24;4;8;-16)\\
	\vec{x}&=(0.5;-23.75;-21;15;-11;18)
\end{align*}	
\end{document}
