\documentclass{article}
\usepackage{amsmath}
\usepackage{amssymb}
\usepackage{amsthm}
\usepackage{mathtools}
\usepackage{physics}
\usepackage{geometry}
\usepackage{siunitx}
\usepackage{bm} % Bold math symbols
\usepackage{commath} % Differential operators, etc.
\usepackage{mathrsfs} % Script math fonts

\geometry{a4paper, margin=1in}

\begin{document}
	{\Large \textbf{Odchylka vektorů -- směrový a normálový vektor}}
	\begin{align*}
		\vec{u}&=(-2;3) \rightarrow \lvert\vec{u}\rvert=\sqrt{(-2)^2+3^2}=\sqrt{4+9}=\sqrt{13}\\
		\vec{v}&=(1;-1) \rightarrow \lvert\vec{v}\rvert=\sqrt{2}\\
		\vec{u}\cdot\vec{v}&=(-2)\cdot 1+3\cdot (-1)=-5\\
		\cos(\alpha)&=\dfrac{\lvert-5\rvert}{\sqrt{13}\cdot\sqrt{2}}=\dfrac{5}{\sqrt{26}}\\
		\cos[-1]({\dfrac{5}{\sqrt{26}}})&=\alpha \rightarrow \alpha\approx 11,3^o
	\end{align*}\\
	\vspace{1em}\\
	Pozn.: Pokud skalární součin 2 vektorů je roven 0,
	pak tyto vektory jsou na sebe kolmé (svírají $90^o$).\\
	vektory -- směrový $\rightarrow \vec{BA}=B-A=(b_1-a_1;b_2-a_2)$\\
	vektory -- normálový $\rightarrow$ kolmý na směrový vektor.\\
	\vspace{1em}\\
	\begin{align*}
		\vec{u}&=(4;-3)\\
		\vec{n}_{\vec{u}}&=(3;4)/(-3;-4)
	\end{align*}
\end{document}
