\documentclass[ekobook.tex]{subfiles}

\begin{document}\setWPP
\lsection{Hospodaření podniku \TODO}
\qaw{Co jsou náklady, výnosy a tržby}{%
\begin{itemize}[label={\color{WPP}\faSlackHash}]
\item \textbl{Náklady (N)} -- spotřeba vyjádřená v peněžních jednotkách
\item \textbl{Výnosy} -- výroba a prodej vyjádřen v penězích
\item \textbl{Tržby} -- část výnosů týkající se pouze prodeje
\end{itemize}
}
\qaw{Jak se dělí N podle druhu}{%
\begin{itemize}[label=\textcolor{WPP}{$\hookrightarrow$}]
\item \textbl{provozní} -- vztahují se k provozní činnosti (spotřeba materiálu, mzdy, energie)
\item \textbl{finanční} -- finanční činnost (úroky, pojistné, poplatky)
\item \textbl{mimořádné} -- mimořádné události (manka, škody na majetku)
\end{itemize}
}
\qaw{Jak se dělí N podle vztahu a závislosti na objemu výroby}{%
Uveď příklady fixních a variabilních N
\begin{itemize}[label=\textcolor{WPP}{$\hookrightarrow$}]
\item \textbl{fixní} -- nezáleží na množství výroby (úroky, nájem, mzdy úředních pracovníků, odpisy)
\item \textbl{variabilní} -- mění se dle množství výroby (spotřeba materiálu, mzdy dělníků)
\end{itemize}
}
\qaw{Jak se liší variabílní N}{%
Uveď příklady těch, co rostou pomaleji, stejně nebo rychleji
\begin{itemize}[label=\textcolor{WPP}{$\hookrightarrow$}]
\item \textbl{rychleji} -- rostou rychleji (odměny za přesčasy)
\item \textbl{stejně rychle} -- rostou s objemem výroby (mzdy za výrobky)
\item \textbl{pomaleji} -- rostou pomaleji (cena hromadné výroby)
\end{itemize}
}
\qaw{Vyjmenuj kalkulační členění N}{%
\begin{itemize}[label=\textcolor{WPP}{$\hookrightarrow$}]
\item \textbl{přímé (jednicové)} -- jdou propočítat na hodnotu jednoho výrobku (materiál, práce...)
\item \textbl{nepřímé (režijní)} -- odpisy, energie, práce ředitele...
\end{itemize}
}
\wbreak
\qaw{Jak se dělí výnosy podle druhu, uveď příklad}{%
\begin{itemize}[label=\textcolor{WPP}{$\hookrightarrow$}]
\item \textbl{provozní} -- tržby z prodeje materiálů, služeb...
\item \textbl{finanční} -- přijaté úroky, tržby z prodeje cenných papírů...
\item \textbl{mimořádné} -- přebytek při inventuře, dary...
\end{itemize}
}
\qaw{Jaký může být hospodářský výsledek}{%
\begin{tabular}{l@{ }l@{ }l}
Zisk, když & Náklady {\color{WPP}\faLessThan} Výnosy \\
Rovnost, když & Náklady {\color{WPP}\faEquals}    Výnosy \\
Ztráta, když & Náklady {\color{WPP}\faGreaterThan} Výnosy
\end{tabular}
}
\qaw{Popiš metody stanovení ceny podle:}{%
nákladů, konkurence, hodnoty vnímané zákazníkem, podle poptávky
}
\wpage
\end{document}