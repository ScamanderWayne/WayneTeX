\documentclass[ekobook.tex]{subfiles}

\begin{document}\setgreen
\lsection{Hospodaření podniku \TODO}
\qaw{Co jsou náklady, výnosy a tržby}{%
\begin{itemize}[label={\color{WPP}\faSlackHash}]
\item \textbl{Náklady (N)} -- spotřeba vyjádřená v peněžních jednotkách
\item \textbl{Výnosy} -- výroba a prodej vyjádřen v penězích
\item \textbl{Tržby} -- část výnosů týkající se pouze prodeje
\end{itemize}
}
\qaw{Jak se dělí N podle druhu}{%
\begin{itemize}
\item \textbl{provozní} --
\item \textbl{finanční} --
\item \textbl{mimořádné} --
\end{itemize}
}
\qaw{Jak se dělí N podle vztahu a závislosti na objemu výroby}{%
Uveď příklady fixních a variabilních N
}
\qaw{Jak se liší variabílní N -- N}{%
Uveď příklady těch, co rostou pomaleji, stejně nebo rychleji
}
\qaw{Vyjmenuj kalkulační členění N}{%
něco
}
\qaw{Jak se dělí výnosy podle druhu, uveď příklad}{%
něco
}
\qaw{Jaký může být hospodářský výsledek}{%
něco
}
\qaw{Popiš metody stanovení ceny podle:}{%
nákladů, konkurence, hodnoty vnímané zákazníkem, podle poptávky
}
\wpage
\end{document}