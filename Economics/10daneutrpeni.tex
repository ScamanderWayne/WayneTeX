\documentclass[ekobook.tex]{subfiles}

\begin{document}\IfDarkModeTF{\definecolor{WPP}{HTML}{FF8A8A}}{\definecolor{WPP}{HTML}{4D0000}}
\lsection{Daňová soustava \DORN}
\vspace{-2em}
\wpgfbox{%
Účelem daňové soustavy je \texthl{zajistit příjmy} do státního rozpočtu, které se používají \texthl{k zajištění chodu} státu a jeho \texthl{potřeb}. Jako jsou \texthl{státní správa} a \texthl{veřejný sektor} (nemocnice, školy, ...).
}
\vspace{-1em}
\qaw{Jaké jsou podmínky placení daní a kde je musíme sledovat}{%
\texthl{Výše} částky, \texthl{čas} splatnosti a \texthl{další podmínky} se neustále \texthl{mění}, proto se musí sledovat \texthl{v aktuálním znění zákonů} náležících k~daním.\\
Například v \href{https://www.e-sbirka.cz/}{\texthl{Elektronické Sbírce zákonů a mezinárodních smluv}}, jmenovitě pak
\href{https://www.e-sbirka.cz/sb/1992/586/2026-01-01}{\emph{daň z příjmu}},
\href{https://www.e-sbirka.cz/sb/1993/16}{\emph{silniční daň}} a
\href{https://www.e-sbirka.cz/sb/2003/353}{\emph{spotřební daň}}.
}
%\vspace{-2em}
\wpgfboxed{%
\subsection*{\faAngleDoubleRight Dělení daní}
\begin{enumerate}[leftmargin=2em,label={\color{WPP}\textbf{\arabic*.}}]
\item \textbf{přímé daně} -- poplatník a plátce jsou jednou a tou samou osobou
\begin{enumerate}[leftmargin=1em,label={\color{WPP}\textbf{\alph*)}}]
\item \textbf{majetkové} -- \emph{daň z nemovitosti}, \emph{daň silniční}
\item \textbf{důchodové} -- \emph{daň z příjmů fyzických osob} a také \emph{daň z příjmu právnických osob}
\end{enumerate}
\item \textbf{nepřímé daně} -- poplatník není zároveň plátcem; kupříkladu \emph{daň z přidané hodnoty}
\end{enumerate}
}{0.9}{center}
%\vspace{-1em}
\qaw{Vysvětli pojmy:}{%
\setlength{\multicolsep}{0pt}% Remove vertical space before and after
\setlength{\columnsep}{2em}% Remove space between columns
\setlength{\leftskip}{0em}%
\SetMCRule{
  custom-line={\path[WPP] (TOP) to [ornament=88] (BOT);}, % Use ornament 88
  width=0.4pt, % Ensure rule is drawn (width > 0pt)
  extend-top=-2pt, % Shrink rule at top by 24pt
  extend-bot=-8pt % Shrink rule at bottom by 8pt
}
\hspace*{2em}
\parbox{\dimexpr\textwidth-2em}{
\begin{multicols}{2}
\textbl{daňová záloha}
-- jedná se o část daně, která se platí ještě \texthl{před ukončením zdaňovacího období}\\
\textbl{plátce daně}
-- ten, \texthl{kdo daň odvádí státu} (u nepřímých daň vybírá, u přímých je zároveň poplatníkem)\\
\textbl{poplatník}
-- ten, \texthl{z jehož zdrojů} se daň \texthl{reálně odvádí} (u DPH zákazník, u daně z příjmů zaměstnanec)\\
\textbl{přeplatek na dani}
-- když státu \texthl{odvedeme víc}, než máme povinnost (například u moc vysokých záloh) a stát nám \texthl{přeplatek vratí}\\
\textbl{daňové přiznání}
-- oficiální dokument, kde je plátce \texthl{povinen} uvést \texthl{všechny skutečnosti} potřebných \texthl{k výpočtu daně}, lživé přiznání je \texthl{trestné}\\
\textbl{správce daně}
-- finanční úřad, který \texthl{vyměřuje výši} daně\\
\columnbreak\\
{\color{antiDP}$\faGreaterThan$}\textbl{předmět daně}
-- všechny příjmy \texthl{podléhající zdanění}, stejně tak \texthl{majetek} či poskytnuté \texthl{služby}\\
{\color{antiDP}$\faGreaterThan$}\textbl{osvobození z daně z příjmů fyzických osob}
-- úlevy stanovené zákonem, které \texthl{vyjímají plátce z povinnosti} daň platit\\
{\color{antiDP}$\faGreaterThan$}\textbl{zdaňovací období}
-- období, za které daň odvádíte, obvykle \texthl{jeden rok} nebo \texthl{jeden měsíc} (u DPH je možno nově i~\texthl{čtvrťletně})
\end{multicols}
}}
\vspace{2em}
\wbreak
\qaw{Daň z příjmu}{%
\textbl{termín pro podání přiznání}
-- daňové přiznání se podává do \texthl{31. 3.} dalšího roku, pokud má plátce daňového poradce, tak až do \texthl{30. 6.}\\
\textbl{kdo je poplatníkem}
-- každý, kdo stráví na území ČR \texthl{více než 183 dní} a má roční příjem \texthl{nad 50 000 Kč}; má pak daňovou povinnost k příjmům \texthl{z České Republiky} ale i \texthl{ze zahraničí}\\
\textbl{co je osvobozeno od daně z příjmu}
-- předmětem daně nejsou \texthl{příjmy z dědictví}, \texthl{darů}, \texthl{úvěrů} a \texthl{půjček}; osvobozeny od daní jsou \texthl{příjmy z nemocenského pojištění}, \texthl{sociálních dávek} a \texthl{movitých věcí}; prodej nemovitostí je osvobozen, pokuj je vlastníme déle než 2 roky (nebo peníze použijeme na nákup jiného bydlení)\\
\textbl{kdy mám povinnost uvést příjem nad pět milionů}
-- vždy, i když je vyjmut z daňové povinnosti, tak se \texthl{musí uvést} (prodej domu, výhra v loterii, kterou již zdanil provozovatel)\\
\textbl{slevy}
-- \texthl{na poplatníka}; \texthl{na manželku} (pod hranicí aktuálního příjmu); \texthl{na děti}; \texthl{ZTP}\\
\textbl{úlevy}
-- úlevy mohou vzniknout: z úroků \texthl{z hypotečního úroku}; \texthl{z dárcovství krve} a \texthl{plasmy} (pokud byla poskytnuta bez odměny); \texthl{z poskytnutí darů} za určitých podmínek
}
\qaw{Jaké možnosti podání daňového přiznání má OSVČ}{%
OSVČ musí podávat přiznání přes \texthl{Datovou Schránku}, má k tomu tři možnosti.\\
\begin{itemize}[label={\color{WPP}\faSlackHash}]
\item \textbl{paušální daň} -- pro neplátce DPH, jedná se o \texthl{kombinaci ZP, SP a daně z příjmů}, vhodná pro menší podnikatele
\item \textbl{výdajový paušál} -- pro základ daně použije \texthl{část svých příjmů} (20 \% -- 80 \% dle sektoru), nemusí tedy dokládat reálné výdaje, což zjednodušuje přiznání
\item \textbl{sledování příjmů a výdajů} -- OSVČ, který vede daňovou evidenci si spočítá daňový základ jako \texthl{rozdíl mezi výdaji a příjmy}, musí tedy řádně vést administrativu, ale platí jen to, co reálně musí
\end{itemize}
}
\qaw{Za jakých podmínek musím platit 23 \% daň z příjmu}{%
Pokud daňový základ přesuhuje \texthl{36násobek průměrné mzdy} (přibližně 1 800 000 Kč), kdy se 23 \% platí jen u příjmů, co hranici přesahuje.
}
\newpage
\qaw{Daň z nemovitosti}{%
\textbl{termín podání přiznání}
-- přiznání se podává k \texthl{31. 1.} roku, ve kterém vznika daňová povinnost; splatnost pak je k \texthl{31. 5.} stejného roku\\
\textbl{z čeho se počítá}
-- z velikosti pozemku, určenému užití pozemku, ale i počtu obyvatel v obci, či umístění nemovitosti\\
\textbl{co je předmětem daně}
-- nemovitosti, jakou jsou \texthl{pozemky} (stavební prostory, lesy, louky, vodní plochy), \texthl{stavby} (prostory k bydlení, nebytové prostory, ale i stodoly a další), \texthl{bytové jednotky} tvoří samostatnou kategorii;\\
předmětem daně jsou jen nemovitosti zapsané v katastru k 1. 1.;\\
výjimku z daně mají \texthl{památky} a nemovitosti sloužící \texthl{veřejnému zájmu} (jako jsou školy a nemocnice)\\
\textbl{kam plyne tato daň}
-- celá částka daně plyne přímo \texthl{do rozpočtů měst a obcí}, kde se nemovitost nachází
}
\qaw{Silniční daň}{%
\textbl{kdy se platí}
-- musí jí platit vlastníci nebo provozovatelé vozidel, co mají vozidlo zapsané k obchodní činnosti a vozidlo \texthl{přesahuje hmotnost 12 tun};\\
vypočítává se dle váhy a počtu náprav\\
\textbl{za který rok}
-- podává se do 31. 1. za rok minulý\\
\textbl{k čemu se používá}
-- hlavně k výstavbě, údržbě a opravě silniční infrastruktury
}
\qaw{Co je dorovnávací daň}{%
Daň minimálně 15 \% z příjmů na území ČR velkých firem, co zde jinak daně neplatí,  protože sídlo mají jinde.
Slouží k zamezení daňových úniků v daňových rajích.
Jedná se o českou verzi Pillar 2.
Má hned několik podmínek:\\
\begin{itemize}[label={\color{WPP}\faSlackHash}]
\item \textbl{minimálně výnosy} -- firma musí mít minimální roční \texthl{výnosy 750 milionů EUR} v alespoň ve 2 z posledních 4 let
\item \textbl{sídlo mimo ČR} -- firma musí mít sídlo mimo ČR a tedy zde ani neplatí klasické daně
\item \textbl{obchoduje na území ČR} -- dceřinná společnost, či jiný druh pobočky, zde musí mít kladné výnosy a z nich se odvádí daň (tedy ne z celkových výnosů)
\end{itemize}
}
\wbreak
\vspace{1em}
\setlength{\multicolsep}{0em}% Remove vertical space before and after
\setlength{\columnsep}{0em}% Remove space between columns
\begin{multicols}{2}
\qaw{DPH} specifická (knihy)\\
-- \texthl{12 \%} snížená (palivové dřevo, točené pivo, některé časopisy)\\
-- \texthl{21 \%} základní (zboží, služby, přeprava)\\
DPH tvoří \texthl{\textasciitilde 35 \% příjmů státu}.\\
\textbl{povinnost registrace k dani}
-- povinnost vzniká, když osoba překročí \texthl{obrat 2~000~000~Kč} během \texthl{12 po sobě jdoucích měsíců}.
Registrovat se pak musí do 15 dní po překročení.
Jinak je přihlášení k dani \texthl{dobrovolné}\\
\texthl{\textasciitilde 70 \% podnikatelů} je přihlášeno k DPH.\\
\textbl{podmínky pro odpočet DPH {\color{antiDP}(plátci {\color{WPP}X} neplátci)}}
-- {plátci} mají \texthl{nárok na odpočet DPH} na \textbf{vstupu}, musí k tomu vést evidenci; {neplátci nárok} na odpočet \texthl{nemají}, DPH je pro ně dalším nákladem\\
\textbl{co je přenesená daňová povinnost}
-- daň \texthl{odvede} sám \texthl{odběratel} namísto dodavatele; oba musí být přihlášeni k DPH
}
\columnbreak
\qaw{Spotřební daň -- z čeho se platí}{%
Tato daň se ukládá na komodity, které \texthl{negativně zatěžují} lidské \texthl{zdraví} nebo \texthl{prostředí}.
Vždy \texthl{pevnou částkou} na jednotku (\texthl{ks},~\texthl{kg},~\texthl{l}).
Patřím sem daně na:
\wemph{minerální oleje};
\wemph{pivo};
\wemph{víno};
\wemph{tabák}.
Tato daň \texthl{zvyšuje} základ výpočtu \texthl{DPH}.
Výrobci jí odvádějí měsíčně.
}
\subsection*{\faAngleDoubleRight Clo}
\begin{indentpara}
Clo je také formou daně, která se platí při \texthl{dovozu} nebo \texthl{vývozu} ze země.
\emph{Neplatí se mezi členskými státy EU.}
\end{indentpara}
\end{multicols}
\wtext{Časová osa splatností daní v roce 2024:}
\begin{scaletikzpicturetowidth}{\textwidth}
\begin{tikzpicture}[scale=\tikzscale, snake=zigzag, line before snake = 5mm, line after snake = 5mm]
    % draw horizontal line   
    \draw[ultra thick] (0,0) -- (12,0);

    % draw vertical lines
    \foreach \x in {0,6,12}
      \draw[ultra thick] (\x cm,6pt) -- (\x cm,-6pt);
      \foreach \x in {0.5,4,5.5,2.5,6.5,7.5,9.5}
      \draw[ultra thick, WPP] (\x cm,3pt) -- (\x cm,-3pt);

    % draw nodes
    \draw (0,0) node[below=7pt] {} node[above=7pt] {2024};
    
    \draw (1.7,0) node[below=4pt] {\rotatebox{-60}{\begin{varwidth}{7cm}\small k \texthl{31. 1.} povinost podat přiznání k \emph{dani z~nemovitosti}\end{varwidth}}} node[above=4pt] {};
    \draw[WPP, ultra thick] (4,0) -- (5.5,0) node[below=4pt] {} node[above=4pt] {};
    \draw (4.7,0) node[below=4pt] {} node[above=4pt] {\rotatebox{0}{\begin{varwidth}{5cm}\small \textbl{splatné měsíčně} \end{varwidth}}};
    \draw (6.4,0) node[below=4pt] {\rotatebox{-60}{\begin{varwidth}{7cm}\small \emph{DPH} se dá splácetale i \texthl{čtvrťletně}; podává se do \texthl{25. dne} po konci období \end{varwidth}}} node[above=4pt] {};
    \draw (4.8,0) node[below=4pt] {\rotatebox{-60}{\begin{varwidth}{2cm}\small \emph{spotřební daň} \end{varwidth}}} node[above=4pt] {};
    \draw (4.2,0) node[below=4pt] {\rotatebox{-60}{\begin{varwidth}{2cm}\small \emph{clo} \end{varwidth}}} node[above=4pt] {};
	\draw[antiDP, thick] (3.3,0) node[below=4pt] {\rotatebox{-60}{\begin{varwidth}{5cm}\small do \texthl{31. 5.} splatnost \emph{daně z~nemovitosti}\end{varwidth}}} node[above=4pt] {};
  
    \draw (6,0) node[below=7pt] {} node[above=7pt] {2025};
    
    \draw[antiDP, thick] (7.2,0) node[below=4pt] {\rotatebox{-60}{\begin{varwidth}{5cm}\small do \texthl{31. 1.} \emph{splatná silniční daň} \end{varwidth}}} node[above=4pt] {};
    \draw[antiDP, thick] (8.2,0) node[below=4pt] {\rotatebox{-60}{\begin{varwidth}{5cm}\small do {\color{WPP}31. 3.} \emph{splatná daň z příjmů}\end{varwidth}}} node[above=4pt] {};
    \draw[antiDP, thick] (10.3,0) node[below=4pt] {\rotatebox{-60}{\begin{varwidth}{8cm}\small do {\color{WPP}30. 6.} \emph{splatná daň z příjmů}\\ pokud ji zpracovává {\color{WPP}daňový poradce}\end{varwidth}}} node[above=4pt] {};
    
    \draw (12,0) node[below=7pt] {} node[above=7pt] {2026};

\end{tikzpicture}
\end{scaletikzpicturetowidth}

\wpage
\end{document}