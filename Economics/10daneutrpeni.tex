\documentclass[ekobook.tex]{subfiles}

\begin{document}
\lsection{Daňová soustava}
\qaw{Jaké jsou podmínky placení daní a kde je musíme sledovat}{%
\texthl{Výše} částky, \texthl{čas} splatnosti a \texthl{další podmínky} se neustále \texthl{mění}, proto se musí sledovat \texthl{v aktuálním znění zákonů} náležících k~daním.\\
Například v \href{https://www.e-sbirka.cz/}{\texthl{Elektronické Sbírce zákonů a mezinárodních smluv}}, jmenovitě pak
\href{https://www.e-sbirka.cz/sb/1992/586/2026-01-01}{\texthl{daň z příjmu}},
\href{https://www.e-sbirka.cz/sb/1993/16}{\texthl{silniční daň}} a
\href{https://www.e-sbirka.cz/sb/2003/353}{\texthl{spotřební daň}}.
}
\qaw{Vysvětli pojmy:}{%
\textbl{daňová záloha}
-- \\
\textbl{plátce daně}
-- \\
\textbl{poplatník}
-- \\
\textbl{přeplatek na dani}
-- \\
\textbl{daňové přiznání}
-- \\
\textbl{správce daně}
-- 
}
\qaw{Daň z příjmu}{%
\textbl{termín pro podání přiznání}
-- \\
\textbl{kdo je poplatníkem}
-- \\
\textbl{co je osvobozeno od daně z příjmu {\color{antiDP}(bydlení, dopravní prostředek)}}
-- \\
\textbl{kdy mám povinnost uvést příjem nad pět milionů}
-- \\
\textbl{slevy}
-- \\
\textbl{úlevy}
-- 
}
\qaw{Jaké možnosti podání daňového přiznání má OSVČ}{%
něco
}
\qaw{Za jakých podmínek musím platit 23 \% daň z příjmu}{%
něco
}
\qaw{Daň z nemovitosti}{%
\textbl{termín podání přiznání}
-- \\
\textbl{z čeho se počítá}
-- \\
\textbl{co je předmětem daně}
-- \\
\textbl{kam plyne tato daň}
-- 
}
\qaw{Silniční daň}{%
\textbl{nad kolik tun se platí}
-- \\
\textbl{za který rok}
-- \\
\textbl{k čemu se používá}
-- 
}
\qaw{Co je dorovnávací daň}{%
něco
}
\qaw{DPH}{%
\textbl{sazby DPH}
-- \\
\textbl{povinnost registrace k dani}
-- \\
\textbl{podmínky pro odpočet DPH {\color{antiDP}(plátci {\color{WPP}X} neplátci)}}
-- \\
\textbl{co je přenesená daňová povinnost}
-- 
}
\qaw{Spotřební daň -- z čeho se platí}{%
něco
}
\begin{scaletikzpicturetowidth}{\textwidth}
\begin{tikzpicture}[scale=\tikzscale, snake=zigzag, line before snake = 5mm, line after snake = 5mm]
    % draw horizontal line   
    \draw[ultra thick] (0,0) -- (17,0);

    % draw vertical lines
    \foreach \x in {0,5,13,17}
      \draw[ultra thick] (\x cm,6pt) -- (\x cm,-6pt);
      \foreach \x in {1,2,7,9,10}
      \draw[ultra thick, WPP] (\x cm,3pt) -- (\x cm,-3pt);

    % draw nodes
    \draw (0,0) node[below=7pt] {} node[above=7pt] {2024};
    
    \draw[WPP, ultra thick] (1,0) -- (2,0) node[below=4pt] {\begin{varwidth}{5cm}splatné měsíčně\end{varwidth}} node[above=4pt] {};
    
    \draw (5,0) node[below=7pt] {} node[above=7pt] {2025};
    
    \draw[WPP, thick] (7,0) node[below=4pt] {březen} node[above=4pt] {};
    \draw[WPP, thick] (9,0) node[below=4pt] {} node[above=4pt] {květen};
    \draw[WPP, thick] (10,0) node[below=4pt] {červen} node[above=4pt] {};
    
    \draw (13,0) node[below=7pt] {} node[above=7pt] {2026};
    \draw (17,0) node[below=7pt] {} node[above=7pt] {2027};

\end{tikzpicture}
\end{scaletikzpicturetowidth}

\vspace{\fill}
\wbreak
Účelem daňové soustavy je zajistit příjmy do státního rozpočtu, které se používají k zajištění chodu státu a jeho potřeb. Jako jsou státní správa a veřejný sektor (nemocnice, školy...).
\wpage
\end{document}