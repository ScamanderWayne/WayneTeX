\documentclass[ekobook.tex]{subfiles}

\begin{document}\IfDarkModeTF{\definecolor{WPP}{HTML}{FF8A8A}}{\definecolor{WPP}{HTML}{4D0000}}
\lsection{Daňová soustava \DORN}
\vspace{-2em}
\wpgfbox{%
Účelem daňové soustavy je \texthl{zajistit příjmy} do státního rozpočtu, které se používají \texthl{k zajištění chodu} státu a jeho \texthl{potřeb}. Jako jsou \texthl{státní správa} a \texthl{veřejný sektor} (nemocnice, školy, ...).
}
\vspace{-1em}
\qaw{Jaké jsou podmínky placení daní a kde je musíme sledovat}{%
\texthl{Výše} částky, \texthl{čas} splatnosti a \texthl{další podmínky} se neustále \texthl{mění}, proto se musí sledovat \texthl{v aktuálním znění zákonů} náležících k~daním.\\
Například v \href{https://www.e-sbirka.cz/}{\texthl{Elektronické Sbírce zákonů a mezinárodních smluv}}, jmenovitě pak
\href{https://www.e-sbirka.cz/sb/1992/586/2026-01-01}{\emph{daň z příjmu}},
\href{https://www.e-sbirka.cz/sb/1993/16}{\emph{silniční daň}} a
\href{https://www.e-sbirka.cz/sb/2003/353}{\emph{spotřební daň}}.
}
%\vspace{-2em}
\wpgfboxed{%
 \subsection*{\faAngleDoubleRight Dělení daní}
\begin{enumerate}[leftmargin=2em,label={\color{WPP}\textbf{\arabic*.}}]
\item \textbf{přímé daně} -- poplatník a plátce jsou jednou a tou samou osobou
\begin{enumerate}[leftmargin=1em,label={\color{WPP}\textbf{\alph*)}}]
\item \textbf{majetkové} -- \emph{daň z nemovitosti}, \emph{daň silniční}
\item \textbf{důchodové} -- \emph{daň z příjmů fyzických osob} a také \emph{daň z příjmu právnických osob}
\end{enumerate}
\item \textbf{nepřímé daně} -- poplatník není zároveň plátcem; kupříkladu \emph{daň z přidané hodnoty}
\end{enumerate}
}{0.9}{center}
%\vspace{-1em}
\qaw{Vysvětli pojmy:}{%
\setlength{\multicolsep}{0pt}% Remove vertical space before and after
\setlength{\columnsep}{2em}% Remove space between columns
\setlength{\leftskip}{0em}%
\SetMCRule{
  custom-line={\path[WPP] (TOP) to [ornament=88] (BOT);}, % Use ornament 88
  width=0.4pt, % Ensure rule is drawn (width > 0pt)
  extend-top=0pt, % Shrink rule at top by 24pt
  extend-bot=0pt % Shrink rule at bottom by 8pt
}
\hspace*{2em}
\parbox{\dimexpr\textwidth-2em}{
\begin{multicols}{2}
\textbl{daňová záloha}
-- jedná se o část daně, která se platí ještě \texthl{před ukončením zdaňovacího období}\\
\textbl{plátce daně}
-- ten, \texthl{kdo daň odvádí státu} (u nepřímých daň vybírá, u přímých je zároveň poplatníkem)\\
\textbl{poplatník}
-- ten, \texthl{z jehož zdrojů} se daň \texthl{reálně odvádí} (u DPH zákazník, u daně z příjmů zaměstnanec)\\
\textbl{přeplatek na dani}
-- když státu \texthl{odvedeme víc}, než máme povinnost (například u moc vysokých záloh) a stát nám \texthl{přeplatek vratí}\\
\textbl{daňové přiznání}
-- oficiální dokument, kde je plátce \texthl{povinen} uvést \texthl{všechny skutečnosti} potřebných \texthl{k výpočtu daně}, lživé přiznání je \texthl{trestné}\\
\textbl{správce daně}
-- finanční úřad, který \texthl{vyměřuje výši} daně\\
\columnbreak\\
{\color{antiDP}$\faGreaterThan$}\textbl{předmět daně}
-- všechny příjmy \texthl{podléhající zdanění}, stejně tak \texthl{majetek} či poskytnuté \texthl{služby}\\
{\color{antiDP}$\faGreaterThan$}\textbl{osvobození z daně z příjmů fyzických osob}
-- úlevy stanovené zákonem, které \texthl{vyjímají plátce z povinnosti} daň platit\\
{\color{antiDP}$\faGreaterThan$}\textbl{zdaňovací období}
-- období, za které daň odvádíte, obvykle \texthl{jeden rok} nebo \texthl{jeden měsíc} (u DPH je možno nově i~\texthl{čtvrťletně})
\end{multicols}
}}
\qaw{Daň z příjmu}{%
\textbl{termín pro podání přiznání}
-- \\
\textbl{kdo je poplatníkem}
-- \\
\textbl{co je osvobozeno od daně z příjmu {\color{antiDP}(bydlení, dopravní prostředek)}}
-- \\
\textbl{kdy mám povinnost uvést příjem nad pět milionů}
-- \\
\textbl{slevy}
-- \\
\textbl{úlevy}
-- 
}
\qaw{Jaké možnosti podání daňového přiznání má OSVČ}{%
něco
}
\qaw{Za jakých podmínek musím platit 23 \% daň z příjmu}{%
něco
}
\qaw{Daň z nemovitosti}{%
\textbl{termín podání přiznání}
-- \\
\textbl{z čeho se počítá}
-- \\
\textbl{co je předmětem daně}
-- \\
\textbl{kam plyne tato daň}
-- 
}
\qaw{Silniční daň}{%
\textbl{nad kolik tun se platí}
-- \\
\textbl{za který rok}
-- \\
\textbl{k čemu se používá}
-- 
}
\qaw{Co je dorovnávací daň}{%
něco
}
\qaw{DPH}{%
\textbl{sazby DPH}
-- \\
\textbl{povinnost registrace k dani}
-- \\
\textbl{podmínky pro odpočet DPH {\color{antiDP}(plátci {\color{WPP}X} neplátci)}}
-- \\
\textbl{co je přenesená daňová povinnost}
-- 
}
\qaw{Spotřební daň -- z čeho se platí}{%
něco
}
\begin{scaletikzpicturetowidth}{\textwidth}
\begin{tikzpicture}[scale=\tikzscale, snake=zigzag, line before snake = 5mm, line after snake = 5mm]
    % draw horizontal line   
    \draw[ultra thick] (0,0) -- (13,0);

    % draw vertical lines
    \foreach \x in {0,3,11,13}
      \draw[ultra thick] (\x cm,6pt) -- (\x cm,-6pt);
      \foreach \x in {1,2,5,6,8}
      \draw[ultra thick, WPP] (\x cm,3pt) -- (\x cm,-3pt);

    % draw nodes
    \draw (0,0) node[below=7pt] {} node[above=7pt] {2024};
    
    \draw[WPP, ultra thick] (1,0) -- (2,0) node[below=4pt] {} node[above=4pt] {};
    \draw (1.5,0) node[below=4pt] {\begin{varwidth}{5cm}splatné měsíčně\end{varwidth}} node[above=4pt] {};
    
    \draw (3,0) node[below=7pt] {} node[above=7pt] {2025};
    
    \draw[antiDP, thick] (5,0) node[below=4pt] {\begin{varwidth}{5cm}do {\color{WPP}31.3.} \emph{splatná daň z příjmů}\\ za rok {\color{WPP}2024} pro podnikatele\end{varwidth}} node[above=4pt] {};
    \draw[WPP, thick] (6,0) node[below=4pt] {} node[above=4pt] {květen};
    \draw[antiDP, thick] (8.5,0) node[below=4pt] {\begin{varwidth}{4cm}do {\color{WPP}30.6.} \emph{splatná daň z příjmů} pokud ji zpracovává {\color{WPP}daňový poradce}\end{varwidth}} node[above=4pt] {};
    
    \draw (11,0) node[below=7pt] {} node[above=7pt] {2026};
    \draw (13,0) node[below=7pt] {} node[above=7pt] {2027};

\end{tikzpicture}
\end{scaletikzpicturetowidth}

\wpage
\end{document}