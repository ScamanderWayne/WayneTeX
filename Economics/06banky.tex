\documentclass[ekobook.tex]{subfiles}

\begin{document}\IfDarkModeTF{\definecolor{WPP}{HTML}{39DDD5}}{\definecolor{WPP}{HTML}{0A3D3B}}
\lsection{Bankovní systém}
\qaw{Jaký je v ČR bankovní systém}{%
Dvoustupňový bankovní systém
\vspace{1em}
\begin{wnumbered}
\setlength{\leftmargini}{\dimexpr\leftmargini + 2em\relax} % Shift list text
\setlength{\itemindent}{2em} % Shift labels to match indentpara
\item Česká Národní Banka
\item ostaní banky
\end{wnumbered}
}
%ČNB 2-6
%\noindent\rule{\textwidth}{1pt}\vspace{2pt}
%\noindent{\color{WPP}\rule{\textwidth}{2pt}}
\wbreak
\begin{multicols}{2}
\qaw{ČNB}{%
\begin{itemize}
\item sídlo \texthl{v Praze}
\item vznikla \texthl{1. 1. 1993}\\
(po rozpadu České a Slovesnké Federativní Republiky -- ČSFR)
\item nejvyšším orgánem \texthl{bankovní rada}
\item cílem je \texthl{udržovat inflaci} a zabezpečit \texthl{stabilitu měny}
\item zřízena zákonem \texthl{O České národní bance}
\end{itemize}
}
\qaw{Čím je zajištěna nezávislost ČNB}{%
Nezávislot spočívá danou existencí \texthl{v ústavě}, do které je možno zasahovat pouze \texthl{zákony}.
Bankovní radu tvoří \texthl{7 členů}, které jmenuje sám \texthl{prezident České Republiky} a odvolat jdou jen z důvodů daných zákonem.
Nesmějí přijímat rozkazy od vlády.
}
\qaw{Vyjmenuj orgány ČNB}{%
Nejvyšším orgánem je sedmičlenná \texthl{bankovní rada}, jmenována prezidentem \texthl{na šest let}, kde žádný z členů nesmí zastávat funkci více \texthl{než dvakrát}.\\
Nejvyšším představitelem je \textbl{guvernér}.
Dále radu tvoří \textbl{dva víceguvernéři} a \textbl{čtyři další členové} s vysokou odborností.
}
%\columnbreak
\qaw{Jaké funkce má ČNB}{%
\begin{wnumbered}
\item \texthl{Cenová stabilita}\\
svými procesy ovlivňuje inflaci a tím pomáhá stabilitě koruny
\item \texthl{Emise hotových peněz}\\
jako jediná banka má právo tisknout více peněz a zároveň je může stahovat z oběhu
\columnbreak
\item \texthl{Řízení peněžního oběhu, platebního styku a zúčtování bank}\\
každá banka má účet u ČNB a mezibankovní převody jsou tedy spravovány právě ČNB
\item \texthl{Bankovní dohled}\\
sleduje činnost bank, spravuje jejich licence k činnosti
\item \texthl{Ostatní činnosti}\\
vede účty bankám, poskytuje úvěry, spravuje účet státu (a jeho dluh)
\end{wnumbered}
}
\qaw{Jaké jsou nástroje ČNB}{%
\begin{wnumbered}
\item \texthl{Diskontní sazba}\\
sazba, za kterou si obchodní banky mohou ukládat peníze
\item \texthl{Repo sazba}\\
základní úroková sazba pro termínované vklady a cenné papíry
\begin{align*}
\text{nízká inflace}\-\ &{\color{WPP}\textbf{\Leftrightarrow}}\-\ \text{vysoká inflace}\\
\text{ČNB sníží sazbu}\-\ &{\color{WPP}\textbf{\Leftrightarrow}}\-\ \text{ČNB ji naopak zvýší}
\end{align*}
\item \texthl{Lombardní úvěr}\\
doplňková sazba pro poskytování úvěrů bankám -- movité věci zástavou
\item \texthl{Povinné minimální rezervy}\\
určité \% z vkladů klientů u bank musí být uloženo u samotné ČNB
\item \texthl{Pravidla likvidity}\\
určuje strukturu aktiv a pasiv
\item \texthl{Operace na volném trhu}\\
ČNB obchoduje s cennými papíry
\end{wnumbered}
}
\end{multicols}
\wbreak
%KONEC ČNB 7-
\qaw{Vyjmenuj univerzální a specializované druhy bank}{%
\textbl{Univerzální $\Rightarrow$}
mají \texthl{plnou licenci} od ČNB, poskytují \texthl{širokou škálu služeb všem skupinám} zákazníků.
Např. ČSOB, Česká Spořitelna, KB atd.\\
\textbl{Specializované $\Rightarrow$}
specializují se na \texthl{určitý typ služeb, území, klientely}.
Například Hypoteční Banka atd.
}
\qaw{Jaké služby nabízejí komerční banky}{%
\begin{wnumbered}
\item \textbl{Vkladové služby}
\begin{itemize}[label=\textcolor{WPP}{$\hookrightarrow$}]
\item bankovní (bankovní účet) $\rightarrow$
Soustřeďuje netermínované vklady, skládají se peníze v hotovosti i~bezhotovostní.
Disponuje s ním pouze \texthl{majitel účtu}, podpis vzor, trvalé, hromadné i jednorázové příkazy.
Platební karty k tomuto účtu.
Většinou \texthl{nízký úrok}.
\item vkladní knížka $\rightarrow$
Ukládání a vybírání peněz po \texthl{předložení knížky}.
Může být výpovědní doba.
\item termínovaný vklad $\rightarrow$
Váže klientův \texthl{vklad na určitou dobu}, čím vyšší doba splatnosti, tím \texthl{vyšší úrok}.
Po uplynutí je vydána částka i s úrokem.
\item devizový účet $\rightarrow$
Je veden ve volně směnitelné \texthl{zahraniční měně}.
\item stavební spoření $\rightarrow$
Klient \texthl{pravidelně spoří} určitou částku, po dobu \texthl{minimálně šesti let}.
Má nárok na \texthl{příspěvek od státu}, maximálně \texthl{tři tisíce korun} ročně.
Může zažádat i o \texthl{stavební úvěr}.
\end{itemize}\newpage
\item \textbl{Úvěry}\\
Dle doby splatnosti:
\begin{itemize}[label=\textcolor{WPP}{$\hookrightarrow$}]
\item krátkodobé (do 1 roku)
\item střednědobé (do 4 let)
\item dlouhodobé (nad 4 roky)
\end{itemize}
Dle zajištění:
\begin{itemize}[label=\textcolor{WPP}{$\hookrightarrow$}]
\item zajištěné\\
$\rightarrow$ zástavou \texthl{nemovitosti} či \texthl{movitosti} (odhad ceny, výpis z katastru, doklady o vlastnictví, nabyvatelský doklad)\\
$\rightarrow$ \texthl{ručitelem} nebo \texthl{zárukou} (zavazuje se, že v případě nesplácení dluhu ho zaplatí, může ručit i banka)
\item nezajištěné\\
$\rightarrow$ ničím se neručí
\end{itemize}
Dle způsobu poskytování:
\begin{itemize}[label=\textcolor{WPP}{$\hookrightarrow$}]
\item kontokorentní \texthl{$\rightarrow$} bankovní účet s \texthl{vlastními zdroji}, kde se ale dá \texthl{čerpat do mínusu}; v případě nezaplacení nastává \texthl{vysoký úrok}
\item eskontní \texthl{$\rightarrow$} banka \texthl{odkoupí cenný papír} před \texthl{lhůtou splatnosti}, prodejce \texthl{získá peníze} dřív, banka si srazí \texthl{diskont} a pak požaduje peníze od dlužníka
\item lombardní \texthl{$\rightarrow$} úvěr zajištěný \texthl{movitou věcí}, je \texthl{rizikovější}, než eskontní, takže je i \texthl{úrok vyšší}
\item hypoteční \texthl{$\rightarrow$} zástavou je \texthl{nemovitost}
\end{itemize}
\item \textbl{Ostatní bankovní služby}
\begin{itemize}[label=\textcolor{WPP}{$\hookrightarrow$}]
\item TODO
\end{itemize}
\end{wnumbered}
}
% 9-15
\wbreak
\begin{multicols}{2}
\qaw{Vysvětli rozdíl mezi trvalým příkazem a příkazem k úhradě}{%
\begin{align*}
\text{trvalý příkaz}\-\ &{\color{WPP}\textbf{\vartimes}}\-\ \text{příkaz k úhradě}\\
\text{trvale se opakuje}\-\ &{\color{WPP}\textbf{\vartimes}}\-\ \text{jednorázová platba}
\end{align*}
}
\qaw{Co je bankovní účet, devizový účet}{%
Bankovní účet je \texthl{klientův účet} na \texthl{jeho jméno }a v \texthl{národní měně}.
Devizový je \texthl{klientův účet} v \texthl{cizí měně}.
}
\qaw{Debetní x kreditní karta}{%
\begin{align*}
\text{debetní karta}\-\ &{\color{WPP}\textbf{\vartimes}}\-\ \text{kreditní karta}\\
\text{platební karta}\-\ &{\color{WPP}\textbf{\vartimes}}\-\ \text{forma úvěru, platby}\\
\text{k běžnému účtu\-\ } &{\-\hspace{1em}} \text{se bezúročně splácejí do 45 dní}
\end{align*}
}
\qaw{Vyjmenuj druhy šeků}{%
\texthl{Soukromý} --vystaven subjektem, který má účet u banky\\
\texthl{Bankovní} -- vystavuje jej banka, kvalitní a vždy krytý
}
\columnbreak
\qaw{Na přiloženém šeku udělej křížkování a vysvětli, co to znamená}{%
udělám \texthl{dvě vodorovné čáry} na šeku, může být vyplacen pouze na bankovní účet
}
\qaw{Směnka -- jak se dá napsat}{%
Dá se napsat na \texthl{jakýkoli podklad} (klidně i na tácek pod sklenici), musí být uvedeno, že \texthl{jde o směnku}.
}
\qaw{Směnka -- druhy}{%
\begin{itemize}
\item vlastní \texthl{$\rightarrow$} obsahuje slovo \textbl{zaplatím}
\item cizí \texthl{$\rightarrow$} obsahuje slovo \textbl{zaplaťte}
\item vista \texthl{$\rightarrow$} splatná \texthl{kdykoli} po \texthl{předložení}
\item lhůtní směnka vista \texthl{$\rightarrow$} po \texthl{předložení} začne běžet \texthl{lhůta splatnosti}
\end{itemize}
}
\end{multicols}
\newpage
\end{document}