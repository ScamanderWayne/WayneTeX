\documentclass[ekobook.tex]{subfiles}

\begin{document}
\wsection{Bankovní systém}
\qaw{Jaký je v ČR bankovní systém}{%
Dvoustupňový bankovní systém
\vspace{1em}
\begin{wnumbered}
\setlength{\leftmargini}{\dimexpr\leftmargini + 2em\relax} % Shift list text
\setlength{\itemindent}{2em} % Shift labels to match indentpara
\item Česká Národní Banka
\item ostaní banky
\end{wnumbered}
}
%ČNB 2-6
%\noindent\rule{\textwidth}{1pt}\vspace{2pt}
%\noindent{\color{WPP}\rule{\textwidth}{2pt}}
\vspace{-4em}
\wbreak
\vspace{-2em}
\begin{multicols}{2}
\qaw{ČNB}{%
\begin{itemize}
\item sídlo \texthl{v Praze}
\item vznikla \texthl{1. 1. 1993}\\
(po rozpadu České a Slovesnké Federativní Republiky -- ČSFR)
\item nejvyšším orgánem \texthl{bankovní rada}
\item cílem je \texthl{udržovat inflaci} a zabezpečit \texthl{stabilitu měny}
\item zřízena zákonem \texthl{O České národní bance}
\end{itemize}
}
\qaw{Čím je zajištěna nezávislost ČNB}{%
Nezávislot spočívá danou existencí \texthl{v ústavě}, do které je možno zasahovat pouze \texthl{zákony}.
Bankovní radu tvoří \texthl{7 členů}, které jmenuje sám \texthl{prezident České Republiky} a odvolat jdou jen z důvodů daných zákonem.
Nesmějí přijímat rozkazy od vlády.
}
\qaw{Vyjmenuj orgány ČNB}{%
Nejvyšším orgánem je sedmičlenná \texthl{bankovní rada}, jmenována prezidentem \texthl{na šest let}, kde žádný z členů nesmí zastávat funkci více \texthl{než dvakrát}.\\
Nejvyšším představitelem je \textbl{guvernér}.
Dále radu tvoří \textbl{dva víceguvernéři} a \textbl{čtyři další členové} s vysokou odborností.
}
%\columnbreak
\qaw{Jaké funkce má ČNB}{%
\begin{wnumbered}
\item \texthl{Cenová stabilita}\\
svými procesy ovlivňuje inflaci a tím pomáhá stabilitě koruny
\item \texthl{Emise hotových peněz}\\
jako jediná banka má právo tisknout více peněz a zároveň je může stahovat z oběhu
\columnbreak
\item \texthl{Řízení peněžního oběhu, platebního styku a zúčtování bank}\\
každá banka má účet u ČNB a mezibankovní převody jsou tedy spravovány právě ČNB
\item \texthl{Bankovní dohled}\\
sleduje činnost bank, spravuje jejich licence k činnosti
\item \texthl{Ostatní činnosti}\\
vede účty bankám, poskytuje úvěry, spravuje účet státu (a jeho dluh)
\end{wnumbered}
}
\qaw{Jaké jsou nástroje ČNB}{%
\begin{wnumbered}
\item \texthl{Diskontní sazba}\\
sazba, za kterou si obchodní banky mohou ukládat peníze
\item \texthl{Repo sazba}\\
základní úroková sazba pro termínované vklady a cenné papíry
\begin{align*}
\text{nízká inflace}\-\ &{\color{WPP}\textbf{\Leftrightarrow}}\-\ \text{vysoká inflace}\\
\text{ČNB sníží sazbu}\-\ &{\color{WPP}\textbf{\Leftrightarrow}}\-\ \text{ČNB ji naopak zvýší}
\end{align*}
\item \texthl{Lombardní úvěr}\\
doplňková sazba pro poskytování úvěrů bankám -- movité věci zástavou
\item \texthl{Povinné minimální rezervy}\\
určité \% z vkladů klientů u bank musí být uloženo u samotné ČNB
\item \texthl{Pravidla likvidity}\\
určuje strukturu aktiv a pasiv
\item \texthl{Operace na volném trhu}\\
ČNB obchoduje s cennými papíry
\end{wnumbered}
}
\end{multicols}
\vspace{-4em}
\wbreak
\vspace{-2em}
%KONEC ČNB 7-
\qaw{Vyjmenuj univerzální a specializované druhy bank}{%
\textbl{Univerzální $\Rightarrow$}
mají \texthl{plnou licenci} od ČNB, poskytují \texthl{širokou škálu služeb všem skupinám} zákazníků.
Např. ČSOB, Česká Spořitelna, KB atd.\\
\textbl{Specializované $\Rightarrow$}
specializují se na \texthl{určitý typ služeb, území, klientely}.
Například Hypoteční Banka atd.
}
\qaw{Jaké služby nabízejí komerční banky}{%
\begin{wnumbered}
\item \texthl{Vkladové služby}
\begin{itemize}[label=\textcolor{WPP}{$\hookrightarrow$}]
\item bankovní $\rightarrow$
\end{itemize}
\item \texthl{Úvěry}
\begin{itemize}[label=\textcolor{WPP}{$\hookrightarrow$}]
\item krátkodobé $\rightarrow$
\end{itemize}
\item \texthl{Ostatní bankovní služby}
\begin{itemize}[label=\textcolor{WPP}{$\hookrightarrow$}]
\item něco $\rightarrow$
\end{itemize}
\end{wnumbered}
}
\qaw{Vysvětli rozdíl mezi trvalým příkazem a příkazem k úhradě}{%
Něco
}
\qaw{Co je bankovní účet, devizový účet}{%
Něco
}
\qaw{Debetní x kreditní karta}{%
Něco
}
\qaw{Vyjmenuj druhy šeků}{%
Něco
}
\qaw{Na přiloženém šeku udělej křížkování a vysvětli, co to znamená}{%
Něco
}
\qaw{Směnka -- jak se dá napsat}{%
Něco
}
\qaw{Směnka -- vlastní, cizí, vista, lhůtní směnka vista}{%
Něco
}
\newpage
\end{document}