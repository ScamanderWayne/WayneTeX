\documentclass[ekobook.tex]{subfiles}

\begin{document}\setWPP
\lsection{Národní hospodářství}
\begin{multicols}{2}
\qaw{Co je národní hospodářství}{%
jedná se o \texthl{souhrn} všech \texthl{činností} splňující hospodářský charakter \texthl{na území ČR}.
}
\qaw{Vyjmenuj 4 sektory NH a jejich příklady}{%
\begin{wnumbered}
\item \textbl{primární} -- základní získávání surovin, zemědělství, dřevořežba, těžařství ...
\item \textbl{sekundární} -- zpracovatelský průmysl, také stavební, výrobní ...
\item \textbl{terciární} -- doprava, pošta, pohostinství, peněžnictví, služby ...
\item \textbl{kvadrální} -- věda a výzkum
\end{wnumbered}
}
\qaw{Vysvětli pojmy:}{%
\begin{itemize}
\item \textbl{HDP}\\
$\hookrightarrow$ \texthl{Hrubý Domácí Produkt}
-- jedná se o \texthl{souhrn} statků a~služeb na \texthl{území ČR}
\columnbreak
\item \textbl{HNP}\\
$\hookrightarrow$ \texthl{Hrubý Národní Produkt}
-- jde o \texthl{souhrn HDP} sečteného s \texthl{tvorbou Čechů v zahraničí}
\item \textbl{ČDP}\\
$\hookrightarrow$ \texthl{Čistý Domácí Produkt}
-- od \texthl{HDP} odečteme \texthl{opotřebení kapitálu}
\end{itemize}
}
\qaw{Šedá ekonomika \textbl{x} černá ekonomika}{%
Do \texthl{šedé ekonomiky} patří například \texthl{podplácení} úředníků pro \texthl{získávání úlev}, či \texthl{státních zakázek}.
Do \texthl{černé ekonomiky} patří \texthl{přímé porušování zákonů}, prodej \texthl{ilegálních drog}, \texthl{praní peněz} z takových činností.
}\vspace{1em}
\end{multicols}
\wbreak
\begin{multicols}{2}
\qaw{Hospodářský cyklus}{%
Cyklus nemá přesně stanovenou dobu trvání, ale má tvar podobný sinusoidě.
Začíná se vždy na vrcholu, klesá se ke dnu daného cyklu (nikdy nedosáhneme nuly), pak zase začne růst k novému vrcholu a cykles začíná znovu.
\begin{wnumbered}
\item \textbl{Krize}\\
$\hookrightarrow$ \texthl{pokles}
-- \texthl{klesá poptávka} i \texthl{investice}, klesá tedy \texthl{výroba} a \texthl{zvyšuje} se \texthl{nezaměstnanost}, tím ještě více klesá poptávka.
\item \textbl{Deprese}\\
$\hookrightarrow$ \texthl{dno}
-- \texthl{HDP kleslo na minimum} v daném cyklu (nikdy ale není nulové), podniky vyhlašují bankrot, \texthl{nezaměstnanost} je \texthl{maximální}, naopak \texthl{výroba} zas \texthl{minimální}.
\item \textbl{Expanze}\\
$\hookrightarrow$ \texthl{oživení}
-- \texthl{roste výroba} a \texthl{klesá nezaměstnanost}, zvyšují se mzdy, tím i \texthl{roste poptávka}.
\item \textbl{Konjunktura}\\
$\hookrightarrow$ \texthl{vrchol}
-- výroba \texthl{dosahuje maxima} a \texthl{nezaměstanost minima}, HDP není jen vysoko, ale \texthl{zpravidla výš}, než bylo \texthl{před krizí},
lidé poptávají všechno, ale \texthl{trh} se rychle \texthl{přesytí}, tedy \texthl{cyklus začíná nanovo}.
\end{wnumbered}
}
\qaw{Vnější příčiny}{%
\begin{itemize}[label=\textcolor{WPP}{$\hookrightarrow$}]
\item \textbl{populační změny} -- při náhlém nárustu populace \texthl{vzniká} zvýšená \texthl{poptávka} po určitých druzích zboží, při \texthl{poklesu} populace je pak \texthl{trh přesycen} nadbytečným zbožím
\item \textbl{vynálezy a inovace} -- nové vrobky mohou rychle nahradit staré, vzniká tak \texthl{zvýšená poptávka} po \texthl{nedostatečné novotě} a je \texthl{nadbytek} nežádané \texthl{staroty}
\item \textbl{války a politické změny} -- omezí se \texthl{dovoz} nebo \texthl{vývoz zboží}, což ovlivní \texthl{stav trhu}, můžou nastat i \texthl{krize} spojené s typem \texthl{zboží}, kterého je tu \texthl{nedostatek} (například ropa)
\end{itemize}
}
\qaw{Vnitřní příčiny}{%
\begin{itemize}[label=\textcolor{WPP}{$\hookrightarrow$}]
\item \textbl{spotřeba} -- vývoj ekonomiky určuje, zda lidé \texthl{šetří} nebo \texthl{nakupují} zboží
\item \textbl{podnikové investice} -- s investicemi podniku \texthl{rostou} i pracovní \texthl{příležitosti}
\item \textbl{činnost státu} -- stát může ovlivnit vývoj \texthl{hospodářského cyklu} pomocí \texthl{rozpočtové} a \texthl{peněžní ekonomiky};
projevuje se v \texthl{daních}, pojištění v \texthl{nezaměstanosti}, státními výdaji \texthl{na výstavbu} ...
\end{itemize}
}
\end{multicols}
\wbreak
\begin{multicols}{2}
\qaw{Vysvětli:}{%
\textbl{inflace}
-- růst \texthl{cenové hladiny} za určité období (většinou ročně)\\
\textbl{deflace}
-- opak inflace, \texthl{nejedná se} ale \texthl{o vítaný jev}, protože populace \texthl{přestává věřit} v \texthl{hodnotu peněz}\\
\textbl{spotřební koš}
-- souhrn \texthl{675 výrobků} ze všech odvětví, jejichž cenu sledujeme (když se zvýší cena bot o 20 \%, ale klesne cena masa o 25 \%, tak se nejedná o inflaci)
}
\qaw{Druhy inflace}{%
\textbl{mírná} (do 10 \%)
-- tato míra je považována za zdravou, ekonomiku nijak negativně neovlivňuje\\
\textbl{pádivá} (do 100 \%)
-- tato míra je považována za škodlivou, může snižovat životní úroveň a zpomalovat ekonomiku\\
\textbl{hyperinflace} (nad 100 \%)
-- extrémně nebezpečná, může vést až k rozvratu ekonomiky a k návratu k směnnému obchodu
}
\qaw{Nezaměstnanost}{%
\texthl{míra nezaměstnanosti} = počet nezaměstnaných / celkový počet aktivního obyvatelstva x 100\\
\textbl{dobrovolná}
-- může pracovat, ale \texthl{nechce}\\
\textbl{nedobrovolná}
-- chce pracovat, ale nemůže \texthl{najít práci}\\
\textbl{frikční}
-- krátkodobá, vzniká \texthl{po odejití} z jedné práce \texthl{před nastoupením} do nové\\
\textbl{strukturální}
-- jedno \texthl{odvětví} pomalu \texthl{upadá}, pracovníci v~něm vyškoleni tedy \texthl{nemohou najít práci}, řešením je~\texthl{rekvalifikace} do vzrůstajícího odvětví\\
\textbl{cyklická}
-- vychází z hospodářského cyklu, jde o nezaměstnanost v období recese\\
\textbl{sezónní}
-- jedná se o práce, které \texthl{jde vykonávat} jen v \texthl{určitou sézónu}; prodavači zmrzlin, pracovníci na lyžařských střediskách ...
}
\qaw{Mezi ekonomicky aktivní obyvatelstvo nepatří}{%
důchodci, studenti, ženy v domácnosti, rentiéři (žijí z~úroků, nebo dědictví)
}
\qaw{SALDO OBCHODNÍ BILANCE}{%
Je \texthl{rozdíl} mezi \texthl{exportem} a \texthl{importem}.
Může být \texthl{kladná} (vývoz je vyšší než dovoz) nebo \texthl{záporná} (dovoz je vyšší než vývoz).
}
\end{multicols}
\wpage
\end{document}