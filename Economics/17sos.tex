\documentclass[ekobook.tex]{subfiles}

\begin{document}\setWPP
\lsection{Zdroje financování podniku \TODO}
\qaw{Vyjmenuj a popiš základní zdroje financování podniku}{%
Podle vztahu k podniku se dělí na vlastní a cizí.
}
{
\setlength{\multicolsep}{1em}% Remove vertical space before and after
\setlength{\columnsep}{2em}% Remove space between columns
\setlength{\leftskip}{0em}%
\SetMCRule{
  custom-line={\path[DP] (TOP) to [ornament=88] (BOT);}, % Use ornament 88
  width=0.4pt, % Ensure rule is drawn (width > 0pt)
  extend-top=-2pt, % Shrink rule at top by 24pt
  extend-bot=-2pt % Shrink rule at bottom by 8pt
}
%\hspace*{2em}
\parbox{\dimexpr\textwidth-2em}{
\begin{multicols}{2}
\begin{itemize}[label={\color{WPP}\faSlackHash}]
\item \textbl{Vlastní:}
\begin{itemize}[label=\textcolor{WPP}{$\hookrightarrow$}]
\item základní kapitál
\item kapitálové fondy
\item zisk
\item odpisy
\item ostatní
\end{itemize}
\columnbreak
\item \textbl{Cizí:}
\item[] jsou dluhem firmy, který musí splatit
\begin{itemize}[label=\textcolor{WPP}{$\hookrightarrow$}]
\item krátkodobé
\item střednědobé
\item dlouhodobé
\end{itemize}
\end{itemize}
\end{multicols}
}}
\wbreak
\qaw{Co je to úvěr}{%
Dluh
}
{
\setlength{\multicolsep}{1em}% Remove vertical space before and after
\setlength{\columnsep}{2em}% Remove space between columns
\setlength{\leftskip}{0em}%
\SetMCRule{
  custom-line={\path[DP] (TOP) to [ornament=88] (BOT);}, % Use ornament 88
  width=0.4pt, % Ensure rule is drawn (width > 0pt)
  extend-top=-2pt, % Shrink rule at top by 24pt
  extend-bot=-2pt % Shrink rule at bottom by 8pt
}
%\hspace*{2em}
\parbox{\dimexpr\textwidth-2em}{
\begin{multicols}{2}
\begin{itemize}[label={\color{WPP}\faSlackHash}]
\item \textbl{kontokorentní} -- umožňuje čerpat prostředky z běžného účtu do mínusu; je možný jen po smlouvě s bankou, která zároveň určuje limit čerpání; jedná se o úvěr s nejvyšší úrokovou sazbou
\item \textbl{směnečný eskontní} -- banka odkoupí směnku před dobou její splatnosti, za částku sníženou o diskont
\item \textbl{investiční} -- klient si sjedná úvěr na stroje, cenné papíry atd.; banka vyplácí peníze na základě předložených faktur
\item \textbl{provozní} -- pro financování běžného provozu, jako je nákup materiálů, vyplacení mezd
\columnbreak
\item \textbl{spotřebitelský} -- dělí se na \textbf{účelový} (na určitou věc) a \textbf{neúčelový} (s vyšším úrokem bez dokládání účelu)
\item \textbl{hypoteční} -- je zajištěn nemovitostí, často tou, na kterou si hypotéku bereme, mívá nejnižší úroky a nejdelší dobu splatnosti
\item \textbl{lombardní} -- je zajištěn movitou věcí (cennými papíry, zbožím...)
\end{itemize}
\end{multicols}
}}
\vspace*{1em}
\wbreak
\qaw{Co je to RPSN}{%
Roční Procentní Sazba Nákladů – ukazuje celkové roční náklady úvěru v procentech z půjčené částky (zahrnuje úrok + všechny poplatky, např. za sjednání, vedení účtu apod.) → čím nižší RPSN, tím výhodnější úvěr
}
\qaw{Jaké jsou způsoby ručení}{%
Úvěry se dle druhu ručení dělí ty, kde ručíme osobně (buď my sami, či ručitel za nás) a na ty, kde ručíme věcně (movitě či nemovitě).
}
\qaw{Co znamená vinkulace u úvěru}{%
Vinkulace (pojistného plnění) – banka (nebo leasingová společnost) si nechá „zavázat“ pojistku ve svůj prospěch. V praxi to znamená, že v případě pojistné události (např. požár, totální škoda na autě, smrt u životní pojistky) pojišťovna nevyplatí peníze vám, ale přímo bance. Banka z těchto peněz použije částku na splacení zbývajícího dluhu (úvěru/hypotéky/leasingu) a případný zbytek vrátí vám (nebo dědicům)
}
\wpage
\end{document}