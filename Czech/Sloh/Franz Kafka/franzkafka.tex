\documentclass{article}
\usepackage[czech]{babel}
\usepackage[enable]{darkmode}
\usepackage[left=2.5cm,right=2.5cm,top=2.5cm,bottom=2.5cm]{geometry}
\usepackage{nopageno}
\usepackage{yfonts}
\usepackage{hyperref}
\usepackage{fancyhdr}

\hypersetup{
	pdftitle={Franz Kafka},
	pdfauthor={Libor Halík},
	pdfsubject={Referát na Franze Kafku},
	pdfkeywords={Franz Kafka, literatura, čeština},
	pdfcreator={LaTeX with hyperref},
	pdfproducer={pdfLaTeX} % Or whatever engine you are using.
}

\pagestyle{fancy}
\fancyhf{}
\fancyhead[R]{Libor Halík}
\fancyhead[L]{Franz Kafka}
\fancyfoot[R]{\small Powered by \LaTeX.}

\begin{document}
	Franz Kafka byl německy píšící pražský spisovatel, považovaný za jednoho z nejvýznamnějších romanopisců 20. století. Narodil se 3. července 1883 v Praze, v tehdejším Rakousku-Uhersku, do zámožné židovské rodiny. Jeho otec, Hermann Kafka, byl obchodník, který měl na Franze silný vliv, často negativní. Toto napětí mezi otcem a synem se odráží v mnoha Kafkových dílech.
	
	Kafka studoval práva na Německé univerzitě v Praze a promoval v roce 1906. Po studiích pracoval jako úředník v pojišťovně, což mu poskytovalo jistotu, ale zároveň ho to omezovalo v jeho literárních ambicích. Práce v pojišťovně byla pro něj rutinní a byrokratická, což se stalo jedním z hlavních témat jeho tvorby.
	
	Ve svém volném čase se věnoval psaní beletrie. Jeho nejznámější díla zahrnují "Proměnu" (Metamorphosis), "Proces" (Der Prozess) a "Zámek" (Das Schloss). Jeho psaní je charakteristické svými tématy odcizení, úzkosti a byrokracie. Kafka často zobrazoval postavy, které se ocitají v absurdních a nepochopitelných situacích, což~vedlo ke vzniku pojmu "kafkaeskní".
	
	Franz Kafka zemřel na tuberkulózu 3. června 1924 ve vídeňském sanatoriu. Během svého života publikoval jen několik kratších děl. Většina jeho románů byla vydána posmrtně díky jeho příteli Maxu Brodovi, který~navzdory Kafkově přání jeho rukopisy nezničil.
	
	Jeho díla nebyla během jeho života široce publikována, ale po jeho smrti se stala velmi vlivnými. Kafka je považován za jednoho ze zakladatelů moderní literatury a jeho dílo ovlivnilo mnoho spisovatelů a umělců po celém světě. Jeho schopnost zachytit úzkost a odcizení moderního člověka je stále aktuální a~rezonuje s~čtenáři i v 21. století.
\end{document}