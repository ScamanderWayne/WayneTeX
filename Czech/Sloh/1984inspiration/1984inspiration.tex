\documentclass[11pt]{article}
\usepackage[czech]{babel}
\usepackage[a4paper]{geometry}
\usepackage{graphicx} % Required for inserting images
\usepackage{fancyhdr}
\usepackage{yfonts}
\usepackage{coffeestains}
\usepackage{lipsum}
\usepackage{xcolor}
\usepackage{csquotes}
\usepackage{hyperref}


\setlength{\parskip}{0pt}
\setlength{\parindent}{1em}
\setlength{\headheight}{11pt}
\setlength{\footskip}{5pt}

\pagestyle{fancy}
\fancyhf{}
\fancyhead[L]{Fuzzy noční můra}
\fancyhead[R]{{Libor Halík}}
% \fancyhead[C]{dekadence}
% \fancyfoot[C]{\footnotesize \textit{\thepage . strana}}
\renewcommand{\headrulewidth}{0.1pt}
% \renewcommand{\footrulewidth}{0.1pt}

\begin{document}

\setlength{\parindent}{2em}

% \coffeestainA{0.4}{1}{90}{28em}{38em}

Lidé mají tendenci se poměřovat. Takoví prostě jsou. Poměřují si své plány, své zážitky, své dovednosti. Rádi si sami sebe představují na číselné ose a každému úspěchu dávají určitou hodnotu. Však toto se lidé učí už od školy. Toto je důležitější, toto méně, toto jim šlo, toto naopak ne. Nemá cenu snažit se tento bodový systém zhodnotit. To mi zde nepřísluší. Ale lidé, lidé jej vnímají negativně. A~já~se~ptám. Proč? \par
Jistě, můžeme připustit, že stížnosti podávají hlavně ti, co do systému nezapadají. Co v něm prohrávají. A pár těch, co se jim zželí první skupiny. Je-li ale skupina nespokojených lidí majoritou, je na vině skutečně ona, či samotný systém? Dovolte mi zde uvést příklad. Noční můry. \par
Všichni lidé mají jednu, dvě, tři, někteří i dvaačtyřicet, někteří rádi tvrdí, že nulu. Ale jen málo lidí se shodne na objektu svých hrůz. Někdo se bojí krys, jiný pavouků, ta~zase stínů v rozích svého zorného pole, ten smrti, někdo zase naopak dlouhého života. A někdo má krysu za mazlíčka, pavouka v terárku, má v oblibě radši stinné prostory. Zatímco někdo si hledá důvody, kterými si odůvodní brzké znežití v dopise na rozloučenou, někdo jiný zase žádá o čas navíc. Sledujete kam tím mířím? Co pro jednoho největší hrůza, pro~jiného prkotinnou. Někdo se schoulí do klubíčka při existenční krizi, jiný při zmínce o~hadech. Jeden druhému by se vysmáli, jak jsou hrůzy toho druhého úplně zbytečné. \par
A teď kam s nimi na naši bodovou osu? Jste schopni říct, že je váš strach horší, než strach někoho jiného jen proto, že vás jeho strach neděsí? Když u svého bojácného človíčka vyvolá stejně silnou reakci, jako váš strach u vás? Nejde jim určit přesnou hodnotu, jen poměrovou. Nejdou umístit na bodovou osu. Nemůžete srovnat dva strachy a dát mezi ně znamínka menší, větší. Tak proč se přesně toto lidé snaží dělat se vším, co se odehrává v jejich životech už od nejbrzčejších dní?

\end{document}