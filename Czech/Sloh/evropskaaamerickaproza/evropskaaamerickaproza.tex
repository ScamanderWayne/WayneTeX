\documentclass{article}
\usepackage[czech]{babel}
\usepackage{geometry}
\geometry{a4paper, margin=2cm}
\usepackage{enumitem}
\usepackage{parskip}
\usepackage{nopageno}
\usepackage{xcolor}
\usepackage[enable]{darkmode}
\usepackage{mdframed}
\usepackage{lmodern}
\usepackage[fixed]{fontawesome5}

\definecolor{WPP}{HTML}{00FFFF} %cyan
\definecolor{WPP}{HTML}{0066CC} %blue
\definecolor{WPP}{HTML}{C9A12A} %gold
\definecolor{WPP}{HTML}{EC2B2B} %red

\setlist[itemize]{label=\textcolor{WPP}{$\hookrightarrow$}} %\textbullet \faGreaterThan

\definecolor{DP}{HTML}{293133}

\NewDocumentCommand{\wextbf}{m}{%
	{\vspace{-6pt}\hspace{-7pt}\noindent\colorbox{WPP}{\color{DP}\textbf{#1}}}
}

\NewDocumentCommand{\textbw}{m}{%
	{\color{WPP}{#1}}
}

\NewDocumentCommand{\wemph}{m}{%
	{\color{WPP}\emph{#1}}
}

\newcommand{\mysection}[2]{
	\setlength\fboxsep{4pt} % spacing around box contents
	\section*{\colorbox{DP}{\makebox[\dimexpr\textwidth-2\fboxsep\relax]{\color{WPP}#1\hfill#2}}}\vspace{-4pt}
}

\begin{document}
\fontsize{10pt}{10pt}\selectfont
\setlength\parindent{0em}
\mysection{Evropská a americká próza 1. poloviny 20. století}{Libor Halík}

\begin{mdframed}[backgroundcolor=DP,
	linecolor=WPP,
	linewidth=2pt,
	innertopmargin=5pt,
	innerbottommargin=5pt,
	innerleftmargin=5pt,
	innerrightmargin=5pt]\color{white}	
\wextbf{Definujte následující pojmy:}
	\begin{itemize}[leftmargin=*]\setlength\itemsep{-4pt}
	\item \textbw{Expresionismus} -- Umělecký směr, který klade důraz na vyjádření vnitřních emocí a subjektivních pocitů autora, často prostřednictvím deformace reality, symboliky a intenzivních obrazů.
	\item \textbw{Vnitřní monolog} -- Literární technika, která zachycuje proud myšlenek a vnitřních prožitků postavy, často bez přímé návaznosti na vnější děj, čímž odhaluje její psychiku.
	\item \textbw{Ztracená generace} -- Označení pro skupinu amerických spisovatelů 20. let 20. století, kteří vyjadřovali deziluzi a ztrátu ideálů po první světové válce (např. E. Hemingway, F. S. Fitzgerald).
	\item \textbw{Novela} -- Kratší prozaická forma s jedním hlavním dějovým motivem, zaměřená na dramatický konflikt a hutnou výstavbu příběhu.
	\item \textbw{Dystopie} -- Literární žánr zobrazující negativní, často totalitní nebo represivní společnost budoucnosti, která kritizuje společenské nedostatky.
	\end{itemize}
\end{mdframed}\vspace{-2pt}

\begin{mdframed}[backgroundcolor=DP,
	linecolor=WPP,
	linewidth=2pt,
	innertopmargin=5pt,
	innerbottommargin=5pt,
	innerleftmargin=5pt,
	innerrightmargin=5pt]\color{white}	
\wextbf{Charakterizujte jedno z děl G. Orwella.}\par
Román \wemph{1984} od George Orwella je dystopické dílo, které líčí totalitní společnost pod dohledem všemocného Velkého bratra. Hlavní hrdina,.Winston Smith, se vzpírá proti manipulaci, sledování a potlačování individuality. Dílo kritizuje totalitu, propagandistickou kontrolu pravdy a ztrátu osobní svobody, přičemž zkoumá moc jazyka a pravdy ve společnosti.
\end{mdframed}\vspace{-2pt}

\begin{mdframed}[backgroundcolor=DP,
	linecolor=WPP,
	linewidth=2pt,
	innertopmargin=5pt,
	innerbottommargin=5pt,
	innerleftmargin=5pt,
	innerrightmargin=5pt]\color{white}	
\wextbf{Uveďte podstatné motivy v povídce Proměna Franze Kafky.}\par
Hlavní motivy povídky \wemph{Proměna} zahrnují:
\begin{itemize}[leftmargin=*]\setlength\itemsep{-12pt}
	\item \textbw{Alienace} -- Proměna Gregora Samsy v brouka symbolizuje jeho odcizení od rodiny, práce i společnosti. \\
	\item \textbw{Rodinné vztahy} -- Napětí mezi Gregorem a jeho rodinou odhaluje sobeckost a závislost na jeho finanční podpoře. \\
	\item \textbw{Existenciální úzkost} -- Gregorova proměna zkoumá otázky smyslu života a lidské identity. \\
	\item \textbw{Odpovědnost a oběť} -- Gregorova role živitele rodiny a jeho postupná ztráta lidskosti. \\
	\item \textbw{Společenský tlak} -- Kritika očekávání společnosti a dehumanizace jedince.
\end{itemize}
\end{mdframed}\vspace{-2pt}

\begin{mdframed}[backgroundcolor=DP,
	linecolor=WPP,
	linewidth=2pt,
	innertopmargin=5pt,
	innerbottommargin=5pt,
	innerleftmargin=5pt,
	innerrightmargin=5pt]\color{white}	
\wextbf{Charakterizujte dílo A. de Saint Exupéryho.}\par
Dílo \wemph{Malý princ} od Antoina de Saint Exupéryho je filozofická pohádka, která kombinuje poetický styl s hlubokými myšlenkami o lidských vztazích, lásce a smyslu života. Příběh vypráví o setkání letce s Malým princem, který cestuje po planetách a zkoumá lidskou přirozenost. Dílo zdůrazňuje hodnotu přátelství, jednoduchosti a schopnosti dívat se na svět „srdcem“.
\end{mdframed}\vspace{-2pt}

\begin{mdframed}[backgroundcolor=DP,
	linecolor=WPP,
	linewidth=2pt,
	innertopmargin=5pt,
	innerbottommargin=5pt,
	innerleftmargin=5pt,
	innerrightmargin=5pt]\color{white}
\wextbf{Uveďte hlavní rysy románu Na západní frontě klid.}\par
Román \wemph{Na západní frontě klid} od Ericha Maria Remarqua je protiválečný román, jehož hlavní rysy zahrnují:
\begin{itemize}[leftmargin=*]\setlength\itemsep{-12pt}
	\item \textbw{Realismus} -- Detailní popis hrůz první světové války z pohledu mladého vojáka Paula Bäumera. \\
	\item \textbw{Antiválečná kritika} -- Zobrazení zbytečnosti a brutality války, dehumanizace vojáků. \\
	\item \textbw{Kamarádství} -- Síla přátelství mezi vojáky jako prostředek přežití. \\
	\item \textbw{Ztráta nevinnosti} -- Psychické a fyzické dopady války na mladou generaci. \\
	\item \textbw{Existenciální otázky} -- Hledání smyslu v chaotickém válečném prostředí.
\end{itemize}
\end{mdframed}\vspace{-2pt}

\begin{mdframed}[backgroundcolor=DP,
	linecolor=WPP,
	linewidth=2pt,
	innertopmargin=5pt,
	innerbottommargin=5pt,
	innerleftmargin=5pt,
	innerrightmargin=5pt]\color{white}	
\wextbf{Petr a Lucie.}\par
Román \wemph{Petr a Lucie} od Romaina Rollanda je lyrická novela o lásce dvou mladých lidí, Petra a Lucie, na pozadí první světové války. Jejich láska je konfrontována s tragédií války, což zdůrazňuje křehkost lidského štěstí a krásu v čase zkázy. Dílo je protiválečnou výpovědí a oslavou humanismu.
\end{mdframed}\vspace{-2pt}

\begin{mdframed}[backgroundcolor=DP,
	linecolor=WPP,
	linewidth=2pt,
	innertopmargin=5pt,
	innerbottommargin=5pt,
	innerleftmargin=5pt,
	innerrightmargin=5pt]\color{white}
\wextbf{Určete autory těchto děl:}
	\begin{itemize}[leftmargin=*]\setlength\itemsep{-4pt}
	\item \textbw{Zámek} -- Franz Kafka
	\item \textbw{O myších a lidech} -- John Steinbeck
	\item \textbw{Odysseus (Ulysses)} -- James Joyce
	\item \textbw{Noční let} -- Antoine de Saint Exupéry
	\item \textbw{Nebe nezná vyvolených} -- Erich Maria Remarque
	\end{itemize}
\end{mdframed}\vspace{-2pt}
\end{document}