\documentclass{article}
\usepackage[czech]{babel}
\usepackage{geometry}
\geometry{a4paper, margin=2cm}
\usepackage{enumitem}
\usepackage{parskip}
\usepackage{nopageno}
\usepackage{xcolor}
\usepackage{mdframed}
\usepackage{lmodern}
\usepackage[fixed]{fontawesome5}
\usepackage{lipsum}
\usepackage{csquotes}

\usepackage{tgbonum}
\usepackage{fontspec}
\setmainfont{Tex Gyre Termes}

\usepackage[enable]{darkmode}

\usepackage{hologo}

\usepackage{titlesec}
\titlespacing*{\section}{0pt}{0pt}{0pt}
\titlespacing*{\subsection}{0pt}{0pt}{0pt}

\IfDarkModeTF{%
\definecolor{DP}{HTML}{293133}%
%
\definecolor{antiDP}{HTML}{FFFFFF}%
%
\definecolor{WPP}{HTML}{00FFFF}%cyan
\definecolor{WPP}{HTML}{0066CC}%blue
\definecolor{WPP}{HTML}{C9A12A}%gold
\definecolor{WPP}{HTML}{EC2B2B}%red
\definecolor{WPP}{HTML}{39DDD5}%sort of cyan
}{%
\definecolor{DP}{HTML}{FFFFFF}%
%
\definecolor{antiDP}{HTML}{000000}%
%
\definecolor{WPP}{HTML}{702632}%lightmode wine
}

\setlist[itemize]{label=\textcolor{WPP}{$\hookrightarrow$}}%\textbullet \faGreaterThan

\NewDocumentCommand{\wextbf}{m}{%
	{\subsection*{\vspace{-6pt}\hspace{-7pt}\noindent\colorbox{WPP}{\color{DP}{#1}}}}%
}

\NewDocumentCommand{\texthl}{m}{%
	{\color{WPP}{#1}}%
}

\NewDocumentCommand{\textbl}{m}{%
	{\textbf{\color{WPP}{#1}}}%
}

\NewDocumentCommand{\wemph}{m}{%
	{\color{WPP}\emph{#1}}%
}

\NewDocumentCommand{\mysection}{m m}{%
	\setlength\fboxsep{4pt}% spacing around box contents
	\section*{\colorbox{DP}{\makebox[\dimexpr\textwidth-2\fboxsep\relax]{\color{WPP}#1\hfill#2}}}\vspace{-3pt}\par%
}

\NewDocumentCommand{\waybox}{m m}{%
\begin{mdframed}[backgroundcolor=DP,%
	linecolor=WPP,%
	linewidth=2pt,%
	innertopmargin=5pt,%
	innerbottommargin=5pt,%
	innerleftmargin=5pt,%
	innerrightmargin=5pt]\color{antiDP}%
	\wextbf{#1}\par%
	#2%
\end{mdframed}%
}

\NewDocumentCommand{\wayitem}{m m}{%
\begin{itemize}[leftmargin=*]\setlength\itemsep{#2}%
	#1%
\end{itemize}%
}

\NewDocumentCommand{\wtem}{m}{%
\item \texthl{#1} --%
}

\begin{document}\fontsize{10pt}{0pt}\selectfont%
\setlength{\parskip}{0pt}%
\setlength{\parindent}{2em}%
	\mysection{Pozapomenut v lese}{}
	\-\hspace{2em}Byl podvečer chladného podzimního dne, kdy se vzduch naplňoval svěží vlhkostí a tichým šuměním přírody.
	Blížil se večer, a s ním přicházela melancholická atmosféra, která obalovala krajinu do jemného šera.
	Kolem se rozprostíral jehličnatý les, jehož vysoké koruny borovic a smrků se tyčily k nebi jako strážci času.
	Jejich větve, hustě porostlé věčně zeleným jehličím, vytvářely spletitou klenbu, která jako by chránila lesní půdu před vnějším světem.
	Když jste vzhlédli k nebi, spatřili jste tisíce drobných větviček, jež se pyšnily svým jehličím, jež nikdy neopadává, a mezerami mezi nimi prosvítala šedá obloha, těžká a nasycená vlhkostí.\par
	Obloha, jako by cítila tíhu podzimu, se rozhodla prolít svůj nektar na zemi.
	Jemné kapky deště se tiše snášely, vytvářejíc neslyšitelnou symfonii, která se rozléhala lesem.
	Jehličí, jako přirozený štít, zachytávalo většinu těchto kapek, a~tak na lesní půdu dopadalo jen minimum vláhy.
	Půda, pokrytá vrstvou opadaného jehličí a mechu, voněla zemitou svěžestí, kterou déšť ještě více prohluboval.
	Kůra majestátních stromů, nyní tmavší a lesklá od vlhkosti, vypouštěla intenzivní vůni mízy, která se mísila s chladným vzduchem a vytvářela omamnou esenci podzimu.\par
	V dálce se ozývaly jemné zvuky lesa -- šelest větru, který proplétal větvemi, a~občasné zakrákání ptáka, který hledal úkryt před deštěm.
	Kapky se rozrážely o miliony jehliček, vytvářejíc tichou, ale rytmickou melodii, která jako by doprovázela pomalý tanec přírody.
	Mlha se pomalu zvedala od země, splývajíc s šedými odstíny oblohy, a dodávala lesu tajemný nádech.
	Každý strom, každý keř, každý kousek mechu vypadal, jako by byl součástí věčného koloběhu, který se v tomto chladném podvečeru zpomalil, aby vynikla jeho tichá krása.
	Les~dýchal klidem, a přesto byl plný života, který se ukrýval v drobných detailech -- v~kapkách rosy na jehličí, v~šepotu větru, v aroma vlhké kůry.
	\vfill
	\begin{flushright}
	\footnotesize Vypracoval \textbl{Libor Halík} v sázecím systému \wemph{\hologo{LuaLaTeX}}.
	\end{flushright}
\end{document}