\documentclass{article}
\usepackage[czech]{babel}
\usepackage[enable]{darkmode}
\usepackage[a4paper, left=2.5cm,right=2.5cm,top=2.5cm,bottom=2.5cm]{geometry}
\usepackage{nopageno}
\usepackage{yfonts}
\usepackage{hyperref}
\usepackage{fancyhdr}
\usepackage{xcolor}
\usepackage{tikz}
\usepackage{lmodern}
\usepackage{csquotes}

\usepackage{titlesec}
\titlespacing*{\section}{0pt}{0pt}{0pt}
\titlespacing*{\subsection}{0pt}{0pt}{0pt}
\titlespacing*{\paragraph}{0pt}{0pt}{0pt}

\usepackage{mdframed}

\NewDocumentCommand{\setws}{m m m}{%
	\def\wauthor{#1}%
	\def\wname{#2}%
	\def\wsubject{#3}%
}

\setws{Libor Halík}{Analýza uměleckého textu (Harry Potter)}{document subject}

\hypersetup{
	pdftitle={\wname},
	pdfauthor={\wauthor},
	pdfsubject={\wsubject},
	pdfkeywords={},
	pdfcreator={LaTeX with hyperref},
	pdfproducer={pdfLaTeX} % Or whatever engine you are using.
}

\definecolor{DP}{HTML}{293133}

\pagestyle{fancy}
\fancyhf{}
\fancyhead[R]{%
	\textsc{\IfDarkModeTF{%
				\color{cyan}\wauthor%
			}{%
				\wauthor%
			}%
}}%
\fancyhead[L]{%
	\textbf{\LARGE\IfDarkModeTF{%
			\colorbox{cyan}{\color{DP}\wname}%
		}{%
			\wname%
		}%
}}%
\fancyfoot[R]{%
	\IfDarkModeTF{%
		{\color{cyan}\small Powered by \LaTeX.}%
	}{%
		{\small Powered by \LaTeX.}%
	}%
}%

\renewcommand{\headrule}{}

\usepackage{calc}   % Required for \widthof or \dimexpr if not already loaded

\newcommand{\myoldsection}[1]{
	\setlength\fboxsep{4pt} %% spacing around box contents
	\section*{\colorbox{cyan}{\makebox[\textwidth]{\color{DP}#1\hfill}}}
}
\newcommand{\mysection}[1]{
	\setlength\fboxsep{4pt} % spacing around box contents
	\section*{\colorbox{cyan}{\makebox[\dimexpr\textwidth-2\fboxsep\relax]{\color{DP}#1\hfill}}}
}
\newcommand{\mysubsection}[1]{
	\setlength\fboxsep{4pt} %% spacing around box contents
	\subsection*{\colorbox{DP}{\makebox[\textwidth][l]{\color{cyan}#1\hfill}}}
}
\newcommand{\mysubsubsection}[1]{
	\setlength\fboxsep{3pt} %% spacing around box contents
	\paragraph{\colorbox{cyan}{\color{DP}#1}}\hspace{0.3em}
}

\begin{document}\fontsize{10pt}{10pt}\selectfont
	\begin{mdframed}[backgroundcolor=DP,linecolor=cyan,linewidth=2pt]
		\color{white}Velká síň byla ohromující.
		Tisíce svíček se vznášely ve vzduchu nad čtyřmi dlouhými stoly,
		u nichž seděli ostatní studenti.
		Stoly byly prostřeny zlatými talíři a poháry.
		Na druhém konci síně stál další dlouhý stůl,
		za nímž seděli učitelé.
		Strop Velké síně byl temně modrý a posetý hvězdami,
		takže vypadal jako noční obloha.
		„Je začarovaný, aby vypadal jako obloha venku,“ zašeptala Hermiona.
		„Četla jsem o tom v Dějinách Bradavic.“
		Bylo těžké uvěřit, že nad nimi je strop, a ne otevřené nebe.
		Harry se rychle rozhlédl po stolech.
		U jednoho z nich zahlédl ducha s hlavou nakřivo.
		„To je skoro bezhlavý Nick!“ zašeptal Ron.
		„Říkali o něm na koleji!“ skoro bezhlavý Nick, duch Nebelvíru,
		měl na sobě staromódní límec a dlouhý plášť.
		Jeho hlava se nebezpečně nakláněla na stranu,
		jako by měla každou chvíli spadnout.
	\end{mdframed}
	\vspace{3em}
	\mysection{1. část}	
	\mysubsection{Zasazení výňatku do kontextu díla:}
	Ukázka pochází z knihy Harry Potter a Kámen mudrců, prvního dílu sedmidílné série. V tomto bodě příběhu Harry Potter, jedenáctiletý chlapec, který se právě dozvěděl, že je čaroděj, přijíždí do Bradavic, školy čar a kouzel. Tato scéna se odehrává během uvítací hostiny po třídění nových studentů do kolejí. Jde o klíčový moment, kdy je čtenář poprvé uveden do magického prostředí Bradavic a seznamuje se s jejich atmosférou a pravidly.
	\mysubsection{Téma a motiv:}
	Hlavním tématem je objevování kouzelného světa a přechod z obyčejného života do světa magie. Mezi motivy patří kouzla (začarovaný strop, svíčky), přátelství (interakce mezi Harrym, Ronem a Hermionou), tajemství (duch skoro bezhlavého Nicka) a začlenění do nového prostředí. Motiv obdivu a úžasu nad kouzly je zde klíčový, protože odráží Harryho pohled nováčka.
	\mysubsection{Časoprostor:}
	Časově je scéna zasazena do současnosti z pohledu Harryho (konec 20. století, i když přesný rok není uveden). Prostorově se odehrává ve Velké síni Bradavic, která je popsána jako majestátní místo s magickými prvky (začarovaný strop, svíčky ve vzduchu). Prostor je uzavřený, ale díky iluzi noční oblohy působí otevřeně a kouzelně.
	\mysubsection{Kompoziční výstavba:}
	Výňatek je součástí většího vyprávění, které sleduje chronologický děj. Tato scéna slouží jako expozice prostředí Bradavic a představuje klíčové prvky světa (např. duchy, kouzla). Kompozice je lineární, s důrazem na popis prostředí a krátké dialogy, které prohlubují charakterizaci postav.
	\mysubsection{Literární druh a žánr:}
	Literární druh je epika, konkrétně próza. Žánrově jde o fantasy literaturu pro mládež s prvky dobrodružného románu. Přítomny jsou také prvky iniciačního příběhu, protože Harry prochází přechodem do nového světa.
	\mysection{2. část}	
	\mysubsection{Vypravěč:}
	Vypravěč je vševědoucí, v er-formě, s fokalizací na Harryho perspektivu. Popisuje prostředí a události zvenčí, ale reflektuje Harryho pocity a úžas nad magickým světem, což umožňuje čtenáři vžít se do jeho role.
	\mysubsection{Postava:}
	Hlavními postavami ve výňatku jsou Harry Potter (protagonista, nováček v kouzelnickém světě), Ron Weasley (jeho přítel, poskytuje informace o světě) a Hermiona Grangerová (vzdělaná a zvědavá, představuje racionální pohled). Vedlejší postavou je Nick Bezhlavý, duch, který přidává mystický prvek.
	\mysubsection{Vyprávěcí způsoby:}
	Převažuje popis (Velká síň, její vzhled, atmosféra), doplněný krátkým dialogem (Hermionin a Ronův šepot). Popis je detailní a evokuje magickou atmosféru. Vyprávění je v er-formě, bez přímého vnitřního monologu postav.
	\mysubsection{Typy promluv:}
	V textu je použita přímá řeč (Hermionino „Je začarovaný...“ a Ronovo „To je skoro bezhlavý Nick!“), která prohlubuje charakterizaci postav a poskytuje informace o světě. Promluvy jsou stručné a slouží k vysvětlení prostředí.
	\mysubsection{Veršová výstavba:}
	Text je psán v próze, takže veršová výstavba není přítomna.
	\mysection{3. část}	
	\mysubsection{Jazykové prostředky a jejich funkce:}
	\mysubsubsection{Popisné výrazy:} Slova jako „ohromující“, „tisíce svíček“, „temné modrý“ a „posetý hvězdami“ vytvářejí živý obraz Velké síně a zdůrazňují její kouzelnou atmosféru.
	\mysubsubsection{Přídavná jména:} Např. „staromódní“ (u Nickova límce) evokují historický ráz světa Bradavic.
	\mysubsubsection{Dynamické sloveso:} „Vznášely se“ (o svíčkách) podtrhuje magii prostředí.
	\mysubsubsection{Vyjadřovací prostředky:} Fráze „bylo těžké uvěřit“ odráží Harryho úžas a pomáhá čtenáři sdílet jeho perspektivu nováčka.
	\mysubsection{Tropy a figury a jejich funkce:}
	\mysubsubsection{Přirovnání:} „Vypadal jako noční obloha“ – strop Velké síně je přirovnán k nebi, což zdůrazňuje jeho kouzelnou iluzi a vytváří pocit nekonečna.
	\mysubsubsection{Personifikace:} „Jeho hlava se nebezpečně nakláněla“ – dodává skoro bezhlavý Nickovi humorný a zároveň strašidelný rys, což posiluje jeho výraznou charakteristiku.
	\mysubsubsection{Hyperbola:} „Tisíce svíček“ – nadsázka zdůrazňuje velkolepost Velké síně.
	\mysubsubsection{Metonymie:} „Dějiny Bradavic“ (místo názvu knihy) – stručně odkazuje na zdroj
\end{document}
