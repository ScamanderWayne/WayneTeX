\documentclass{article}
\usepackage[czech]{babel} % čeština
\usepackage[a4paper]{geometry} % A4 papír
\usepackage{fancyhdr} % custom záhlaví a zápatí
\usepackage{multicol} % multi columns
\usepackage{csquotes} % uvozovky

\setlength{\parskip}{1em} % mezi řádkové mezery
\setlength{\parindent}{0pt} % odsazení prvního řádku

\pagestyle{fancy}
\fancyhf{}
\fancyhead[C]{\Huge Oliwer Twist}
\fancyfoot[L]{\scriptsize Zdroje: \newline učebnice Literatury pro SOU od Děcicové c/o Dillia, 2022. \newline anglická Wikipedie \enquote{Oliver Twist} a \enquote{Charles Dickens}}

\begin{document}
\begin{itemize}
    \item [--] román Oliver Twist napsal Charles Dickens
    \item [--] hrdinou je mladý sirotek Oliver Twist, který se kvůli okolnímu prostředí vydává na špatnou cestu, ale za pomoci dobrých lidí se nakonec napraví
    \item [--] Charles Dickens zde bravurně popisuje prostředí sirotčince, ze kterého mladý Oliwer Twist pochází
    \item [--] skvělému vyobrazení se zde těší i londýnské podsvětí, ve kterém se Oliver pohybuje během příběhu
    \item [--] toto dílo autor nejprve vydával jako kratší stripsy do novin, následně je zorganizoval do tří dílné série a tu poskládal do jednoho bestselleru
    \item [--] dílo je i uznáváno pro popis života mladých chudáků, kdy zde nechybí zneužívání v sirotčincích a dětská robota
    \item [--] Dickens zde mísí realismus, při popisu života mladistvého siroty, se satirou
    \item [--] díky této kombinaci pak dokonale poukazuje na stav, v jakém anglická společnost byla v 19. století, stejně jako kritizuje tehdější postavení vlády vůči těmto problémům
    \item [--] pro realistické dílo se zde ale objevuje zmínění hodné množství symbolismu, kdy starý Fagin, jenž představuje londýnské podsvětí, má satanistické rysy
\end{itemize}
\hspace{1em}
{\LARGE
O autorovi
}
\begin{itemize}
    \item dle mnohých byl nejvlivnějším autorem \enquote{Viktoriánské éry}
    \item ve svých dvanácti letech byl okolnostmi donucen jít pracovat do továrny, na celé tři roky, tyto zkušenosti pak formovaly jeho další život
    \item stal se z něj novinář, kdy za své roky praxe napsal či zeditoval nespočet článků, také už navždy bojoval za dětská práva
    \item své práce jako žurnalisty pak využil při publikování své tvorby, kdy si ho svět začal rychle vážit pro jeho satiristické sklony a pozorování okolního dění
    \item tento styl publikování mu také přinesl další výhodu, že své příběhy mohl často pozměňovat po chuti svých diváků, takže vždy před vydáním další části věděl, co čtenáři chtějí
    \item prodeje a zájem pak zvyšoval díky svému stylu, kdy příběh vždy ukončil v tom nejnapínavějším
    \item mezi jeho nejpopulárnější díla patří Vánoční koleda, Oliver Twist či David Copperfield
\end{itemize}
\end{document}