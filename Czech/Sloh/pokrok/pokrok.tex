\documentclass{article}
\usepackage[czech]{babel}
\usepackage{darkmode}
\usepackage[a4paper, left=2.5cm,right=2.5cm,top=2.5cm,bottom=2.5cm]{geometry}
\usepackage{nopageno}
\usepackage{yfonts}
\usepackage{hyperref}
\usepackage{fancyhdr}
\usepackage{xcolor}
\usepackage{tikz}
\usepackage{lmodern}

\NewDocumentCommand{\setws}{m m m}{%
	\def\wauthor{#1}%
	\def\wname{#2}%
	\def\wsubject{#3}%
}

\setws{Libor Halík}{Pokrok}{V hlavní roli přístoje, úvahový text o tom, co lidstvo užíváním technologií získává a ztrácí}

\hypersetup{
	pdftitle={\wname},
	pdfauthor={\wauthor},
	pdfsubject={\wsubject},
	pdfkeywords={},
	pdfcreator={LaTeX with hyperref},
	pdfproducer={pdfLaTeX} % Or whatever engine you are using.
}

\definecolor{DP}{HTML}{293133}

\pagestyle{fancy}
\fancyhf{}
\fancyhead[R]{%
	\textsc{\small{\IfDarkModeTF{%
				\color{cyan}\wauthor%
			}{%
				\wauthor%
			}%
}}}%
\fancyhead[L]{%
	\textbf{\IfDarkModeTF{%
			\colorbox{cyan}{\color{DP}\wname}%
		}{%
			\wname%
		}%
}}%
\fancyfoot[R]{%
	\IfDarkModeTF{%
		{\color{cyan}\small Powered by \LaTeX.}%
	}{%
		{}%
	}%
}%

\begin{document}
	\fontsize{10pt}{10pt}\selectfont
	Každý den vyrazí do práce.
	Práce, která jej nebere jako individuum, nýbrž jako článek v řetězci.
	Vrací~se, bez většího uznání.
	Každý další den, se~to~opakuje, až do konce jeho života.
	Mnozí tuto noční můru dávají za vinu právě funkčnímu politickému režimu.
	A přitom toto byl popis života mravence.\par
	Většina živočišných druhů zná určitou formu pokroku.
	Ptáci, co objevili gravitační zrychlení a~s~tím~spojenou sílu úderu.
	Dřevěné nástroje pro lov hmyzu.
	Včely, které zase znají vliv proudění vzduchu a~staví podle něj své úly.
	I lidé, když se ještě nedali odlišit od opic, už znali spolupráci a dělbu práce.
	Také se~naučili používat nástroje.
	Zjistili, jak stvořit oheň.
	Své nástroje vylepšovali.
	A nyní už umí jejich nástroje pracovat téměř samostatně.\par
	Zavedli si pracovní dobu, nástroje, které zvládnou většinu práce, dohled nad pracovními odděleními, podíly v takovýchto institutech.
	A také zbraně.
	Kdykoli někdo přišel na silnější oheň, rychlejší těžbu kamení, lepší komunikaci, tak~první využití bylo pro válku.
	Práce se zlepšovala až sekundárně.
	Ale~také~zábavu.
	Stínohra, divadlo, film, hudba, hry.
	Kde medvěd obdivuje krásu přírody, člověk se naučil, jak si tuto krásu k sobě přiblížit.\par
	Někdo může stále snít svůj sen o individualismu, kdy je každý sám za sebe.
	Ale člověk nikdy nebyl sám za sebe.
	Už od prvotních věků jsme spolupracovali a zlepšovali se vzájemně.
	Člověk nyní ztrácí svobodu, kterou nikdy neměl.
	Má~pocit mrhání časem, který nikdy neměl.
	Vadí mu nesamostatnost, kdy~nikdy nebyl samostatný.\par
	Technologie jej omezuje, ale dává mu toho mnohem více.
	Technologie se~stává novým božstvem lidstva a to je součástí pokroku.
\end{document}
