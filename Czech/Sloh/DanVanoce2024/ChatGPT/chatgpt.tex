\documentclass{article}
\usepackage{wsloh}

\begin{document}
\setws{Jack Frost a zamrzlé Vánoce}{Daniel Kejval}

Na chladný Štědrý večer, vysoko na mrazivé obloze, se Jack Frost cítil poněkud… znuděně. Jako nezbedný duch zimy obvykle měl spoustu práce: maloval okna jemnými ledovými vzory, nechával sněhové bouře převalovat ospalými městy a zanechával za sebou cestičky mrazivých stop. Ale dnes večer? Dnes večer bylo všechno jinak. Svět pod ním působil klidně a široko daleko nebyla ani vločka sněhu. Jack Frost měl poprvé pocit, že si nemá s kým hrát.

Měsíc zářil jasně, vrhal stříbrný lesk na všechno kolem, a Jack se vznášel nad městečkem Pinebrook. Jeho ledový dech proměňoval stromy v třpytící se sochy. Jak se neslyšně vznášel nad tichými ulicemi, všiml si něčeho zvláštního: celé město bylo ponořené do tmy. V oknech nesvítila světla, domy byly temné a ani jediný vánoční stromek nezářil ozdobami. Jako by samotné Vánoce zamrzly v čase.

Zvědavý Jack se snesl dolů na náměstí, kde uviděl skupinku dětí. Stály tam sklesle, jejich hlavy byly svěšené a tváře bledé chladem. Tiše přistál vedle nich, a sníh pod jeho botami se zatřpytil ledovým leskem.

„Hej, vy tam!“ zavolal Jack hravým hlasem. „Copak to tu máte za dlouhé obličeje? Vždyť je Štědrý večer!“

Jedna z dětí, malá holčička v jasně červených rukavicích, zvedla hlavu a upřela na něj oči plné strachu a zmatku. „To nejsou Vánoce,“ zašeptala, jako by hlasitější slova mohla něco rozbít. „Nemůžeme je oslavit. Vánoční světla… jsou zamrzlá. Sníh nepadá. Je prostě… příliš velká zima.“

Jack Frost zvedl obočí. „Zamrzlá, říkáš?“ Rozhlédl se kolem a jeho ledový dech proměňoval stromy v třpytivé ledové sochy. „Myslel jsem, že já jsem ten, kdo přináší chlad… ale že bych dokázal zmrazit celé Vánoce? To jsem nevěděl!“

Jeden z chlapců s rozcuchanými vlasy dodal: „Každý rok dostává město vánoční kouzlo a všechno se rozzáří! Ale letos… kouzlo zmizelo. Zkusili jsme všechno, ale nic nefunguje. Jako by sezóna sama zapomněla, jak být veselá.“

Jackův obvyklý úsměv se mu roztáhl po tváři. Výzva. Záhada. A něco, čemu nemohl odolat. „No, vypadá to, že tu mám trochu námrazy k nápravě! Nebojte se, prckové. Já na to přijdu.“

Děti na něj upíraly nejisté pohledy. „Ale… ty jsi přece Jack Frost. Ty přinášíš jen zimu, ne?“ zeptal se jeden z nich třesoucím se hlasem.

Jack se zasmál – zvuk jeho smíchu připomínal vítr hvízdající mezi stromy. „To je jen část mé práce. Jsem ale taky duchem zimy. Znamená to, že mohu přinést chlad, jistě, ale mohu ho také zahnat. A myslím, že… je čas rozmrazit vaše Vánoce!“

Chlapec s rozcuchanými vlasy se zamračil. „A jak to chceš udělat?“

Jack mrkl a zatočil se ve vzduchu. „Jednoduše. Trochou kouzel, špetkou šibalství… a spoustou vánočního ducha.“

Natáhl ruce a vzduch kolem něj zchladl ještě víc. Země se zatřpytila, námraza zesílila a s bleskem stříbrného světla začal Jack broukat veselou melodii – pradávnou vánoční píseň, kterou mohl nést jen mráz. Tančil po náměstí a s každým krokem začaly z nebes padat sněhové vločky. Ale ne obyčejné vločky. Tyhle se třpytily, zářily radostí a jiskřily jako hvězdy.

Jak Jack tančil, vánoční světla na domech se rozzářila, jejich teplý svit se rozlil ulicemi. Zamrzlé stromy začaly světélkovat a jejich větve se pokryly čerstvým sněhem. Z kostelní věže se rozezněl zvuk vánočních zvonů a brzy bylo celé město zalité kouzlem Vánoc.

Děti úžasem zalapaly po dechu, oči jim zářily radostí. „Dokázal jsi to!“ vykřikla holčička v červených rukavicích.

Jack mrkl a zatočil se. „Samozřejmě, že ano. Ale vánoční kouzlo není jen o světýlkách, sněhu nebo dárcích. Je to o radosti, kterou sdílíme s ostatními.“ Luskl prsty a sněhové vločky se snesly k dětem, které se smály a tančily pod sněhem, teď už teplým od smíchu a štěstí.

Když město ožilo vánočními zvuky a světly, Jack Frost se vznášel nad tím vším a sledoval, jak kouzlo roste. Cítil ve vzduchu teplo Vánoc, teplo, které v jeho ledovém království často nezažíval. Smích dětí, hudba, oslava – jako by mráz v jeho srdci začal pomalu tát.

Městečko Pinebrook znovu ožilo, jasnější než kdy dřív, a to vše díky Jacku Frostovi, duchovi, který nejen přinesl chlad, ale dokázal zažehnout i teplo Vánoc, když to bylo nejvíce potřeba.

Když se děti shromáždily kolem vánočního stromu, jejich tváře zářily ve světle svíček a podívaly se na Jacka s úsměvem.

„Zůstaneš a oslavíš to s námi?“ zeptala se holčička s nadějí v hlase.

Jack Frost zaváhal. Jeho ledové prsty se lehce dotkly sněhu. Na okamžik ucítil něco zvláštního, něco jako teplo v hrudi. Mohl by se prostě vytratit do noci, tak jako vždy.

Ale pak se usmál – vzácným, upřímným úsměvem. „Myslím, že tentokrát zůstanu,“ řekl tichým hlasem, jemným jako zimní vánek. „Přece jen, ne každý den můžu být součástí vánočního kouzla.“

A tak, pro jednu kouzelnou vánoční noc, Jack Frost nepřinesl jen chlad – přinesl i ducha Vánoc. Zůstal, tančil a smál se s dětmi a poprvé pochopil, co dělá Vánoce skutečně kouzelnými: nebyl to mráz, sníh ani světýlka, ale teplo vzájemnosti, radost z dávání a láska, která zahřeje i tu nejchladnější noc.

Když Jack Frost zmizel s prvním světlem vánočního rána, zanechal po sobě poslední dárek – přikrývku třpytivého sněhu, připomínku toho, že i ta nejchladnější srdce mohou být zahřátá kouzlem Vánoc.

Konec.

\end{document}