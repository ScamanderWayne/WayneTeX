\documentclass{article}
\usepackage{wsloh}

\begin{document}
\setws{Jack Frost a zmrzlé Vánoce}{Daniel Kejval}

O chladném Štědrém večeru, daleko nahoře na mrazivé obloze, se Jack Frost poněkud... nudil. Jako rozpustilý duch zimy měl obvykle spoustu práce: maloval okna jemnými ledovými vzory, způsoboval sněhové bouře, které se proháněly ospalými městy, a zanechával za sebou stopy mrazivých stop. Ale dnes v noci? Dnešní noc byla jiná. Svět pod námi vypadal klidně a v dohledu nebyla ani vločka sněhu. Pro jednou měl Jack Frost pocit, že si nemá s kým hrát.

Měsíc nad ním jasně zářil, vrhal na všechno stříbrnou záři a Jack se vznášel nad městečkem Pinebrook a jeho ledový dech měnil stromy v třpytivé sochy. Když klouzal tichými ulicemi, všiml si něčeho zvláštního: celá vesnice byla temná. Žádná světla na domech, žádné lucerny v oknech, dokonce ani jediný vánoční stromek se netřpytil ozdobami. Vypadalo to, jako by samotné Vánoce zamrzly v čase.

Jack ze zvědavosti sletěl dolů na náves, kde uviděl shromážděnou skupinku dětí, které se tvářily opuštěně. Měly svěšené hlavy a tváře bledé zimou. Tiše přistál vedle nich a sníh pod jeho botami se třpytivě zaleskl.

„Ahoj!“ Jack zavolal hravým hlasem. „Co to máš za protáhlé obličeje? Je Štědrý večer!“

Jedno z dětí, mladá dívka se zářivě červenými rukavicemi, k němu vzhlédlo s očima rozšířenýma směsicí strachu a zmatku. „Nejsou Vánoce,“ zašeptala, jako by příliš hlasité mluvení mohlo něco rozbít. „Nemůžeme slavit. Vánoční světýlka... jsou zamrzlá. Sníh nepadá. Je prostě... příliš velká zima.“

Jack Frost zvedl obočí. „Říkáš zmrzlé?“ Rozhlédl se kolem a jeho mrazivý dech proměnil stromy v třpytivé ledové sochy. „Přísahal bych, že já jsem ten, kdo dělá věci studené... ale nikdy jsem nevěděl, že dokážu zmrazit samotné Vánoce!“

Jeden z chlapců s rozcuchanými vlasy dodal: „Každý rok město dostane své vánoční kouzlo a všechno se rozzáří! Ale letos je kouzlo pryč. Zkoušeli jsme všechno, ale... nic nefunguje. Jako by to období samo zapomnělo, jak být veselé.“

Jackovi se na tváři rozlil obvyklý úsměv. Výzva. Záhada. A té nemohl odolat. „No, vypadá to, že musím napravit nějaký mráz! Nebojte se, děcka. Já to vyřeším.“

Děti na něj zíraly, tvářily se stále nejistě. „Ale... ty jsi Jack Frost. Nepřinášíš jenom mráz?“ zeptalo se jedno z nich a hlas se mu chvěl.

Jack se zasmál, zvuk připomínal vítr svištící mezi stromy. „To k tomu prostě patří. Ale já jsem také duch zimy. To znamená, že dokážu přinést chlad, to jistě, ale taky ho dokážu rozpustit. A já si říkám... že je čas rozmrazit vaše Vánoce!“

Chlapec s rozcuchanými vlasy se zamračil. „Jak to chceš udělat?“

Jack mrkl a zatočil se ve vzduchu. „Jednoduše. S trochou kouzel, trochou rošťáren... a spoustou vánoční nálady.“

Rozpřáhl ruce a vzduch kolem něj se ochladil. Země zajiskřila, mráz zhoustl a Jack si v záblesku stříbrného světla začal pobrukovat veselou melodii - prastarou vánoční melodii, kterou dokázal unést jen mráz. Tančil po náměstí a s každým krokem se z nebe začaly snášet sněhové vločky. Ne však jen tak obyčejné sněhové vločky. Byly to jiskřivé, zářící vločky, které vířily radostí a třpytily se jako hvězdy.

Jak Jack tančil, vánoční světýlka podél domů ožila a jejich teplá záře se rozlévala po ulicích. Zmrzlé stromy začaly bzučet světlem, jejich větve pokrývala peřina čerstvého sněhu. Z kostelní věže se rozezněly vánoční zvony a brzy se celá vesnice koupala v kouzelné záři tohoto ročního období.

Děti v úžasu zalapaly po dechu a oči měly doširoka otevřené radostí. „Dokázali jste to!“ jásala dívka v červených rukavicích.

Jack mrkl a zavrtěl se. „Samozřejmě, že ano. Ale kouzlo Vánoc není jen o světýlkách, sněhu nebo dárcích. Je to o radosti, kterou sdílíme s ostatními.“ Švihl prsty a poslal k dětem smršť sněhových vloček. Ty se chichotaly a tančily pod sněhem, zima je teď hřála smíchem a štěstím.

Zatímco vesnice ožívala vánočními pohledy a zvuky, Jack Frost se vznášel nad ní a sledoval, jak se kouzlo rozvíjí. Ve vzduchu cítil teplo svátků, teplo, které ve své ledové říši často nezažíval. Smích dětí, hudba, oslavy, jako by mráz v jeho srdci pomalu roztával.

Městečko Pinebrook znovu ožilo, rozzářilo se víc než kdy dřív, a to všechno díky Jacku Frostovi, duchovi, který nejenže přinášel chlad, ale uměl zažehnout teplo Vánoc, když ho bylo nejvíc potřeba.

Když se děti shromáždily kolem vesnického stromečku a jejich tváře zářily ve vánočním světle, s úsměvem se obrátily na Jacka.

„Zůstaneš a budeš s námi slavit?“ zeptala se dívka s nadějí v hlase.

Jack Frost zaváhal a ledovými prsty se otřel o sníh. Na okamžik pocítil něco zvláštního, něco téměř jako teplo v hrudi. Mohl se rozplynout v noci a zmizet jako vždycky.

Pak se ale usmál - vzácný, opravdový úsměv. „Myslím, že tentokrát tu zůstanu,“ řekl a jeho hlas byl jemný jako zimní vánek. „Koneckonců, ne každý den se mi podaří být součástí vánočního kouzla.“

A tak na jednu kouzelnou vánoční noc nepřinesl Jack Frost jen zimu - přinesl ducha tohoto období. Zůstal, tančil a smál se s dětmi a poprvé pochopil, co dělá Vánoce skutečně kouzelnými: nebyl to jen mráz, sníh nebo blikající světýlka, bylo to teplo pospolitosti, radost z dávání a láska, která naplnila nejchladnější z nocí.

A když Jack Frost zmizel s prvním světlem vánočního rána, zanechal za sebou poslední dárek - přikrývku třpytivého sněhu, připomínku toho, že i ta nejchladnější srdce může zahřát kouzlo Vánoc.

Konec.

\end{document}