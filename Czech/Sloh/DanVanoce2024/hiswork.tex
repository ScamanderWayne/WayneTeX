\documentclass{article}
\usepackage{wsloh}

\begin{document}
\setws{Jack Frost and the Frozen Christmas}{Daniel Kejval}

On a cold Christmas Eve, far up in the frosty skies, Jack Frost was feeling rather… bored. As the mischievous spirit of winter, he usually had plenty to do: painting the windows with delicate ice patterns, causing snowstorms to sweep through sleepy towns, and leaving trails of frosty footprints behind him. But tonight? Tonight was different. The world below seemed peaceful, and there wasn’t a snowflake in sight. For once, Jack Frost felt like he had no one to play with.

The moon shone brightly above, casting a silver glow over everything, and Jack drifted over the town of Pinebrook, his icy breath turning the trees into shimmering sculptures. As he glided through the quiet streets, he spotted something odd: the entire village was dark. No lights on the houses, no lanterns in the windows, not even a single Christmas tree twinkling with decorations. It was as if Christmas itself had been frozen in time.

Curious, Jack floated down to the village square, where he saw a group of children gathered, looking forlorn. Their heads were hung low, and their faces were pale from the cold. He silently landed next to them, causing the snow beneath his boots to frost over with a gleaming shimmer.

“Hey there!” Jack called out, his voice playful. “What’s with all the long faces? It’s Christmas Eve!”

One of the children, a young girl with bright red mittens, looked up at him, her eyes wide with a mixture of fear and confusion. “It’s not Christmas,” she whispered, as if speaking too loud might shatter something. “We can’t celebrate. The Christmas lights… they’re frozen. The snow won’t fall. It’s just… too cold.”

Jack Frost raised an eyebrow. “Frozen, you say?” He looked around, his frosty breath turning the trees into glittering ice sculptures. “I could have sworn I was the one who makes things cold… but I never knew I could freeze Christmas itself!”

One of the boys, with a messy shock of hair, added, “Every year, the town gets its Christmas magic, and everything lights up! But this year, the magic’s gone. We tried everything, but… nothing works. It’s like the season itself forgot how to be merry.”

Jack’s usual grin spread across his face. A challenge. A mystery. And one he couldn’t resist. “Well, looks like I’ve got some frost to fix! Don’t you worry, kiddos. I’ll figure this out.”

The children stared up at him, their faces still unsure. “But… you’re Jack Frost. Don’t you just bring the cold?” one of them asked, his voice trembling.

Jack laughed, a sound like wind whistling through the trees. “That’s just part of the job. But I’m also the spirit of winter. That means I can bring the chill, sure, but I can also melt it away. And I’m thinking… it’s time to thaw out your Christmas!”

The boy with the messy hair frowned. “How do you plan to do that?”

Jack winked and twirled in midair. “Easy. With a little magic, a little mischief… and a lot of Christmas spirit.”

He stretched out his arms, and the air around him grew colder. The ground sparkled, the frost thickened, and with a flash of silver light, Jack began to hum a merry tune—an ancient Christmas melody that only the frost could carry. He danced across the town square, and with each step, snowflakes began to drift from the heavens. Not just ordinary snowflakes, though. These were sparkling, glowing flakes that swirled with joy and shimmered like stars.

As Jack danced, the Christmas lights along the houses flickered to life, their warm glow spreading across the streets. The frozen trees began to hum with light, their branches covered in a blanket of fresh snow. The Christmas bells rang from the church tower, and soon, the entire village was bathed in the magical glow of the season.

The children gasped in awe, their eyes wide with delight. “You did it!” the girl in the red mittens cheered.

Jack gave a wink and a twirl. “Of course I did. But Christmas magic isn’t just about the lights, or the snow, or the gifts. It’s about the joy we share with others.” He flicked his fingers, sending a flurry of snowflakes toward the children. They giggled as they danced under the snow, the cold now warm with laughter and happiness.

As the village came to life with the sights and sounds of Christmas, Jack Frost hovered above, watching the magic unfold. He could feel the warmth of the holiday in the air, a warmth he didn’t often get to experience in his icy realm. The laughter of the children, the music, the celebration it was like the frost in his heart was slowly melting away.

The town of Pinebrook was alive again, brighter than ever, and it was all thanks to Jack Frost, the spirit who not only brought the chill but knew how to spark the warmth of Christmas when it was needed most.

As the children gathered around the village tree, their faces glowing in the Christmas light, they turned to Jack with smiles.

“Will you stay and celebrate with us?” the girl asked, her voice hopeful.

Jack Frost hesitated, his icy fingers brushing against the snow. For a moment, he felt something strange, something almost like warmth in his chest. He could’ve just melted away into the night, disappearing as he always did.

But then he smiled—a rare, true smile. “I think I’ll stick around this time,” he said, his voice as soft as a winter breeze. “After all, it’s not every day I get to be part of the Christmas magic.”

And so, for one magical Christmas night, Jack Frost didn’t just bring the cold—he brought the spirit of the season. He stayed, danced, and laughed with the children, and for the first time, he understood what made Christmas truly magical: it wasn’t just the frost, the snow, or the twinkling lights it was the warmth of togetherness, the joy of giving, and the love that filled the coldest of nights.

And as Jack Frost vanished with the first light of Christmas morning, he left behind one last gift a blanket of sparkling snow, a reminder that even the coldest hearts could be warmed by the magic of Christmas.

The End.

\end{document}