\documentclass[11pt]{article}
\usepackage[czech]{babel}
\usepackage[a4paper]{geometry}
\usepackage{graphicx} % Required for inserting images
\usepackage{fancyhdr}
\usepackage{yfonts}
\usepackage{coffeestains}
\usepackage{lipsum}
\usepackage{xcolor}
\usepackage{csquotes}
\usepackage{hyperref}

\setlength{\parskip}{0pt}
\setlength{\parindent}{1em}
\setlength{\headheight}{11pt}
\setlength{\footskip}{5pt}

\pagestyle{fancy}
\fancyhf{}
\fancyhead[L]{{\textsc{Karel Hlaváček} -- dekadence -- \textit{Pozdě k ránu}}}
\fancyhead[R]{{Libor Halík}}
% \fancyhead[C]{dekadence}
% \fancyfoot[C]{\footnotesize \textit{\thepage . strana}}
\renewcommand{\headrulewidth}{0.1pt}
% \renewcommand{\footrulewidth}{0.1pt}

\begin{document}

\setlength{\parindent}{2em}

\coffeestainA{0.4}{1}{90}{28em}{38em}

\textsc{Karel Hlaváček} byl dekadentním spisovatelem, kreslířem a sokolem. Svá spisovatelská díla publikoval v rámci časopisu \textit{Sokol}, kde tedy i propagoval združení Sokol, které v Praze--Libni pomáhal založit. Dále své spisy publikoval i v časopise \textit{Moderní Revue}, kde publikoval i svá díla výtvarná. Časopis \textit{Moderní Revue} se vyznačoval právě svým dekadentním obsahem. \par
Dekadence je termín znám téměř již od starého Říma, právě proto, že je spojován s úpadkem Říma kvůli úpadku civilizace. Dekadentní styl se mezi umělci vyznačoval pocity zmaru, nudy, depresí, a jejich životní styl často zahrnoval okultismus, nevázaný sexuální život, extrémní náboženské praktiky či užíváním drog. Mezi nejznámnější dekadentní umělce ve světě patřil \textsc{Oscar Wilde}. Citovanou "biblí dekadence" je román \textit{Naruby}, v němž se objevuje archetyp dekadentního antihrdiny, \textit{Naruby} v podstatě dalo dekadenci její tvář. \par
I když se dekadence neomezovala jenom na literaturu, tak je poměrně obtížné najít obraz s dekadentními prvky. A když už se takový objeví, jedná se o grafiku dle předlohy v podobě dekadentní literatury. \par
Básnická sbírka \textit{Pozdě k ránu} obsahuje 23 básní, jež se vyznačují dekadentními motivy a symbolismem. \textsc{Karel Hlaváček} zde využívá atmosféru pozdní noci a blížícího se rána. Doby šera a mlhy, kdy jsou pocity zmaru a nejistoty nejsilnější. Popisuje zde dosti přírodu a zvláštní péči věnuje popisu měsíce. Stejnojmenná báseň \textit{Pozdě k ránu}, jenž celou básnickou sbírku otevírá, se vyznačuje i vysokou mírou hudebních motivů, které umocňují atmosféru blížícího se rána.

\end{document}