\documentclass[11pt]{article}
\usepackage[czech]{babel}
\usepackage[a4paper]{geometry}
\usepackage{csquotes}
\usepackage{fancyhdr}

\setlength{\parskip}{0pt}
\setlength{\parindent}{1em}
\setlength{\headheight}{10pt}
\setlength{\footskip}{12pt}

\title{\textbf{Vánoce na Sibiři}}
\author{\textsc{Libor Halík}}
\date{\textit{Vytvořeno pro školní vánoční soutěž 2023}}

\begin{document}

\begin{titlepage}
\maketitle
\thispagestyle{empty}
\end{titlepage}
\newpage

\pagestyle{fancy}
\fancyhf{}
\fancyhead[L]{\footnotesize \textbf{Vánoce na Sibiři}}
\fancyhead[R]{\footnotesize \textsc{Libor Halík}}
\fancyfoot[C]{\footnotesize \textit{\thepage . strana}}
\renewcommand{\headrulewidth}{0.1pt}
% \renewcommand{\footrulewidth}{0.1pt}

Byla zima, nekončící zima, stejně tak jako každý rok během tohoto období. Ono i~na~tomto místě. Raskolnikov vždy uvažoval o chladné náruči, kterou mu Sibiř nabídla. Konečky prstů již zcela modré, až na červené šrámy, to se odřel při práci. Viděl, že~dozorci rozdávají rukavice. Už už o jeden pár požádal, pak se ale znovu ozvala ta jeho hrdost. Ne, rukavice nepotřebuje. Z jeho bloumání ho však vytrhl hlas dozorce.

\enquote{Rodion Romanovič Raskolnikov?}

Bleskově mu proběhlo hlavou, co mu tak mohli chtít, ale odpověděl rychle, už se naučil, že~dozorci s nimi moc slitování nemají. Podíval se směrem ke dozorci a zvedl ruku.

\enquote{Přišlo nařízení shora,} dozorce se křenil od ucha k uchu, když mu to říkal. Raskolnikovi se~na~tom něco nezdálo, \enquote{na týden prý máte volno, ale nesmíte opustit Sibiř.}

Dříve, než se vůbec stihl zamyslet, mu vylétla z úst otázka, na kterou by býval radši neznal odpověď.

\enquote{Co že tak náhle?}

Dozorci už byly vidět všechny zuby, ne ve zrovna dobrém stavu.

\enquote{Vaše matička zemřela.}

\hspace{1em}

Příštích pár hodin měl Raskolnikov jako v mlze. Věděl, že ho předali Soně. Ta~ho~odvedla do~svého příbytku, který si pronajímala jen kousek od věznice. Zaznamenal, jak tu Soňu už~všichni zdravili, očividně byla v okolí oblíbená, ale on si nebyl schopen zapamatovat ani cestu, kterou ho vlekla, natožpak lidi, kteří je míjeli. Soňa ho~uložila u sebe do postele a on během pár minut usnul.

Moc si ale neodpočinul, myšlenky na matičku ho pronásledovaly i ve spánku. Hlava mu neustále přehrávala ty momenty, kdy byl na svoji matičku nespravedlivě zlý. Z jeho nočních můr ho ale vytrhl milý obličej Sofji, když se nad ním skláněla.

\enquote{Prospal jsi celé odpoledne, Roďko.} Roďko, tak mu říkaly jen matka se sestrou. Ale~ještě předtím, než zase začal přemýšlet o matičce, tak ho Soňa zase uvedla do reality. \enquote{Vím, že~na~to asi~nemáš moc pomyšlení. Ale blíží se Vánoce a přijede i tvoje sestra s~Razumichinem.}

Ve vězení nikdy nepřemýšlel, jak rychle se Vánoce blíží. Zjištění, že dostal propustku na přesně toto období roku ho překvapilo snad stejně, jako nedávná zpráva. Ale stejně neměl na slavení náladu, vlastně na nic.

\enquote{Říkáš, že přijedou Duňa a Razumichin?} Jeho sestra mu sházela, možná ještě více, než si byl ochoten přiznat.

\enquote{Slíbili to v posledním dopisu, který jsem od nich dostala.} A hned jej už od někud vytahovala. \enquote{Podívej se sám, tady mi to sama psala.} Soňa ho zanechala vlastním úvahám a~šla se~věnovat své práci.

\hspace{1em}

Během dalších dnů byl ve špatném rozpoložení. Pomáhal sice Soně s veškerou prací, ale odmítal se s ní vybavovat o tom, o čem přemýšlí. Po pár takto strávených dnech dorazila Duňa. Se Soňou se přivítala, od Raskolnika odtáhla a~Razumichina omluvila, že přijede až další den dostavníkem, potřeboval ještě něco zařídit u sebe v~práci.

Večer měl prazvláštní atmosféru. Raskolnikov nevěděl co říct. Duňu neviděl už od soudního procesu, vždy si byli blízcí, ale teď mu došly slova. Očima zkoumal každý kousek jídla, jenž nabodl na vidličku, jen aby nemusel čelit pohledu vlastní sestry. Na druhé straně stolu, kde se mohlo zdát, že Duňa prochází stejným myšlenkovým pochodem, protože zarytě mlčela jako její bratr, se ale odehrával pravý opak. Duňa~mu~měla co říct, chtěla mu toho říct až moc, ale nevěděla jak.

V čele stolu, tedy mezi dvěma sourozenci, seděla Soňa. Ta na tento druh situace nebyla zvyklá. Její sourozenci byli moc mladý a neměli spolu tento druh vztahu. Takže~jí~nezbylo nic jiného, než jen za zvuků praskání v kamnech jíst a čekat, kdy jeden ze~sourozenců prolomí toto ticho.

\hspace{1em}

Po večeři se Raskolnikov usadil blízko ke kamnům. Soňa s Duňou uklízely po večeři a~on~tak měl čas si ještě jednou přečíst dopis, který sestra napsala Soně.

\enquote{Ty se nikdy nezměníš Roďko, že?} Ozvalo se mu za zády tak znenadání, že sebou škubnul a~málem se tak připekl ke kamnům. Otočil se i se židlí, aby se díval na Duňu, která ho tak vylekala a~dopis si položil do klína.

Napadala ho spousta možností, o čem by jeho sestra mohla hovořit, ale ani~jednu se~neodvažoval vyslovit nahlas. Jeho sestra tedy po chvíli ticha pokračovala. \enquote{Víš na~co~matka zemřela? Do dopisu jsem to nenapsala, ale zničilo jí, co jsi udělal té bábě.}

To Raskolnika vytočilo, stejně jako jakákoliv zmínka o jeho činu. \enquote{Slíbili jste s Razumichinem, že nic neřeknete!} Nevěděl, zda vyskočit, či zůstat sedět, takže z jeho další akce vyšel spíše jakýsi tik.

\enquote{To jsme také dodrželi. Sama se nějak dopídila. Nikdy o tom nemluvila a to jí zničilo. Ten den jsi nevzal jen dva životy, ty ty životy bereš pořád.} To Raskolnika úplně ohromilo. Dunin výraz teď vypadal jen jako hromádka neštěstí.

\enquote{Vlastně ani nevím, co jsem si od této návštěvy slibovala. Zítra se sbalím a odjedu s~Razumichinem hned dalším dostavníkem.}

V tichu, které nastalo po posledním řečeném, ani jeden nevěděl co říct. Duňa pak odešla, už~to~ticho nemohla snést a tím zanechala Raskolnika zpět jeho myšlenkám.

\hspace{1em}

V noci se mu zase zdálo o matce. Byl v hostinci, kde Razumichin tenkrát jeho matku se~sestrou ubytoval. Byli tam sami, on a matička. Našel jí tam, jak sedí u nevelkého stolu. Zmocnil se~ho~strach, ale šel blíž.

Zmohl se jen na jediné. \enquote{Jak se cítíte, maminko?} Hned ho napadlo, jak nicotně to~znělo, ale~ta~otázka už~visela ve~vzduchu.

Obraz jeho matky k němu pozvedl své oči. Najednou si všiml, jak mrtvolně bledá její tvář je. Jak její propadlé oči ho sledují ze dna její lebky.

\enquote{Proč jsi to udělal?} Ozval se suchý hlas, který nemohl patřit jeho matce, ale stejně jí vyházel z úst.

A Rodion to už nevydržel. Tady stál, před podobiznou své milované matky. Slyšel z jejích úst stejnou výčitku, jako tého večera od jeho sestry, kterou viděl v očích Soni a kterou cítil někde vzadu, ve svých myšlenkách, kdykoliv byl sám se sebou. Už~to~nevydržel a~padl. Padl na kolena a~plakal. Plakal a~omlouval~se. Snažil se ospravedlnit se, jen aby dodal, že si nic lepšího nezaslouží. Po všech výčitkách, které slyšel, byly výčitky jeho svědomí ty nejhorší. Byly těžší, než jakýkoliv trest mohl dostat a teď si~to~konečně přiznal.

Mohlo uběhnout několik hodin, stejně tak i jen pár minut, ale když konečně vzhlédl ke~své matce, viděl jí tam na židli. Ne její karikaturu, ale svoji matku, tak jak si ji pamatoval, když se~na~něj naposledy usmála.

\enquote{Ale já už ti odpustila. Teď je to jen na tobě.} A s těmito slovy se sen rozplynul.

\hspace{1em}

Probudil se, sice ne zrovna čilí, ale zvláštně odlehčen. Jak kdyby z něj spadla tíha, kterou si~sebou přivezl už tenkrát, když ho odvezli na Sibiř.

Po snídani šli vyhlížet Razumichina. Soňa s nimi nešla, říkala, že musí připravit jídlo. Po chvíli ticha se Duňa rozmohla. Začali mluvit, Duňa se mu omlouvala, ale že to už~ze~sebe potřebovala vypustit. Raskolnikov její omluvy nepřijímal, ne proto, že~by~byl uražen, ale protože cítil, že jeho sestra měla pravdu. Takhle se~přetahovali o vinu, dokud je nevyrušil známý hlas.

\enquote{O čem se to tu hádáte tak zapáleně? Smím se přidat?}

Byl to Razumichin, stál opodál a očividně si užíval hádky, kterou mohl sledovat, jako kdyby sledoval divadlo. Duňa mu skočila kolem krku, jen aby mu začala hubovat.

\enquote{To se sluší, Dmitriji, mě takto lekat z pozadí?}

Ale Razumichin se usmíval dál. Tu stočil pohled na Raskolnika, který nervózně, ale~zároveň šťastný, z toho, že vidí svého starého přítele, stál stranou a čekal, až~se~man-želský pár přivítá. Razumichinův úsměv se snad ještě rozšířil.

\enquote{Rodione! Příteli!} A tu už ho chytil do náručí a Raskolnikov nemohl jinak, než se~usmívat a~gesto mu oplatit.

\hspace{1em}

A tak se stalo, že se všichni sešli k večeři. Stůl prostřen a obložen jídlem. Byt vyzdoben, provoněn vůní čerstvě upečeného jídla a svíček. Navzdory dění dní minulých se tady sešli jako rodina. A smáli se a jedli.

Tu Razumichin povídá, co za knížku se mu dostala pod ruku, tu zas co se Duněčce stalo, u~nich doma. Následně Soňa, jak se tu sžila s místními. I Rodion se přidal, vyprávěl, jak mu už i ta práce tolik nesmrdí.

Nevypadalo to na to, ale lépe strávený večer si žádný z nich nemohl přát. Ani pití nesházelo. Když bylo po večeři, tak se splečně uchýlili ke krbu a tam jejich konverzace pokračovala.

A i když ten večer musel skončit, Razumichin s Duňou odjet, Raskolnikov vrátit ke~svému trestu a Soňa na něj nadále čekat a mezitím vydělávat jak se dalo, tak~i~přesto všechno byl~ten~večer dokonalý. Mnohdy na něj vzpomínali, když jim bylo hůř, ale~to už~je~jiný příběh.

\end{document}
