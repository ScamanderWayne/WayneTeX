\documentclass[11pt]{article}
\usepackage[czech]{babel}
\usepackage[a4paper]{geometry}
\usepackage{graphicx} % Required for inserting images
\usepackage{fancyhdr}
\usepackage{yfonts}
\usepackage{coffeestains}
\usepackage{lipsum}
\usepackage{xcolor}
\usepackage{csquotes}
\usepackage{lettrine}

\setlength{\parskip}{0pt}
\setlength{\parindent}{1em}
\setlength{\headheight}{14pt}
\setlength{\footskip}{5pt}

\pagestyle{fancy}
\fancyhf{}
\fancyhead[L]{{Přítel po ruce a přesto na míle daleko}}
\fancyhead[R]{{Libor Halík}}
% \fancyfoot[C]{\footnotesize \textit{\thepage . strana}}
\renewcommand{\headrulewidth}{0.1pt}
% \renewcommand{\footrulewidth}{0.1pt}

\begin{document}

\setlength{\parindent}{1em}

\coffeestainA{0.4}{1}{90}{28em}{38em}

\subsubsection*{Milí žáci, dovolte mi se dnes zaměřit na nám všem známou elektronickou komunikaci.}

\hspace{1em}Telefony, snad všichni je máme v kapsách od kalhot.
Představují napojení na \textit{Síť}, imaginární místo, kde vás červí díra spojuje s každým jednotlivým člověkem na světě, co má v kapse také tuto krabičku.
Nevidíte druhého člověka, necítíte ho, ale je tam.
Někde, ve svém světě, ve světě, který se s vaším nikdy nemusí protnout, ale je tam.
A~na tomto je něco zvláštně fascinujícího.
Člověk, co tu je a přesto není.
Člověk kterého vnímáte jenom vy, protože nikdo jiný z vašeho okolí o něm nemusí vědět.
\par Snad právě díky tomu vzrostla natolik popularita takzvaných \enquote{vztahů na dálku,} kdy je vám přítelem člověk, kterého jste nikdy nepotkali.
Je to jenom váš člověk.
Je tu jenom pro vás, ne pro vaši rodinu, ne pro vaše přátele, jenom pro vás.
Samozřejmě přítel na~telefonu nemusí být jenom neznámý cizinec.
Někdy se takový cizinec dostane do vašeho okolí a už nejde jen o elektronickou komunikaci.
Jindy zase člověk vám blízký se musí odstěhovat a naopak povýší na přítele na telefonu.
Určitě jste již zpozorovali změnu dynamiky vztahu, když se jedna z těchto dvou možností stane realitou.
\par Jenže z toho plyne i nebezpečí.
Vaše konverzace je sice utajená před okolím.
Ale právě proto si nikdo nevšimne, když se to zvrtne.
A zvrtne se to často.
Až moc často.
Když se přátelské svěřování změní v šikanu a vyhrožování.
Okolí ke tomu je slepé již teď.
Natož když se vše odehrává jen v malém obdelníku v kapse pod opaskem.
A i přesto to stále je nebezpečné.
Možná ještě více než klasická fyzická konverzace.
\par Dovolte mi ještě slovo závěrem.
Soukromá konverzace je lákavá.
Snad i svým způsobem romantická.
Jak z nějakého starého románu.
Ale neuzavírejme se před světem.
Nevní-mejme okolí jen skrze pár pixelovou obrazovku v dlani.
Elektronická komunikace určitě má~spoustu výhod.
Jednou z nich je právě propojení s celým světem.
Ale neodstřihávejme se kvůli tomu od světa kolem nás.
\vspace{1em}
\par\noindent Děkuji za pozornost.

\end{document}