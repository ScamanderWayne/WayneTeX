\documentclass[11pt]{article}
\usepackage[czech]{babel}
\usepackage[a4paper]{geometry}
\usepackage{graphicx} % Required for inserting images
\usepackage{fancyhdr}
\usepackage{yfonts}
\usepackage{coffeestains}
\usepackage{lipsum}
\usepackage{xcolor}
\usepackage{csquotes}
\usepackage{hyperref}

\setlength{\parskip}{0pt}
\setlength{\parindent}{1em}
\setlength{\headheight}{11pt}
\setlength{\footskip}{5pt}

\pagestyle{fancy}
\fancyhf{}
\fancyhead[L]{{\textsc{Jáchym Topol}}}
\fancyhead[R]{{Libor Halík}}
% \fancyhead[C]{dekadence}
% \fancyfoot[C]{\footnotesize \textit{\thepage . strana}}
\renewcommand{\headrulewidth}{0.1pt}
% \renewcommand{\footrulewidth}{0.1pt}

\begin{document}

\setlength{\parindent}{2em}

\coffeestainA{0.4}{1}{90}{28em}{38em}

Jáchym Topol je významným českým prozaikem, žurnalistou, básníkem a hudebníkem.
Maturitu složil v devatenácti letech, začátkem 80. let.
V období 80. a 90. let byl aktivní jako básník, kdy v tomto období vydal samizdatem i svojí básnickou sbírku \textit{Miluji tě k zbláznění}, za kterou dostal ocenění Toma Stopparda, Nadace Charty 77.
V těchto letech účinkoval i jako skladatel a zpěvák pro skupinu, jíž frontman byl jeho bratr.
Začátkem devatesátých let mohl opustit svojí předchozí práci, topiče, a stal se spisovatelem z povolaní, v tomto období se stal i šéfredaktorem čtvrťletníku \textit{Revolver Revue}. \par
Byl také součástí českého undergroundu.
Jako český underground je chápána skupina autorů, co se vymezovala proti cenzuře během režimu, co v ČSSR tou dobou vládnul.
Zastánci tohoto hnutí se vyznačovali svojí vůlí psát svobodně a bez dohlížení režimu.
Pronásledování undergroundové hudební skupiny PPU také vedlo k založení Charty 77, která dále kritizovala praktiky tehdejšího režimu a rozšířila tím povědomí i mezi další sociální třídy obyvatel ČSSR.
Undergroundové období Jáchyma Topola souvisí i s jeho samizdatovým vydáváním, čímž je chápáno vydání díla na vlastní náklady a tím obejití nakladatelství a cenzury s tím spojené.
V tomto období se i účastnil českého monarchistického hnutí, co žádalo návrat krále na český trůn a navrhovalo, aby se politické strany začaly zabývat jinými odvětvími. \par
Topolovo dílo \textit{Sestra}, vydána v polovině 90. let, která je jeho nejvýznamnější a také první napsanou prózou, popisuje život v 80. a 90. letech v ČSSR. \textit{Sestra} také pojednává o kapitalismu, rasismu, či narkomanii v českých zemích. \par
Mezi jeho slavné příbuzné patří otec spisovatel Josef Topol, matka, dcera spisovatele Karla Schulze, kteří oba patřili do kruhu přátel Václava Havla.
Dále i jeho již zesnulí mladší bratr Filip Topol, který se proslavil jako frontman undergroundové skupiny Psí vojáci.

\end{document}