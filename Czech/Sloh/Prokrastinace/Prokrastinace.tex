\documentclass[12pt]{article}
\usepackage[czech]{babel}
\usepackage[a4paper]{geometry}
\usepackage{fancyhdr}
\usepackage{csquotes}

\setlength{\parskip}{0pt}
\setlength{\parindent}{2em}

\pagestyle{fancy}
\fancyhf{}
\fancyhead[R]{Libor Halík}
\fancyhead[L]{Prokrastinace -- nemoc či výmysl?}

\begin{document}
Prokrastinace, v dnešní době tak diskutované téma, je snad všudypřítomná.
Pro~ty, jež o ní ale stále neslyšeli, stačí snad jenom krátká definice.
\enquote{Prokrastinace je~výrazná a chronická tendence odkládat plnění (většinou administrativní či~psychicky náročných) povinností a úkolů (zejména těch nepříjemných) na pozdější dobu.}
Lidově řečeno, lenost.

Nejde však o lenost v pravém slova smyslu, což je ta tolik potřebná osvěta, protože to stále spousta lidí ani nepřipouští.
Člověk, trpící prokrastinací, ani~není schopen se byť jen soustředit na těžší úkol -- natolik je tento ochranný mechanismus zakódovaný přímo v člověku.
Při, laicky řečeno, prokrastinování, je~člověku natolik bídně z těžkého úkolu před ním, že se ho celý organismus snaží ochránit.
Furt přemýšlí nad jednodušší prací, kterou by mohl zvládnout.
Pořád myšlenkami odbíhá jinam.
Nejedná se o lenivost, kdy by taková osoba jenom nechtěla dělat svojí práci.
Takový jedinec si je dobře vědom své povinnosti pracovat, a~tak~se~to snaží vyrovnat aspoň tou méně náročnou prací, protože fyzicky není schopen splnit práci náročnější.

Mohli bychom házet vinu na takového prokrastinátora, který se přeci rozhodne nedělat práci, co má.
Ale uvědomme si, že prokrastinace je nemoc.
Sic civilizační a~lidé jí netrpící jí mají tendenci zlehčovat, ale stále nemoc.
Však~můžou za~to~snad ti nemocní sami, že nezvládají danou práci, či my, že jim tuto práci nutíme?
V~dnešní době máme až moc standardů, které se snažíme vnutit celé společnosti, že~zapomínáme na individualitu a potřeby jedince.
Dnešní doba je~nejenom zrychlená, ale i náročná.
Vše se musí dělat rychle, precizně, ale zároveň levně.
Lidem nestačí pracovat, oni musí týdny vymýšlet, jak danou práci zlehčit, jen~aby~další týdny vymýšleli, jak jí zkvalitnit.
A nejlépe tak, aby se týdenní práce stihla za~jednu osmi hodinovou směnu.
Můžeme prokrastinaci chápat jako fenomén mezi těmi lenivějšími z nás, ale zkusme se zamyslet -- nemůže snad za~selhání jedince společnost?
\end{document}