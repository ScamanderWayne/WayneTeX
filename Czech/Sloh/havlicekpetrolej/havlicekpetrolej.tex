\documentclass{article}
\usepackage[czech]{babel}
\usepackage{geometry}
\geometry{a4paper, margin=2cm}
\usepackage{enumitem}
\usepackage{parskip}
\usepackage{nopageno}
\usepackage{xcolor}
\usepackage[enable]{darkmode}
\usepackage{mdframed}
\usepackage{lmodern}
\usepackage[fixed]{fontawesome5}

\setlist[itemize]{label=\textcolor{cyan}{$\hookrightarrow$}} %\textbullet \faGreaterThan

\definecolor{DP}{HTML}{293133}

\NewDocumentCommand{\wextbf}{m}{%
	{\vspace{-6pt}\hspace{-7pt}\noindent\colorbox{cyan}{\color{DP}\textbf{#1}}}
}

\NewDocumentCommand{\textbw}{m}{%
	{\color{cyan}{#1}}
}

\NewDocumentCommand{\wemph}{m}{%
	{\color{cyan}\emph{#1}}
}

\newcommand{\mysection}[2]{
	\setlength\fboxsep{4pt} % spacing around box contents
	\section*{\colorbox{DP}{\makebox[\dimexpr\textwidth-2\fboxsep\relax]{\color{cyan}#1\hfill#2}}}\vspace{-4pt}
}

\begin{document}
\fontsize{10pt}{10pt}\selectfont
\setlength\parindent{0em}
\mysection{Jaroslav Havlíček}{Život a literární odkaz}
	
% Biografie a literární kontext
\begin{mdframed}[backgroundcolor=DP,
	linecolor=cyan,
	linewidth=2pt,
	innertopmargin=5pt,
	innerbottommargin=5pt,
	innerleftmargin=5pt,
	innerrightmargin=5pt]\color{white}
\wextbf{Biografie a literární kontext}\par
Jaroslav Havlíček (1896–1943) byl významným českým spisovatelem, jehož tvorba patří k vrcholům meziválečné české literatury.
Narodil se v Jilemnici, městě v Krkonoších, které inspirovalo prostředí jeho děl, včetně románu \wemph{Petrolejové lampy}.
Jeho život byl poznamenán předčasnou smrtí na zánět mozkových blan a omezenými možnostmi publikování během druhé světové války.
Havlíček studoval obchodní vědy na Českém vysokém učení technickém v Praze, avšak studia nedokončil a věnoval se především literatuře, kde se prosadil jako mistr psychologického románu.
Jeho tvorba, ovlivněná autory jako \wemph{Fjodor Dostojevskij}, \wemph{Franz Kafka} či \wemph{Marcel Proust}, kombinuje introspekci, realistické zobrazení venkovského prostředí a sociální kritiku.
Mezi jeho klíčová díla patří \wemph{Neviditelný} (1937), \wemph{Ta třetí} (1939), \wemph{Helimadoe} (1940) a zejména \wemph{Petrolejové lampy} (1944), které tvoří vrchol jeho tvorby.
\textit{Funfact}: Havlíčkova díla, včetně \wemph{Petrolejových lamp}, byla přeložena do němčiny.
Havlíčkův syn Zbyněk byl surrealistický básník. 
\end{mdframed}\vspace{-2pt}

% Klíčové rysy Havlíčkovy tvorby
\begin{mdframed}[backgroundcolor=DP,
	linecolor=cyan,
	linewidth=2pt,
	innertopmargin=5pt,
	innerbottommargin=5pt,
	innerleftmargin=5pt,
	innerrightmargin=5pt]\color{white}
\wextbf{Klíčové rysy Havlíčkovy tvorby}\par
Havlíčkova díla se vyznačují následujícími rysy, které odrážejí jeho zájem o lidskou psychiku a společenské poměry:
\begin{itemize}[leftmargin=*]\setlength\itemsep{-4pt}
	\item \textbw{Psychologický realismus} -- Hluboká analýza vnitřních konfliktů, motivací a emocí postav, odhalující složitost lidské psychiky.
	\item \textbw{Sociální kritika} -- Zobrazení společenských tlaků a patriarchálních struktur, zejména v kontextu venkovského života.
	\item \textbw{Fatalismus} -- Tragické motivy a osudovost, kdy postavy čelí nevyhnutelným životním konfliktům.
	\item \textbw{Lyrický styl} -- Poetické vyjadřování propojující realistické popisy se symbolickými rovinami.
	\item \textbw{Rodinné vztahy} -- Zkoumání napětí a obětí v rodinách, často spojených s osobními dilematy.
	\item \textbw{Morální otázky} -- Konflikt mezi povinností a individuální svobodou či hledání smyslu života.
\end{itemize}
\end{mdframed}\vspace{-2pt}
\mysection{Petrolejové lampy}{Literární analýza}

% Vznik a kontext díla
\begin{mdframed}[backgroundcolor=DP,
	linecolor=cyan,
	linewidth=2pt,
	innertopmargin=5pt,
	innerbottommargin=5pt,
	innerleftmargin=5pt,
	innerrightmargin=5pt]\color{white}
	\wextbf{Vznik a kontext díla}\par
Román \wemph{Petrolejové lampy}, původně vydaný jako \wemph{Vyprahlé touhy} v roce 1935 a posmrtně přepracovaný v roce 1944, je Havlíčkovým nejvýznamnějším dílem. Měl tvořit první část plánované trilogie o Štěpce Kiliánové a prostředí Jilemnice, ale Havlíček dokončil pouze tento román a fragment \wemph{Vlčí kůže}. Příběh sleduje Štěpku Kiliánovou, dceru zednického mistra, která se z finančních důvodů provdá za svého bratrance, důstojníka Pavla Malinu, trpícího syfilidou a progresivní paralýzou. Tento sňatek ji uvězní v tíživém prostředí, kde bojuje s nemocí manžela, osamělostí a společenskými očekáváními. Román je hlubokou studií ženské psychiky a kritikou venkovských struktur na přelomu 19. a 20. století. \textit{Funfact}: Filmová adaptace od Juraje Herze (1971) s Ivou Janžurovou byla vybrána na Filmový festival v Cannes 1972.
\end{mdframed}\vspace{-2pt}
	
% Hlavní motivy a témata
\begin{mdframed}[backgroundcolor=DP,
	linecolor=cyan,
	linewidth=2pt,
	innertopmargin=5pt,
	innerbottommargin=5pt,
	innerleftmargin=5pt,
	innerrightmargin=5pt]\color{white}
\wextbf{Hlavní motivy a témata}\par
Román \wemph{Petrolejové lampy} je bohatý na motivy, které odrážejí Havlíčkův zájem o lidskou psychiku:
\begin{itemize}[leftmargin=*]\setlength\itemsep{-4pt}
	\item \textbw{Oběť a povinnost} -- Štěpka se obětuje pro rodinu, což ji vede k vnitřnímu rozporu mezi povinností a touhou po štěstí.
	\item \textbw{Společenský tlak} -- Kritika patriarchálních norem venkovské společnosti omezujících svobodu žen.
	\item \textbw{Psychologická hloubka} -- Detailní zobrazení Štěpčiných vnitřních konfliktů a smiřování s osudem.
	\item \textbw{Symbolika petrolejových lamp} -- Lampy symbolizují kontrast mezi světlem naděje a temnotou nemoci a společenských omezení.
	\item \textbw{Tragédie a fatalismus} -- Osudový průběh příběhu podtrhuje nemožnost uniknout osobním a společenským překážkám.
	\item \textbw{Láska a osamělost} -- Zkoumání nenaplněné touhy po lásce, potlačované rodinnými povinnostmi.
\end{itemize}
\end{mdframed}\vspace{-2pt}
	
% Literární význam
\begin{mdframed}[backgroundcolor=DP,
	linecolor=cyan,
	linewidth=2pt,
	innertopmargin=5pt,
	innerbottommargin=5pt,
	innerleftmargin=5pt,
	innerrightmargin=5pt]\color{white}
\wextbf{Literární význam}\par
\wemph{Petrolejové lampy} přesahují rámec regionálního románu díky univerzální tematice a hluboké analýze psychiky. Havlíček propojuje realistické zobrazení Jilemnice s lyrickými a symbolickými prvky, čímž vytváří dílo o oběti, svobodě a hledání štěstí. Román zůstává klíčovým příspěvkem k české literatuře, podpořený i filmovou adaptací a zájmem čtenářů. \textit{Funfact}: Havlíčkův styl byl přirovnáván ke Kafkovi díky jeho schopnosti zachytit absurditu a tragiku lidského osudu.
\end{mdframed}\vspace{-2pt}
\end{document}