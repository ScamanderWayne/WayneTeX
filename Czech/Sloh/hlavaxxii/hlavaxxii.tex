\documentclass{article}
\usepackage[czech]{babel}
\usepackage{geometry}
\geometry{a4paper, margin=2cm}
\usepackage{enumitem}
\usepackage{parskip}
\usepackage{nopageno}
\usepackage{xcolor}
\usepackage{darkmode}
\usepackage{mdframed}
\usepackage{lmodern}
\usepackage[fixed]{fontawesome5}
\usepackage{lipsum}

\usepackage{titlesec}
\titlespacing*{\section}{0pt}{0pt}{0pt}
\titlespacing*{\subsection}{0pt}{0pt}{0pt}

\IfDarkModeTF{%
\definecolor{DP}{HTML}{293133}%
%
\definecolor{antiDP}{HTML}{FFFFFF}%
%
\definecolor{WPP}{HTML}{00FFFF}%cyan
\definecolor{WPP}{HTML}{0066CC}%blue
\definecolor{WPP}{HTML}{C9A12A}%gold
\definecolor{WPP}{HTML}{EC2B2B}%red
}{%
\definecolor{DP}{HTML}{FFFFFF}%
%
\definecolor{antiDP}{HTML}{000000}%
%
\definecolor{WPP}{HTML}{702632}%lightmode wine
}

\setlist[itemize]{label=\textcolor{WPP}{$\hookrightarrow$}}%\textbullet \faGreaterThan

\NewDocumentCommand{\wextbf}{m}{%
	{\subsection*{\vspace{-6pt}\hspace{-7pt}\noindent\colorbox{WPP}{\color{DP}{#1}}}}%
}

\NewDocumentCommand{\textbw}{m}{%
	{\color{WPP}{#1}}%
}

\NewDocumentCommand{\wemph}{m}{%
	{\color{WPP}\emph{#1}}%
}

\NewDocumentCommand{\mysection}{m m}{%
	\setlength\fboxsep{4pt}% spacing around box contents
	\section*{\colorbox{DP}{\makebox[\dimexpr\textwidth-2\fboxsep\relax]{\color{WPP}#1\hfill#2}}}\vspace{-3pt}%
}

\NewDocumentCommand{\waybox}{m m}{%
\begin{mdframed}[backgroundcolor=DP,%
	linecolor=WPP,%
	linewidth=2pt,%
	innertopmargin=5pt,%
	innerbottommargin=5pt,%
	innerleftmargin=5pt,%
	innerrightmargin=5pt]\color{antiDP}%
	\wextbf{#1}\par%
	#2%
\end{mdframed}%
}

\NewDocumentCommand{\wayitem}{m m}{%
\begin{itemize}[leftmargin=*]\setlength\itemsep{#2}%
	#1%
\end{itemize}%
}

\NewDocumentCommand{\wtem}{m}{%
\item \textbw{#1} --%
}

\begin{document}\fontsize{10pt}{10pt}\selectfont\setlength\parindent{0em}
	\mysection{Joseph Heller}{Libor Halík}
	\waybox{O životě autora}{
	\wemph{Joseph Heller} byl americkým spisovatelem žijícím a působícím ve 20. století.
	Narodil se do rodiny židovských přistěhovalců \textbw{v Brooklynu} v~New~Yorku.
	Pět let po konci \textbw{Velké války}.
	Během \textbw{Druhé~světové~války} lítal s~bombardérem v rámci amerického letectva.
	Po válce studoval na Kolumbijské univerzitě a později učil tvůrčí psaní.
	\wemph{Hlava~XXII}, vydaná v~roce 1961, mu přinesla světovou slávu, i když zpočátku nebyla komerčně úspěšná.
	Heller napsal i~další romány, například \wemph{Něco~se~stalo} nebo \wemph{Dobrý~jako~zlato}, ale žádný nedosáhl takového ohlasu jako jeho právě \wemph{Hlava~XXII}.
	Jeho styl vyniká černým humorem, satirou a kritikou byrokracie.
	Kromě psaní se věnoval i~scenáristice a~dramatické tvorbě.
	Hellerova díla často zkoumají \textbw{absurditu lidského chování} a \textbw{společenských systémů}.
	Zemřel na infarkt v roce 1999.
	}
	\mysection{Hlava XXII}{}
	\waybox{O knize}{
	Hlava XXII (Catch-22) je satirický román z druhé světové války, vydaný v roce 1961.
	Sleduje osudy kapitána Johna Yossariana, bombardéra na fiktivním ostrově Pianosa.
	Kniha kritizuje absurditu války, vojenské byrokracie a lidské chamtivosti.
	Název „Hlava XXII“ odkazuje na paradoxní pravidlo, které znemožňuje letcům vyhnout se misím.
	Román je proslulý nelineární strukturou a černým humorem.
	Yossarianův boj o~přežití odhaluje nesmyslnost vojenských pravidel a morální dilemata.
	Kniha kombinuje komedii s tragédií, což z ní činí jedinečný literární zážitek.
	Stala se kultovní klasikou a ovlivnila moderní literaturu i popkulturu.
	Byla adaptována do filmu (1970) a později do seriálu (2019).
	Hlava XXII je považována za jedno z~nejvýznamnějších děl 20. století.
	}
\end{document}