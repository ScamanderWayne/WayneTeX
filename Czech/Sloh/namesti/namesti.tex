\documentclass{article}
\usepackage[czech]{babel}
\usepackage[enable]{darkmode}
\usepackage[left=2.5cm,right=2.5cm,top=2.5cm,bottom=2.5cm]{geometry}
\usepackage{nopageno}
\usepackage{yfonts}
\usepackage{hyperref}
\usepackage{fancyhdr}
\usepackage{xcolor}
\usepackage{tikz}
\usepackage{palatino}

\NewDocumentCommand{\setws}{m m m}{%
	\def\wauthor{#1}%
	\def\wname{#2}%
	\def\wsubject{#3}%
}

\setws{Libor Halík}{Náměstí}{Slohová práce na maturitní otázku na popis náměstí}

\hypersetup{
	pdftitle={\wname},
	pdfauthor={\wauthor},
	pdfsubject={\wsubject},
	pdfkeywords={},
	pdfcreator={LaTeX with hyperref},
	pdfproducer={pdfLaTeX} % Or whatever engine you are using.
}

\definecolor{DP}{HTML}{293133}

\pagestyle{fancy}
\fancyhf{}
\fancyhead[R]{%
\textsc{\small{\IfDarkModeTF{%
	\color{cyan}\wauthor%
	}{%
	\wauthor%
	}%
}}}%
\fancyhead[L]{%
\textbf{\IfDarkModeTF{%
	\colorbox{cyan}{\color{DP}\wname}%
	}{%
	\wname%
	}%
}}%
\fancyfoot[R]{%
\IfDarkModeTF{%
	{\color{cyan}\small Powered by \LaTeX.}%
	}{%
	{\small Powered by \LaTeX.}%
	}%
}%

\renewcommand{\headrule}{}

\begin{document}
	Krakonošovo náměstí se otevírá jako náruč, do které vtéká život z tepající pěší zóny. Hlavní vstup, nejširší brána do tohoto světa, je lemován starobylými domy, jejichž zdi nesou stopy času.
	Po levé ruce se majestátně tyčí radnice s věží, z jejíž tváře shlíží Trutnovský drak, symbol města, zavěšený jako strážce legend. \par
	Uprostřed náměstí, tam, kde se kdysi možná zrcadlila hladina, nyní z kamenné nádrže vyrůstá Krakonoš, vytesaný do podoby skalního masivu.
	Jeho tvář, plná moudrosti a síly, shlíží na shon a tichý proud života kolem. Krakonoš, strážce hor, stojí pevně, obklopen kamennou nádrží, jako by sám byl součástí krajiny. \par
	Kousek dál, Sloup Nejsvětější Trojice a socha Josefa II. vyprávějí příběhy minulosti, připomínají návštěvy císaře a duchovní hodnoty, které formovaly toto místo.
	V pravém horním rohu náměstí, ukrytý v podloubí, leží Kámen z Amazonky, tichý svědek dalekých cest a exotických příběhů. \par
	Okolo náměstí se vine řada obchodů a kaváren, jejichž výlohy lákají kolemjdoucí a vůně kávy se mísí s~energií města.
	Lidé se procházejí po dlážděných cestách, zastavují se u obchodů a posedávají v kavárnách, kde si vychutnávají chvíle odpočinku. \par
	Když se slunce skloní k západu, náměstí se ponoří do teplého světla speciálních lamp, jejichž kulatá svítidla vrhají na dlažbu měkké stíny. Drak na věži radnice, osvětlený lampami, vypadá jako by každou chvíli vzlétl.
	Atmosféra náměstí se mění, stává se intimnější a tajemnější. \par
	Krakonošovo náměstí, bez ohledu na to, co tu kdysi stálo, je živým srdcem města, místem, kde se setkává minulost s přítomností, kde se mísí legendy s každodenním životem.
\end{document}
