\documentclass{article}
\usepackage[czech]{babel}
\usepackage{geometry}
\geometry{a4paper, margin=2cm}
\usepackage{enumitem}
\usepackage{parskip}
\usepackage{nopageno}
\usepackage{xcolor}
\usepackage{mdframed}
\usepackage{lmodern}
\usepackage[fixed]{fontawesome5}
\usepackage{lipsum}
\usepackage{csquotes}

\usepackage{darkmode}

\usepackage{hologo}

\usepackage{titlesec}
\titlespacing*{\section}{0pt}{0pt}{0pt}
\titlespacing*{\subsection}{0pt}{0pt}{0pt}

\IfDarkModeTF{%
\definecolor{DP}{HTML}{293133}%
%
\definecolor{antiDP}{HTML}{FFFFFF}%
%
\definecolor{WPP}{HTML}{00FFFF}%cyan
\definecolor{WPP}{HTML}{0066CC}%blue
\definecolor{WPP}{HTML}{C9A12A}%gold
\definecolor{WPP}{HTML}{EC2B2B}%red
\definecolor{WPP}{HTML}{39DDD5}%sort of cyan
}{%
\definecolor{DP}{HTML}{FFFFFF}%
%
\definecolor{antiDP}{HTML}{000000}%
%
\definecolor{WPP}{HTML}{702632}%lightmode wine
}

\setlist[itemize]{label=\textcolor{WPP}{$\hookrightarrow$}}%\textbullet \faGreaterThan

\NewDocumentCommand{\wextbf}{m}{%
	{\subsection*{\vspace{-6pt}\hspace{-7pt}\noindent\colorbox{WPP}{\color{DP}{#1}}}}%
}

\NewDocumentCommand{\texthl}{m}{%
	{\color{WPP}{#1}}%
}

\NewDocumentCommand{\textbl}{m}{%
	{\textbf{\color{WPP}{#1}}}%
}

\NewDocumentCommand{\wemph}{m}{%
	{\color{WPP}\emph{#1}}%
}

\NewDocumentCommand{\mysection}{m m}{%
	\setlength\fboxsep{4pt}% spacing around box contents
	\section*{\colorbox{DP}{\makebox[\dimexpr\textwidth-2\fboxsep\relax]{\color{WPP}#1\hfill#2}}}\vspace{-3pt}%
}

\NewDocumentCommand{\waybox}{m m}{%
\begin{mdframed}[backgroundcolor=DP,%
	linecolor=WPP,%
	linewidth=2pt,%
	innertopmargin=5pt,%
	innerbottommargin=5pt,%
	innerleftmargin=5pt,%
	innerrightmargin=5pt]\color{antiDP}%
	\wextbf{#1}\par%
	#2%
\end{mdframed}%
}

\NewDocumentCommand{\wayitem}{m m}{%
\begin{itemize}[leftmargin=*]\setlength\itemsep{#2}%
	#1%
\end{itemize}%
}

\NewDocumentCommand{\wtem}{m}{%
\item \texthl{#1} --%
}

\begin{document}\fontsize{10pt}{10pt}\selectfont\setlength\parindent{0em}
	\mysection{Absurdní drama}{}
	\waybox{Popis žánru:}{
	Směr \wemph{Avantgardního divadla}, vyskytující se hlavně v \wemph{padesátých letech dvacátého století}.
	Realita je \texthl{nesmyslná}.
	Řeč \texthl{ztrácí} svůj \texthl{význam}.
	Děj se \texthl{vytrácí}, postavy se \texthl{deevolvují} ze svých charakteristik, \texthl{deformace} a \texthl{devalvace} jazyka, který tím ztrácí svojí funkci.
	Hry jsou tragikomické, někdy spíš tragické, častý je černý humor a~grotesknost situací.
	}
	\mysection{Čekání na Godota}{Samuel Beckett}
	\waybox{O díle:}{
	\wemph{Absurdní drama} napsané \wemph{Samuelem Beckettem} ve francouzštíně (\wemph{En attendant Godot}) koncem padesátých let.
	Autor jí i  přeložil do angličtiny (\wemph{Waiting for Godot}), kde přidal i podtitul \wemph{\enquote{tragikomedie o dvou jednáních}}.\\
	\vspace{0.3em}\\
	Scénu tvoří venkovská cesta a strom (někteří uvádějí i kámen), u kterého se setkávají dva spřátelení tuláci, \textbl{Estragon} a \textbl{Vladimír}.
	Tito dva spolu vedou \texthl{absurdní rozhovor}, ze kterého je jisté jen jedno, čekají na jakéhosi pána Godota, který jim změní život.
	Tento nesmysl zakončuje příchod \texthl{sadistického} pána \textbl{Pozza}, který si na provazu vede svého \texthl{podřízeného} sluhu--otroka \textbl{Luckyho}.
	První dějství pak ukončuje příchod chlapce, který pánům oznámí, že~se~Godot \texthl{nedostaví}.\\
	Druhé dějství pak téměř \texthl{opakuje} dějství první.
	Hlavní rozdíl činí příchod \texthl{slepého} pána \textbl{Pozza}, který je nyní \texthl{závislým} na svém \texthl{hluchém} sluhovi \textbl{Luckym}.
	I tentokrát přichází chlapec, oznamující pánům, že Godot ani dnes nedorazí.
	Dva tuláci se chtějí rozejít, že přijdou zase zítra, ale \texthl{nedokážou se pohnout}.\\
	\vspace{0.3em}\\
	Hra má samozřejmě několikero \texthl{významových rovin}.
	Ta nejzákladnější ale vychází z absurdity lidského očekávání, kdy celé životy strávíme čekáním na \wemph{\enquote{něco}} a toto čekání je třeba si zkrátit zabíjením času.
	Dva tuláci spolu konverzují, vyměňují si repliky jako při ping pongovém zápasu.
	Účelem tohoto dialogu není sdělení.
	Není jím ani porozumění.
	Nýbrž pouhopouhé zabíjení času.
	}
	\vspace{-2pt}
	\waybox{O životě autora:}{
	Celým jménem \textbl{Samuel Barclay Beckett} žil skoro po celou délku \wemph{dvacátého století}.
	Narodil se 8 let před začátkem \texthl{Velké války}, v začátku \texthl{Druhé světové války} mu bylo 33 let.
	Zemřel v 83 letech, 11 let před začátkem nového milénia.
	Studoval v \texthl{Dublinu}, kde studoval práva, francouzštinu a italštinu.
	Do psaní se dostal s poezií; 8veršovou sbírkou \textbl{Děvkoskop}.
	4 roky před začátkem \texthl{Druhé světové války} se léčil na psychiatrické léčebně v \texthl{Londýně} se svými \wemph{depresemi} po úmrtí otce.
	Do románového světa se dostal se svým dílem \textbl{Murphy}, který líčí vnitřní rozkol mezi touhou po prostitutce a útěku do svého vnitřního temného světa.
	Během války se~angažoval v protifašistickém odboji -- francouzském hnutí odporu.
	Roku \texthl{69} mu byla udělena \wemph{Nobelova cena za literaturu}, kterou tedy moc neocenil kvůli zvýšené pozornosti ze strany médií.
	Některé zdroje uvádějí, že~všechny peníze, spojené s touto prestižní cenou, rozdal charitím (nutno potvrdit).\\
	\vspace{0.3em}\\
	Napsal 13 divadelních her, z nich většinu v padesátých a šedesátých letech.
	Za významnou je považováno hlavně \textbl{Čekání na Godota}.
	V tomto období napsal i šest ze svých osmi prozaických děl a jednu rozhlasovou hru.
	Napsal také dvě poezijní díla, jedno na začátku své kariéry, \textbl{Děvkoskop} a druhé ke konci svého života, \textbl{Společnost}.\\
	\vspace{0.3em}\\
	Jmenovitě jde zmínit divadelní hra \textbl{Konec hry}; mající jeden akt a 4 postavy, apokalyptické prostředí; absurdní dialog a \wemph{lidská bezmoc} tu je divákovi představena skrz černý humor.
	Dále i \textbl{Šťastné dny}, kde je postava \textbl{Winnie} zahrabána v zemi a navzdory všemu neztráci optimismus; \wemph{proč?}
	A nakonec \textbl{Moloy}, román, sledující vnitřní monolog tuláka.
	Beckettovi byla očividně blízká témata cestovatelů.
	}
	\vfill
	\begin{flushright}
	\footnotesize Vypracoval \textbl{Libor Halík} v sázecím systému \wemph{\hologo{LuaLaTeX}}.
	\end{flushright}
\end{document}