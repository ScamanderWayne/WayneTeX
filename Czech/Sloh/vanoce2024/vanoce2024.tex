\documentclass{article}
\usepackage{wsloh}

\begin{document}
\setws{Naděje}{Libor Halík}
\coffeestainA{0.4}{1}{90}{28em}{38em}

\yinipar{V}ánoce jsou již blizoučko a s tím se nám blíží i období lásky a naděje.
Všichni toto období zbožňujeme, však se jedná o období blikajících světélek, co narušují klidnou noc;
předražených maličkostí z plastu, které se při první příležitosti zahodí;
prostě krásné oslavy křesťanských tradic za dopomoci přímého ignorování křesťanských hodnot.
Oslavy narození našeho pána a spasitele spojené s~dopouštěním se~přinejlepším jednoho smrtelného hříchu je krásnou tradicí, která podtrhuje koncept naděje. \par
Už samotná podstata naděje si protiřečí.
\enquote{\textit{Naděje}, je optimistický stav mysli, který je založen na očekávání kladných výsledků s ohledem na události a okolnosti v životě člověka nebo na celém světě.}
Nynějšek není nadějný, protože koncept naděje nepracuje s realitou.
Naděje je lhář;
deluzijní umělec malující lepší zítřek na stěnu stinné jeskyně;
mozaika sebeklamu, co vábí svými pestrými barvami ale odrazuje pachem hniloby.
Pro~naději existuje na sto synonym a žádné z nich není kladné.
Jen tu naději si~lověk zachovává, ze dna Pandořini skřínky, té skřínky, co obsahovala největší pohromy lidstva. \par

\newpage

\yinipar{O}i

\end{document}