\documentclass[A4paper]{extarticle} % If I am not mistaken it is used for wider range of font sizes
\usepackage[T1]{fontenc} % I have no idea why it is here
\usepackage[utf8]{inputenc} % I have no idea why it is here
\usepackage[czech]{babel} % Used for czech formatting of text
\usepackage{csquotes} % Used for quoting
% \usepackage{graphicx} % Required for inserting images
\usepackage{moresize} % Even wider range of font sizes
\usepackage{nopageno} % Override text numbering
\usepackage[left=1cm,right=1cm,top=1.5cm,bottom=1.5cm]{geometry} % To tell truth I do not know how it works with A4paper in documentclass but it works
% \usepackage{multicol} % multi columns environment
\usepackage{titlesec} % please do not ask
\usepackage{wrapfig} % wrapfigure environment
\usepackage{lipsum} % for nothing appearantly
\usepackage{xcolor} % colors everywhere
\usepackage{fontsize} % ...I think for more font sizes as well...?

\usepackage{coffeestains} % no questions needed

\titleformat{\part}
{\normalfont\Large\bfseries}{\thepart}{}{}
\titleformat{\section}
{\normalfont\large\bfseries}{\thesection}{}{}
\titleformat{\subsection}
{\normalfont\small\bfseries}{\thesubsection}{}{}

\titlespacing{\part}{0pt}{0pt}{0pt}
\titlespacing{\section}{0pt}{0pt}{0pt}
\titlespacing{\subsection}{1em}{0pt}{0pt}

\setlength{\parindent}{0pt}
\setlength{\parskip}{0pt}

\usepackage[
    pdftex,
    pdfauthor={Libor Halík},
    pdftitle={Povinná literatura k maturitní zkoušce},
    pdfsubject={Materiály z literatury k maturitní zkoušce, studijní projekt z období 2021 až 2025}]{hyperref}
% used to add metadata to PDF file

% --------------------------------------------------
% S E Z N A M   L I T E R A T U R Y 
% --------------------------------------------------

\begin{document}

% \changefontsize{8pt}

\coffeestainA{0.4}{1}{90}{30em}{40em}

\part*{Seznam literatury {\hfill \normalfont\tiny\textsc{Libor Halík}}}

\section*{Světová a česká literatura do konce 18. století {\tiny{(min. 2 literární díla)}}}

\subsection*{Epos o Gilgamešovi {\tiny{poezie  -- \textsc{TO--READ}}}}

\subsection*{Hamlet {\tiny{drama  -- \textsc{ke zpracování}}}}

\subsection*{Romeo a Julie {\tiny{drama  -- \textsc{ke zpracování}}}}

\section*{Světová a česká literatura 19. století {\tiny{(min. 3 literární díla)}}}

\subsection*{Havran {\tiny{poezie  -- \textsc{hotovo}}}}

\subsection*{Kytice {\tiny{poezie  -- \textsc{hotovo}}}}

\subsection*{Zločin a trest {\tiny{próza  -- \textsc{hotovo}}}}

\subsection*{Otec Goriot {\tiny{próza  -- \textsc{TO--READ}}}}

\subsection*{Oliver Twist {\tiny{próza  -- \textsc{TO--READ}}}}

\subsection*{Obraz Doriana Graye {\tiny{próza  -- \textsc{hotovo}}}}

\subsection*{Revizor {\tiny{drama  -- \textsc{ke zpracování}}}}

\section*{Světová literatura 20. -- 21. století {\tiny{(min. 4 literární díla)}}}

\subsection*{Inferno {\tiny{próza  -- \textsc{TO--READ}}}}

\subsection*{Šifra Mistra Leonarda {\tiny{próza  -- \textsc{ke zpracování}}}}

\subsection*{Velký Gatsby {\tiny{próza  -- \textsc{ke zpracování}}}}

\subsection*{Devatenáct set osmdesát čtyři {\tiny{próza  -- \textsc{hotovo}}}}

\subsection*{Harry Potter {\tiny{próza  -- \textsc{ke zpracování}}}}

\subsection*{Hobbit {\tiny{próza  -- \textsc{ke zpracování}}}}

\section*{Česká literatura 20. -- 21. století {\tiny{(min. 5 literární díla)}}}

\subsection*{Po nás ať přijde potopa {\tiny{poezie  -- \textsc{TO--READ}}}}

\subsection*{Proměna {\tiny{próza -- \textsc{ke zpracování}}}}

\subsection*{Krysař {\tiny{próza  -- \textsc{ke zpracování}}}}

\subsection*{Bílá nemoc {\tiny{drama  -- \textsc{ke zpracování}}}}

\subsection*{R. U. R. {\tiny{drama  -- \textsc{TO--READ}}}}

\vfill

\noindent\begin{minipage}{\textwidth}
    \textcolor{darkgray}{\rule{\linewidth}{0.4pt}
    \footnotesize
    \textbf{Pravidla --} Dodržet minimální počet literatury z daného období; mít minimálně dvakrát poezii, dvakrát prózu a dvakrát drama; nemít více než dvě díla od jednoho autora.
    }
\end{minipage}

\pagestyle{empty}

% --------------------------------------------------
% Č Á S T   P R V N Í
% --------------------------------------------------

\newpage

% \coffeestainA{0.4}{1}{90}{30em}{40em}

\changefontsize{8pt}

\part*{WILLIAM SHAKESPEARE -- Hamlet {\hfill \normalfont\tiny\textsc{Libor Halík}}}

\noindent\begin{wrapfigure}{r}{0.2\textwidth}
\tiny

\subsection*{Ukázka z díla:}
\setlength{\parindent}{3pt}
\noindent \textsc{Ham.:} Tož nikdy nezjevte, co této noci uviděli jste. \\
\textsc{Hor. a Mar.:} To nezjevíme, pane! \\
\textsc{Ham.:} Přísahejte. \\
\textsc{Hor.:} Já věru, pane, slova neřeknu. \\
\textsc{Mar.:} Já, princi, také ne. \\
\textsc{Ham.:} Zde, na můj meč. \\
\textsc{Mar.:} Vždyť, pane můj, jsme přísahali již. \\
\textsc{Ham.:} Zde, pravím, na můj meč. \\
\textsc{Duch} (pod zemí).: Přísahejte. \\
\end{wrapfigure}

\section*{LITERÁRNÍ TEORIE}

\subsection*{Literární druh a žánr:}
\noindent drama, tragédie

% \subsection*{Literární směr:}

\subsection*{Jazyk a slovní zásoba:}
\noindent Hra je psána spisovným až vytříbeným jazykem.
Postavy od dvora mluví vznešeně.
Autor dodává postavám charakter skrz užitý jazyk. \\
\textsc{Hamlet} je velmi mluvnicky zdatný.
V jeho řeči se vyskytuje ironie až sarkasmus.
Metaforami, slovními hříčkami a~dvojsmysly dodává na důvěře své předstírané šílenosti.
Užívá i mnohých básnických figur, kupříkladu anafory, inverzi slovosledu a asyndoty.
\textsc{Laertes}, \textsc{Ofélie} a hlavně \textsc{Claudius} za ním nezaostávají a mluví vznešeně též.
\textsc{Polonius} je skvělou ukázkou charakteru promítnutého skrz řeč.
Využívá nadbytek vznešených slov pro ukázku svého obsáhlého repertoáru a to na úkor samotné myšlenky svého sdělení.
Dokazuje tím svou snahu dokázat své místo u dvoru.
Výjimku z tohoto pravidla tvoří prostý lid, které Shakespeare nechává mluvit prostě až hrubě.  

% \subsection*{Figury:}

% \subsection*{Tropy:}

\subsection*{Stylistická charakteristika textu:}
\noindent \textit{Hamlet}, typicky pro \textit{Shakespearova} díla, je psán blankversem.
Pro rozvinutí postav a jejich motivů slouží monology.
Dialog posouvá děj.
Důraz je na myšlenku a charaktery, děj je druhořadý a podtrhuje charakter postav.
Autor od děje kličkuje a viditelně oddaluje posun.
Což dává postavám čas na rozvinutí.

\subsection*{Postavy:}
\noindent 
\textsc{Hamlet --} princ, syn královny Gertrudy a již zesnulého krále,
následník trůnu, není schopen se vypořádat se ztrátou svého otce o které se později dozvídá,
že nebyla z přirozených důvodů. \\
\textsc{Horacio --} věrný přítel Hamleta, filosof a rozvážný člověk,
s Hamletem diskutuje o jeho názorech, dotazuje se. \\
\textsc{Polonius --} předseda státní rady, věrný koruně. Až moc horlivý,
snaží se králi pomoci se šíleným Hamletem. Na svou horlivost a sebevědomost doplatí. \\
\textsc{Ofelie --} dcera Poloniova, zamilovaná do Hamleta, láskou zmítaná až utrápena je. \\
\textsc{Laertes --} syn Poloniův, bratr Ofelie, který si vyčítá ztrátu otce i sestry
-- kvůli intrikám nového krále nakonec zabíjí Hamleta. \\
\textsc{Claudius --} nový král, manžel Gertrudy, naplánoval smrt starého krále,
svého bratra, pro svoje obohacení a svými intrikami se pokusil zbavit i Hamleta. \\
\textsc{Gertruda --} dánská královna, snaží se uklidnit manžela a ochraňovat syna,
pro svou nerozhodnost zemře. \\
\textsc{Duch starého krále --} jménem také Hamlet,
zapřísahá syna aby vykonal pomstu ale aby nijak neublížil své matce.

\subsection*{Děj:}
\noindent 
Příběh se odehrává v poměrně krátkém časovém úseku, z pohledu historie, od smrti jednoho Hamleta do smrti Hamleta druhého.
Duch starého krále zjevuje, že byl otráven, Hamlet začne předstírat šílenství a během tohoto období usiluje o usvědčení nového krále, svého strýce.
Nakonec ho usvědčí za pomoci kočovných herců, kteří sehrají na královském dvoře onu bratrovraždu a král se sám usvědčí svojí reakcí na tuto scénu.
Hamlet se setká se svojí matkou a probodne muže, co je odposlouchával za tapisérií v domnění, že jde o jeho strýce, zrádného krále Claudia.
Místo toho k zemi padá Polonius, otec jeho milované Ofelie a jejího bratra Laerta, který byl tou dobou na cestách.
To dává králi příležitost se Hamleta zbavit a vykázat pryč, do Anglie.
Tam už Hamletovi domluvil popravu, aniž by o tom věděla jeho matka.
Hamlet ale lstí podstrčí na popravu muže, kteří naopak měli dohlédnout na jeho popravu.
Po čase se vrací zpátky, zrovna na pohřeb Ofelie. Co sama, bez otce, bratra a milého, zhroutila se do šílenství a nakonec utonula.
Laertes se vrátil ještě před Hamletem, akorát aby našel svou sestru šílenou a viděl její poslední chvíle.
Tohoto využil Claudius, který Laerta poštve proti Hamletovi.
Oba muži souhlasí s šermířským soubojem a běsem zaslepený Laertes přijímá otrávenou čepel od Claudia, aby zaručil, že Hamlet toho dne zemře.
Claudius si chce večer pojistit ještě tím, že připraví otrávenou číši, ze které by nechal Hamleta napít, kdyby souboj vyhrál bez jediného škrábance.
Z číše se napije královna Gertruda, Hamlet je opravdu zasažen otrávenou čepelí, muži si čepele vymění a jedem je zasažen i Laertes, ten zjeví králův hrůzný plán Hamletovi a ten otrávenou čepelí probodne krále a do krku mu nalije jeho otrávené víno.
Všichni tedy umírají a Hamlet svými posledními slovy žádá, aby se jeho příběh nadále tradoval po pravdě a bez Claudiových lží.

\subsection*{Kompozice, prostor a čas:}
\noindent 
Chronologický děj v pěti dějstvích vypráví příběh o tragédii prince z Denmarku, ze hradu Elsinor v Dánsku, v počátku 17. století.

% \subsection*{Význam sdělení:}
% \noindent 
% xXx

\section*{LITERÁRNÍ HISTORIE}

% \subsection*{Politická situace:}

% \subsection*{Základní principy fungování společnosti:}

% \subsection*{Kontext dalších druhů umění:}

% \subsection*{Kontext literárního vývoje:}

\subsection*{Autor {\ssmall -- život autora:}}
\noindent 
\textit{William Shakespeare} je dodnes považován za jednu z nejvýznamnějších postav anglické historie.
Jeho dramata jsou světoznámá, do svých děl vymyslel spoustu slov a slovních spojení, které se později stali základem anglického jazyka. \par
Shakespearova tvorba se dělí do tří období jeho života.
Prvotní, což obsahuje díla s relativně nadějným námětem, patří sem i právě \textit{Romeo a Julie}.
V dalším období nahrazuje vtip za ironii a optimismus za rozpor světa, sem patří i \textit{Hamlet}.
A poslední období obsahuje příběhy s pohádkovými náměty, komedie i tragédie, z tohoto období vzešla kupříkladu \textit{Zimní pohádka}.

\subsection*{Vlivy na dané dílo:}
\noindent
\textit{Shakespeare} se pro svůj příběh inspiroval starou norskou legendou.
Stejně jako u jiných svých významných děl tedy vzal příběh a ve svém podání z něj udělal hit.
\textit{Gesta Danorum}, či \textit{Činy Dánů}, jak je šestnáctidílná série napsaná \textit{Saxo Grammaticus} známá u nás, popisuje historii Dánska a knihy třetí a čtvrtá pojednávají právě o princi, který se stal předlohou pro \textit{Hamleta}.

\subsection*{Další autorova tvorba:}
\noindent 
Mezi další významná díla patří \textit{Romeo a Julie}, \textit{Othello}, \textit{Sen noci svatojánské} a \textit{Zimní pohádka}.

\section*{LITERÁRNÍ KRITIKA}

\subsection*{Dobové vnímání díla a jeho proměny:}
\noindent
xXx

\subsection*{Jak dílo inspirovalo další vývoj literatury:}
\noindent
xXx

% \subsection*{Aktuálnost tématu a zpracování tématu:}


\section*{OSOBNÍ NÁZOR}
\noindent 
xXx

\vfill

\noindent\begin{minipage}{\textwidth}
    \textcolor{darkgray}{\rule{\linewidth}{0.4pt}
    \footnotesize
    \textbf{Blankvers --} xXx
    \textbf{Metafora --} xXx
    \textbf{Anafora --} xXx
    \textbf{Asyndota --} xXx
    }
\end{minipage}

\newpage

% \coffeestainA{0.4}{1}{90}{30em}{40em}

\changefontsize{8pt}

\part*{WILLIAM SHAKESPEARE -- Romeo a Julie {\hfill \normalfont\tiny\textsc{Libor Halík}}}

\noindent\begin{wrapfigure}{r}{0.35\textwidth}
\tiny

\subsection*{Ukázka z díla:}
\setlength{\parindent}{3pt}
\noindent \textit{Jul.} Ach, žel! \\
\textit{Rom.} Teď mluví! -- Promluv poznovu, \par
ó jasný anděle! -- neb nad hlavou \par
mi vystupuješ zářná z noční tmy, \par
jak okřídlený posel nebeský \par
v sloup obráceným očím žasnoucích \par
smrtelných lidí, padajících na znak, \par
by dívali se naň, jak osedlal \par
si lenivě plynoucí oblaky \par
a po nebeských ňadrech vesluje. \\
\textit{Jul.} Romeo, Ó Romeo! -- proč's Romeo? \par
Své jméno zapři, otce zřekni se, \par
neb, nechceš-li, mně lásku přísahej, \par
a nechci dál být Capuletova. \\
\textit{Rom.} (stranou). Mám dále naslouchat, či, promluvit? \\
\textit{Jul.} Jen jméno tvé mým nepřítelem jest; \par
ty's jen ty sám a nejsi Montekem. \par
Co jest to Montek? ruka, ani noha, \par
ni paž, ni tvář, ni jiná část, jež vlastní \par
jest člověku. Ó, jiné jméno měj! \par
Co jest to jméno? To, co růží zvem, \par
pod jiným jménem sladce vonělo \par
by zrovna tak. A tak Romeem nezván, \par
Romeo podržel by veškerou \par
tu vzácnou dokonalost, kterou má \par
bez toho jména. -- Odlož jméno své, \par
a za své jméno, jež tvou částí není, \par
mne vezmi celou. \\
\textit{Rom.} Za slovo tě beru. \par
Jen láskou svou mne zvi a na novo \par
tak budu pokřtěn; od té chvíle dál \par
Romeo nechci býti nikdy víc. 
\end{wrapfigure}

\section*{LITERÁRNÍ TEORIE}

\subsection*{Literární druh a žánr:}
\noindent xXx

% \subsection*{Literární směr:}

% \subsection*{Jazyk a slovní zásoba:}

\subsection*{Figury:}
\noindent 
\enquote{xXx}

\subsection*{Tropy:}
\noindent 
xXx

\subsection*{Stylistická charakteristika textu:}
\noindent 
xXx

\subsection*{Postavy:}
\noindent 
\textsc{Romeo Montek --} teenager, nejmladší potomek rodu Monteků,  \\
\textsc{Julie Kapuletová --} mladší než Romeo, mladá slečna, co se zamiluje do Romea, \\
\textsc{Tybalt --} xXx \\
\textsc{Merkucio --} xXx \\
\textsc{kněz --} xXx \\
\textsc{chůva --} xXx

\subsection*{Děj:}
\noindent 
xXx

\subsection*{Kompozice, prostor a čas:}
\noindent 
xXx

\subsection*{Význam sdělení:}
\noindent 
xXx

\section*{LITERÁRNÍ HISTORIE}

% \subsection*{Politická situace:}

% \subsection*{Základní principy fungování společnosti:}

% \subsection*{Kontext dalších druhů umění:}

% \subsection*{Kontext literárního vývoje:}

\subsection*{Autor {\ssmall -- život autora:}}
\noindent 
\textsc{William Shakespeare} je dodnes považován za jednu z nejvýznamnějších postav anglické historie.
Jeho dramata jsou světoznámá, do svých děl vymyslel spoustu slov a slovních spojení, které se později stali základem anglického jazyka. \par
Shakespearova tvorba se dělí do tří období jeho života.
Prvotní, což obsahuje díla s relativně nadějným námětem, patří sem i právě \textsc{Romeo a Julie}.
V dalším období nahrazuje vtip za ironii a optimismus za rozpor světa, sem patří i \textsc{Hamlet}.
A poslední období obsahuje příběhy s pohádkovými náměty, komedie i tragédie, z tohoto období vzešla kupříkladu \textsc{Zimní pohádka}.

% \subsection*{Vlivy na dané dílo:}

\subsection*{Další autorova tvorba:}
\noindent 
Mezi další významná díla patří \textsc{Hamlet}, \textsc{Othello}, \textsc{Sen noci svatojánské} a \textsc{Zimní pohádka}.

% \subsection*{Jak dílo inspirovalo další vývoj literatury:}

% \section*{LITERÁRNÍ KRITIKA}

% \subsection*{Dobové vnímání díla a jeho proměny:}

% \subsection*{Aktuálnost tématu a zpracování tématu:}


\section*{OSOBNÍ NÁZOR}
\noindent 
xXx

\vfill

\noindent\begin{minipage}{\textwidth}
    \textcolor{darkgray}{\rule{\linewidth}{0.4pt}
    \footnotesize
    \textbf{Placeholder --} xXx
    }
\end{minipage}

% --------------------------------------------------
% Č Á S T   D R U H Á
% --------------------------------------------------

\newpage

% \coffeestainA{0.4}{1}{90}{30em}{40em}

\changefontsize{7pt}

\part*{EDGAR ALLAN POE -- Havran {\hfill \normalfont\tiny\textsc{Libor Halík}}}

\noindent\begin{wrapfigure}{r}{0.35\textwidth}
\tiny

\subsection*{Ukázka z díla:}
\setlength{\parindent}{3pt}
\begin{center}
\noindent
Vyrazil jsem okenici, když tu s velkou motanicí \\
vstoupil starodávný havran z dob, jež jsou tak záslužné; \\
bez poklony, bez váhání, vznešeně jak pán či paní \\
usadil se znenadání v póze velmi výhružné \\
na poprsí Pallady - a v póze velmi výhružné \\
si sedl jen a víc už ne. 
\end{center}
\end{wrapfigure}

\section*{LITERÁRNÍ TEORIE}

\subsection*{Literární druh a žánr:}
\noindent 
Lyricko-epická báseň, spíše balada.

% \subsection*{Literární směr:}

% \subsection*{Jazyk a slovní zásoba:}

% \subsection*{Figury:}

% \subsection*{Tropy:}

\subsection*{Stylistická charakteristika textu:}
\noindent 
Jedná se o krátkou barvitou báseň, kdy autor užívá popisů pro navození atmosféry temného pokoje. Je zde kontrast mezi klidným pokojem a bouří za okny. I přes krátkost básně se autorovi daří bravurně popsat atmosféru i psychické rozpoložení hlavní postavy, která je promítnutím samotného autora.

\subsection*{Postavy:} 
\noindent
\begin{itemize}
    \item [--] \textsc{Vypravěč} -- Autor sám, truchlí nad smrtí Lenory, své milované, stejně jako autor truchlil nad smrtí své ženy. 
    \item [--] \textsc{Havran} -- Tajemný pták, co přiletí z bouřlivé noci a opakuje jedno slovo, které ho klidně mohl naučit majitel, \enquote{nevermore} (v překladu jde většinou o dvou či tří slovní frázi, v mé verzi \enquote{nikdy více}). Jak autor propadá ve svých myšlenkách hloubš a hloubš, připisuje havranovým slovům větší a větší smysl.
    \item [--] \textsc{Lenora} -- Mrtvá milenka, není zde o ní moc řečeno, než to, že nad ní hlavní postava truchlí.
\end{itemize}

\subsection*{Děj:}
{\setlength{\parindent}{0pt}\setlength{\parskip}{0.5em}
Ocitáme se v pokoji učence, starého muže, kterého trápí ztráta milenky a snaží se zapomenout s nosem v knížce. Zatímco je muž uvnitř svého malého kruhu míru, tak za okny zuří bouřka.

Ťuk, ťuk, ťuk. Muže z dlení vytrhne zaťukání, poleká se, racionálně se snaží přesvědčit, že o nic nejde, jde otevřít dveře a tam nikdo.

Ťuk, ťuk, ťuk. Zlověstné zaťukání bez původu. Muž otevírá okenici, čímž je jeho klid narušen už ničím nezjemněnou bouří. Z tohoto nehezkého prostředí do pokoje vlétá Havran. (V originále Krkavec, ale překladatelé se rozhodli, že nevíme, co krkavec je a že havran je u nás více v povědomí jako stvoření Satanovo, čehož autor se snaží docílit. Nevíme, zda byl poslán z druhé strany, zda jenom blouzníme ze zmařené lásky společně s autorem, či zda k nám opravdu přišel posel se zprávou. Havran, i přes veškerou symboliku, může být jen ptákem, který se chtěl jít schovat do teplého příbytku před bouří. Také ale mohl muži přinést důležitou zprávu.) Havran se usadí na poprsí Pallady (socha Pallas Athény) což muži připadá ironické a i se tomu zasměje. Další várka symbolismu.

Muž se ptá na různé otázky, kdy vždy dostane odpověď zápornou a konečnou, \enquote{nevermore}. Ptá se dál, kdy vždy dostává stejnou odpověď a propadá čím dál tím hloubš do sebe trýzně.
}

\subsection*{Kompozice, prostor a čas:}
\noindent 
Báseň o krátkém časovém období, psána chronologicky. 18 slok a pouze 108 veršů. Jde-li skutečně o prakticky auto-biografii, vypsání z depresí básníka, tak by se báseň odehrávala v devatenáctém století.

% \subsection*{Význam sdělení:}

\section*{LITERÁRNÍ HISTORIE}

% \subsection*{Politická situace:}

% \subsection*{Základní principy fungování společnosti:}

% \subsection*{Kontext dalších druhů umění:}

% \subsection*{Kontext literárního vývoje:}

\subsection*{Autor {\ssmall -- život autora:}}
\noindent 
Edgar Allan Poe vedl celkem bídný život. Otec alkoholik se na ně vykašlal, matka zemřela, když byl mladý. Byl umístěn do dětstkého domova, odkud si ho převzal bohatý obchodník a umožnil mu studium. Pár let žili spolu i v Anglii, která se mu mnohokrát stala inspirací pro svojí práci. Díky své povaze a možná i vzpomínkách byl však vysoce nevyrovnaný. Propadal hráčství a závislostem, i když i za života sklidil slávu, tak nikdy nebyl přímo bohatý. Nejspíš i toto mu pomohlo napsat jeho skvosty.

Edgar Allan Poe byl i kritizován za své manželství se svojí sestřenicí. V době svatby jemu bylo sedmadvacet a jí třináct. Dle určitých zdrojů se jejich vztah podobal spíše sourozeneckému než manželskému. Virginia Poe zemřela mladá na nemoc.

% \subsection*{Vlivy na dané dílo:}

\subsection*{Další autorova tvorba:}
\noindent 
\begin{itemize}
    \item Rukopis nalezený v láhvi
    \item Filozofie básnické skladby
    \item Příběhy Arthura Gordona Pyma z Nantucketu
    
\end{itemize}

% \subsection*{Jak dílo inspirovalo další vývoj literatury:}

% \section*{LITERÁRNÍ KRITIKA}

% \subsection*{Dobové vnímání díla a jeho proměny:}

% \subsection*{Aktuálnost tématu a zpracování tématu:}


\section*{OSOBNÍ NÁZOR}
\noindent 
Havran je bravurně zpracovanou básní, která zvládá dokonale obrazutvornost. I týdny po přečtení si jsem schopen vybavit pochmurnost básníkova pokoje a jak jsem soucítil s jeho trápením, které jsem ani nezažil.

\newpage

% \coffeestainA{0.4}{1}{90}{30em}{40em}

\changefontsize{7pt}

\part*{KAREL JAROMÍR ERBEN -- Kytice {\hfill \normalfont\tiny\textsc{Libor Halík}}}

\noindent\begin{wrapfigure}{r}{0.35\textwidth}
\tiny

\subsection*{Ukázka z díla:}
\setlength{\parindent}{3pt}
Pro ukázku jsem vybral \textsc{Záhořovo lože}
\begin{center}
\noindent
Rozlítil se satan a v zlosti své velí: \\
„Vykoupejte jeho v pekelné koupeli!“ \\
Učinila rota dle jeho rozkazu, \\
připravila lázeň z ohně a mrazu: \\
z jedné strany hoří jako uhel vzňatý, \\
z druhé strany mrzne v kámen ledovatý; \\
a když vidí rota míru naplněnu, \\
obrací zmrzlinu opak do plamenů. \\
Strašlivě řve ďábel, jako had se svíjí, \\
a ho pak již smysl i cit pomíjí. \\
Tu pokynul satan, rota odstoupila, \\
a síla zas nová ďábla oživila. \\
Ale když propuštěn opět dýše lehce, \\
krví psané blány přec vydati nechce. -
\end{center}
\vspace{3em}
\end{wrapfigure}

\section*{LITERÁRNÍ TEORIE}

\subsection*{Literární druh a žánr:}
\noindent 
Lyrickoepická skladba básní.
Žánrově jsou každá originál,
tedy příběhy jako \textsc{Zlatý kolovrat} jsou spíše balady,
\textsc{Věšťkyně} je jasná pověst
a \textsc{Záhořovo lože} je legendou.
Celá \textsc{Kytice}, ne~jenom první báseň stejného jména, je dílem poezijním.
Soubor dvanácti děl (plus \textsc{Lilie}, ta~se~udává zvlášť, kvůli svému zvláštnímu stavu).

% \subsection*{Literární směr:}

\subsection*{Jazyk a slovní zásoba:}
\noindent
Úsporný jazyk, který zde budí napětí, je také jedním z hlavních rysů balad. Epický příběh je psán spíše kratšími větami. Citoslovce zde nahrazují slovesa. Básně jsou psány gnónickým veršem. \\
{\small
\textbf{Zvukomalba} -- \textsc{Vodník}, \enquote{a na to pole podle skal zelený mužík zatleskal.} \\
\textbf{Epiteton} -- \textsc{Kytice}, \enquote{Ve skrovnou kytici já tě zavážu, ozdobně stužkou ovinu.} \\
\textbf{Anafora} -- \textsc{Zlatý kolovrat}, \enquote{Kde's má Dorničko! Kde jsi? Kde Jsi?
Kde's má roztomilá?}\\
\textbf{Epifora} -- \textsc{Svatební košile}, \enquote{Co to máš na té tkaničce,
na krku na té tkaničce?}\\
\textbf{Epizeuxis} -- \textsc{Holoubek}, \enquote{Běží časy, běží,
Všecko sebou mění.}\\
\textbf{Apostrofa} -- \textsc{Kytice}, \enquote{Mateří-douško vlasti naší milé,
Vy prosté naše pověsti.}\\
\textbf{Metafora} -- \textsc{Vodník}, \enquote{ale kůra srdce tvého
ničím neobměkla!}\\
\textbf{Metonymie} -- \textsc{Vodník}, \enquote{Mladosti mé jarý štěp
přelomil jsi v půli.}\\
\textbf{Personifikace} -- \textsc{Vodník}, \enquote{Bílé šatičky smutek tají,
v perlách se slzy ukrývají.}\\
\textbf{Přirovnání} -- \textsc{Vodník}, \enquote{v sukničku jako z vodních perel,
nechoď, dceruško, k vodě ven.}\\
}
% \subsection*{Figury:}

% \subsection*{Tropy:}
\vspace{-1em}
\subsection*{Stylistická charakteristika textu:}
\noindent 
Rýmovaný text. Autor jednotlivé části rozdělil do pozic tak, ži vznikla zrcadlová souměrnost vlastností. Tedy \textsc{Kytice} a \textsc{Věštkyně} jsou o naději, \textsc{Polednice} a~\textsc{Vodník} o nadpřirozených bytostech, které působí záporně, \textsc{Poklad} a \textsc{Dceřina kletba} jsou o narušení vztahu mezi matkou a dítětem, \textsc{Svatební košile} a \textsc{Vrba} o určité přeměně člověka, \textsc{Zlatý kolovrat} a~\textsc{Záhořovo lože} mají hned několik společného; vina, pokání, vykoupení; \textsc{Štědrý den} a \textsc{Holoubek} jsou o~lásce a~smrti, \textsc{Lilie} se pak někdy dává na místo \textsc{Vrby}, kde také naplňuje příběh o proměně člověka, ale zpravidla se do této dvojice nezapočítává. Zrovna \textsc{Holoubek} je zároveň přesně v polovině, k čemuž mu dopomáhá bonusová třináctá povídka a představuje pomyslný přechod mezi novými a starými pověstmi.

% \subsection*{Postavy:} 


\subsection*{Děj:}
{\setlength{\parindent}{5pt}
\textsc{Kytice} je příběh o naději. Matka zemře, děti jí oplakávají, tak se matka přemění na mateřídouškum aby jim mohla nadále být na blízku a mysticky ochraňovat.

\textsc{Svatební košile} představuje klasické poučení \enquote{Dej si pozor, co si přeješ, mohlo by se ti to splnit}. Dívka se modlí k paně Marii za svého milého, který odešel do války a kterého si chtěla vzít. Milý zaťukal na dveře a vede jí do kostela. Na cestě jí nutí se zříct Boha a víry, hned několika způsoby. Ona vždy poslušně splní, jak žádá, s vidinou svatby. Nakonec jí však dojde, co se děje a před svým nemrtvým snoubencem se schovává v kostele. Tady je~ještě menší zápas dobra se zlem, kdy se dívka a snoubenec oba snaží přesvědčit mrtvolu v kostele k poslušnosti. Nemrtvý k oddanosti a panna žádá Boha, aby ho nechal spát.

\textsc{Vodník} Dívka i přes matčiny varování jde k rybníku, kde jí chytí vodník a veme si jí za ženu. Dívka mu porodí dítě, ale furt je nespokojená. Žádá ho, aby~se~mohla naposledy vidět s maminkou. Vodník nakonec vyhoví, ale~matka dceru přesvědčí, aby se už nevracela. Vodník burácí, vyhrožuje, ale~dívka se~nevrací. Tak jí jako poslední rozloučení nechá přede dveřmi dítě ve dví.

\textsc{Záhořovo Lože} je druhou nejdelší básní ve sbírce. Mladík byl otcem zaprodán ďáblu. S vírou v Boha se tak dyvává na cestu do pekla. Na této cestě potkává Záhoře, velkého loupežníka, který zabil všechny pocestné před ním. Mladík se s ním domluví, že když ho nechá jít, tak mu povypráví, jaké je~to~v~pekle. Po roce se setkávají a mladík vypráví, jak sám Satan při výhružce křížem rozkázal ďáblu listinu vydat. Ten odmítl i po mučení. Až~nakonec Satan rozkázal k něčemu, zvané \enquote{lože Záhořovo}. Loupežník se zděsí a odhaluje, že to on je Záhoř, pro kterého už tedy v pekle mají přichystané nejhorší mučení ze všech. Však i ďábel, který vydržel do té doby všechno, se zděsil a úpis vydal. Mladík mu radí, ať se dá na víru v Boha a prosí o odpuštění. Také zarazí jeho kyj do země a nařizuje, ať se kaje u toho kyje. Po letech se tam vrací jako biskup, Záhoř se mezitím proměnil v~pařez. Oživá při~pohledu na mladíka, který už není mladíkem a tím jejich cesty končí dalším shledáním. Jejich těla se rozpadnou v prach a jejich duše se změní v bílé holubice směřující k nebi.
}
\subsection*{Kompozice, prostor a čas:}
\noindent 
Žádný z příběhů nemá přesně určené období ani místo děje. I když většina se aspoň částečně odehrává na vesnici.

% \subsection*{Význam sdělení:}

\section*{LITERÁRNÍ HISTORIE}

% \subsection*{Politická situace:}

% \subsection*{Základní principy fungování společnosti:}

% \subsection*{Kontext dalších druhů umění:}

% \subsection*{Kontext literárního vývoje:}

\subsection*{Autor {\ssmall -- život autora:}}
\noindent 
Vystudoval filozofii a práva, snažil se o obnovu českého ducha, tak sepisoval české pověsti. Také usiloval o rovnoprávnost s Němci. České pověsti sbíral na venkově, kam sám rád jezdil.

% \subsection*{Vlivy na dané dílo:}

\subsection*{Další autorova tvorba:}
\noindent 
\begin{itemize}
    \item Sto prostonárodních pohádek a pověstí slovanských v nářečích původních
    \item Češké pohádky
    \item Vybrané báje a pověsti národní jiných větví slovanských
\end{itemize}

% \subsection*{Jak dílo inspirovalo další vývoj literatury:}

% \section*{LITERÁRNÍ KRITIKA}

% \subsection*{Dobové vnímání díla a jeho proměny:}

% \subsection*{Aktuálnost tématu a zpracování tématu:}


\section*{OSOBNÍ NÁZOR}
\noindent 
Kytice, i když bravurně napsaná a jednotlivé balady jsou velice poutavé, je ještě zajímavější z historického hlediska. Příběhy jsou o individuální vině a určitém nadpřirozeném, nepřiměřeném trestu, což je velice poutavá zápletka, která funguje.

\vfill

\noindent\begin{minipage}{\textwidth}
    \textcolor{darkgray}{\rule{\linewidth}{0.4pt}
    \changefontsize{7pt}
    \footnotesize
    \textsc{Legenda} je příběh světce, v českých zemích je význam legendy úzce spjat s katolickými příběhy o svatích mužích a jejich legendárních činnech. Jinak celosvětově je definičně blíže pověsti.
    \textsc{Pověst} je příběh, který může být založen na pravdě (vypravěč tomu alespoň věří), má blízko legendě i pohádce (ta ale na pravdě založená nebývá). Díky tomu, že pověsti jsou často o činech sedláků na vesnici je nemožné dostopovat jejich původ a zjistit, zda skutečně reálný základ mají.
    \textsc{Balada} je dalším druhem příběhu, který se vyznačuje určitou ponurostí a většinou nešťastným koncem. \\
    \textsc{Zvukomalba} je hlásková nápodoba existujícího zvuku, nejčastěji citoslovce, \enquote{kykyryký}.
    \textsc{Epiteton} varianty \textit{constans} pro ustálené přirovnání jako \enquote{zelený háj} a \textit{ornans} pro ozdobné přirovnání jako \enquote{zemřelá slova}.
    \textsc{Anafora} spočívá v opakování slov, či slovních spojení na začátku, proto \textit{znovuuvedení}.
    \textsc{Epifora} spočívá v opakování slov, či slovních spojení na konci, opak anafory.
    \textsc{Epizeuxis} spočívá v opakování slov, či slovních spojení hned po sobě, \textit{připojení}.
    \textsc{Apostrofa} oslovení jiného diváka, než běžného, k nepřítomné či zemřelé osobě, neživé věci.
    \textsc{Metafora} přenášení významu na základě vnější podobnosti, třeba \enquote{zub pily}, či \enquote{kapka štěstí}.
    \textsc{Metonymie} přenášení významu na základě jiné, než vnější podobnosti, tedy \enquote{stráž -- střežení}, \enquote{cestující} a \enquote{\textit{Praha} to neschválila}.
    \textsc{Personifikace}, \textit{zosobnění}, nejobecněji přenos lidských vlastností na jiné, než lidské bytosti, či humanoidní zevření abstraktních pojmů.
    \textsc{Přirovnání} dvou subjektů na základě určité vlastnosti.
    \textsc{Gnómický}, nadčasový, aktuální, neměnný, nepodléhající změnám v čase.
    }
\end{minipage}

\newpage

% \coffeestainA{0.4}{1}{90}{30em}{40em}

\changefontsize{8pt}

\part*{FJODOR MICHALJEVIČ DOSTOJEVSKIJ -- Zločin a trest {\hfill \normalfont\tiny\textsc{Libor Halík}}}

\noindent\begin{wrapfigure}{r}{0.35\textwidth}
\tiny

\subsection*{Ukázka z díla:}
\setlength{\parindent}{3pt}
Úder dolehl přímo na temeno, její malý vzrůst tomu jen napomohl. Vykřikla,
ale docela slabounce, a znenadání se schoulila k~podlaze, ačkoliv ještě stačila zdvihnout obě ruce. V jedné
pořád ještě svírala „zástavu“. Tu udeřil vší silou ještě jednou, zase tupým koncem a zase do temene. Krev
vychlístla jako z převržené číše a tělo se~zvrátilo naznak. Couvl před padajícím tělem a~ihned se sklonil k její
tváři; byla už mrtvá. Oči byly vytřeštěné, jako by~chtěly vylézt z důlků, čelo a~celý obličej měla svraštělý a~zkřivený křečí.

Položil sekyru na zem vedle mrtvé a~rychle, ale~opatrně, aby~se~nepotřísnil prýštící krví, strčil ruku
stařeně do kapsy, do~pravé kapsy, z~níž minule vytáhla klíče. Byl naprosto příčetný, mrákoty ani mdloby se~o~něj už nepokoušely, jen ruce se mu pořád ještě třásly. Později se upamatoval, že byl dokonce velmi vnímavý
a~obezřetný a~že~neustále dával pozor, aby~se~nepotřísnil…
\end{wrapfigure}

\section*{LITERÁRNÍ TEORIE}

\subsection*{Literární druh a žánr:}
\noindent Próza a psychologický román

% \subsection*{Literární směr:}

% \subsection*{Jazyk a slovní zásoba:}

\subsection*{Figury:}
\noindent 
\enquote{Bože můj,} prosil, \enquote{ukaž mi cestu...} (apostrofa) \\
\enquote{a až zítra, zítra si všechno... Kdy jste přijely, už dávno?} (epizeuxis a aposiopese)

\subsection*{Tropy:}
\noindent 
Byl chudý jako kostelní myš (přirovnání) \\
Ach vy milá, nespravedlivá srdce! (synekdocha)

\subsection*{Stylistická charakteristika textu:}
\noindent 
Chronologický postup vyprávění děje er-formou od začátku do konce. Výjimkou jsou retrospektivní sny a vzpomínky u~kterých chronologický postup neplatí a vnitřní monology postav u~kterých se přechází do ich-formy. Text je rozdělen do sedmi částí, z nichž poslední je epilog.

\subsection*{Postavy:}
\noindent 
\textsc{Rodion Romanovič Raskolnikov --} rozpolcená hlavní postava, jak už i trefné jméno napovídá. Jedná se o mladého muže, studenta práv, který žil dlouhou dobu v chudobě a celkově špatných podmínkách. Začal tedy hledat určitý ideál a napadla ho teorie prosazení zločinu, kdy všichni géniové museli spáchat zločin, díky omezenosti zákonů a~prošlo jim to jen proto, že dokázali, že mohou. Jeho nálada byla velice proměnlivá a často vedl monology. \\
\textsc{Dmitrij Prokofjič Razumichin --} skrz na skrz dobrý člověk. Inteligentní, ale dobrý, díky tomu i trochu přihlouplý, protože odmítá pochopit myšlení vypočítavých lidí. Raskolnikovi pomáhá od začátku do konce a nakonec se ožení s Raskolnikovou sestrou, protože k ní během knihy zahoří láskou. \\
\textsc{Avdoťja Romanovna Raskolniková --} zkráceně Duňa, Raskolnikova sestra, která ho bezmezně miluje a pro zaručení zářné budoucnosti svému bratrovi, je ochotna vstoupit zcela dobrovolně i do manželství se zlým, však bohatým a vlivným mužem. I když se jí nedostalo vzdělání jako Raskolnikovi, tak je hned v několika chvílích ukázáno, že je neméně inteligentní, než její bratr. \\
\textsc{Pulcherija Alexandrovna Raskolnikovová --} - matka Raskolnikova, starší venkovská žena. Svého syna miluje a to jí činí slepou k jeho aférám. I~když na konci je odhaleno, že přeci jenom věděla více, než kdy dala najevo. \\
\textsc{Sofja Semjonovna Marmeladová --} zkráceně Soňa, postava ze špatných poměrů, stejně jako Raskolnikov, které ale neopustila pravá víra a odevzdanost k Bohu a~která nakonec Raskolnika zachrání právě díky Božímu učení. \\
\textsc{Arkadij Ivanovič Svidrigajlov --} stejně jako Soňa, i Svidrigajlov ukazuje Raskolnikovi zrcadlo. Narozdíl od hodné Sofji, která stále věří v Boha a~neopouští jí to dobré, představuje Svidrigajlov co se z Raskolnika stane, když svůj hřích v sobě bude dusit, či ho dokonce začne těšit. Bezvěrec jenž má na svém krku nespočet životů a nakonec to ukončí sebevraždou. Byl zamilován do Avdoťji. \\
\textsc{Porfirij Petrovič --} velice inteligentní vyšetřující soudce, který Raskolnika prokoukne hned zezačátku a po celou knihu mu činí mentálního protivníka. Jejich dialogy jsou jedny z nejpovedenějších kousků.

\subsection*{Děj:}
\noindent 
Příběh začíná u zdánlivě běžného dne v životě Raskolnika. Pořád rozmyšlí nad něčím, co je mu natolik odporné, že to neadresuje ani ve svých myšlenkách. Uvažuje co ho k tomu vedlo. Přitom ochoří a mysl se mu zamlží. O několika dnech, při kterých trpěl horečkou ani neví a nakonec spáchá~\enquote{to.} Což~je~odhaleno, že se jedná o vraždu stařeny, která půjčuje peníze. Ukradne peníze, ale schová je a jde se dál kurýrovat ze své nemoci. Začíná vyšetřování, Raskolnikov je jeden z předních podezřelých. Do života se mu vrací matka a sestra, přicházejí i noví lidé. mezi ně patří i Soňa, která nakonec vyvede Raskolnika z temnoty, či Svidrigajlov, který po dlouhou dobu funguje spíš jako záhadná postava v pozadí.

\subsection*{Kompozice, prostor a čas:}
\noindent 
První až šestá část se, s výjimkou vzpomínek a snových pasáží, odehrávají v Petrohradě, Epilog pak na Sibiři. Doba se dá odhadnout na příbližně 2.~polovinu 19.~století.

\subsection*{Význam sdělení:}
\noindent 
Zločin a trest je výstižným názvem pro celé téma. Spáchá zločin a následuje trest. Nejde mu uniknout, nejde jen o trest uložen soudem, ale o trestání vlastním svědomím, což je ukázáno hned na několika postavách, mezi nimi hlavně Raskolnikov a Svidrigajlov, kteří oba zvolí rozdílnou cestu ven. Zatímco Svidrigajlov podléhá a únik již vidí jen v sebevraždě, Raskolnikov znovu nalezne Boha a lásku k lidem (převážně k Soně).

\section*{LITERÁRNÍ HISTORIE}

% \subsection*{Politická situace:}

% \subsection*{Základní principy fungování společnosti:}

% \subsection*{Kontext dalších druhů umění:}

% \subsection*{Kontext literárního vývoje:}

\subsection*{Autor {\ssmall -- život autora:}}
\noindent 
Rusky spisovatel, vystudoval vojenské technické učiliště. S celou svojí skupinou byl odsouzen na smrt vojenským soudem, trest jim byl změněn na~nucené práce na Sibiři. Prošel peklem a nápravu shledal v Bohu. Ve svém okolí si všímal jen toho špatného, čehož také využíval ve svých dílech. Věřil, že spása se dá najít jedině v Bohu a to také hlásal dál.

% \subsection*{Vlivy na dané dílo:}

\subsection*{Další autorova tvorba:}
\noindent 
Mezi nejznámnější díla patří \textsc{Zápisky z mrtvého domu}, právě \textsc{Zločin a trest}, \textsc{Idiot Idiot}, či \textsc{Bratři Karamazovi}

% \subsection*{Jak dílo inspirovalo další vývoj literatury:}

% \section*{LITERÁRNÍ KRITIKA}

% \subsection*{Dobové vnímání díla a jeho proměny:}

% \subsection*{Aktuálnost tématu a zpracování tématu:}


\section*{OSOBNÍ NÁZOR}
\noindent 
Kniha je napsaná poutavě, postavy, které ani ve svých myšlenkách neadresují jádro myšlenky mají pro  takové chování své důvody a nikoliv jen pro napětí čtenáře. Myšlenka je tu s námi věčně a Dostojevskij ji dokonale jmenuje.

\vfill

\noindent\begin{minipage}{\textwidth}
    \textcolor{darkgray}{\rule{\linewidth}{0.4pt}
    \footnotesize
    \textbf{Apostrofa --} řečnická figura, ve které postava mluví k nepřítomné osobě. V tomto případě k Bohu.
    \textbf{Epizeuxis --} básnická figura, opakování stejného slova či~slovního spojení. 
    \textbf{Aposiopese --} literární pojem označující odmlku, nedokončení věty.
    \textbf{Synekdocha --} rétorická figura při níž se celek označí názvem jeho části, či část názvem celku. Spřízněná s metonymií.
    \textbf{Metonymie --} přenos označení. Nazvání objektu názvem jiného objektu, dle určité souvislosti.
    }
\end{minipage}

\newpage

% \coffeestainA{0.4}{1}{90}{30em}{40em}

\changefontsize{8pt}

\part*{HONORÉ DE BALZAC -- Otec Goriot {\hfill \normalfont\tiny\textsc{Libor Halík}}}

\noindent\begin{wrapfigure}{r}{0.35\textwidth}
\tiny

\subsection*{Ukázka z díla:}
\setlength{\parindent}{3pt}
xXx
\end{wrapfigure}

\section*{LITERÁRNÍ TEORIE}

\subsection*{Literární druh a žánr:}
\noindent xXx

% \subsection*{Literární směr:}

% \subsection*{Jazyk a slovní zásoba:}

\subsection*{Figury:}
\noindent 
\enquote{xXx}

\subsection*{Tropy:}
\noindent 
xXx

\subsection*{Stylistická charakteristika textu:}
\noindent 
xXx

\subsection*{Postavy:}
\noindent 
\textsc{xXx --} xXx \\

\subsection*{Děj:}
\noindent 
xXx

\subsection*{Kompozice, prostor a čas:}
\noindent 
xXx

\subsection*{Význam sdělení:}
\noindent 
xXx

\section*{LITERÁRNÍ HISTORIE}

% \subsection*{Politická situace:}

% \subsection*{Základní principy fungování společnosti:}

% \subsection*{Kontext dalších druhů umění:}

% \subsection*{Kontext literárního vývoje:}

\subsection*{Autor {\ssmall -- život autora:}}
\noindent 
xXx

% \subsection*{Vlivy na dané dílo:}

\subsection*{Další autorova tvorba:}
\noindent 
Mezi nejznámnější díla patří \textsc{xXx}

% \subsection*{Jak dílo inspirovalo další vývoj literatury:}

% \section*{LITERÁRNÍ KRITIKA}

% \subsection*{Dobové vnímání díla a jeho proměny:}

% \subsection*{Aktuálnost tématu a zpracování tématu:}


\section*{OSOBNÍ NÁZOR}
\noindent 
xXx

\vfill

\noindent\begin{minipage}{\textwidth}
    \textcolor{darkgray}{\rule{\linewidth}{0.4pt}
    \footnotesize
    \textbf{Placeholder --} xXx
    }
\end{minipage}

\newpage

% \coffeestainA{0.4}{1}{90}{30em}{40em}

\changefontsize{8pt}

\part*{CHARLES DICKENS -- Oliver Twist {\hfill \normalfont\tiny\textsc{Libor Halík}}}

\noindent\begin{wrapfigure}{r}{0.35\textwidth}
\tiny

\subsection*{Ukázka z díla:}
\setlength{\parindent}{3pt}
xXx
\end{wrapfigure}

\section*{LITERÁRNÍ TEORIE}

\subsection*{Literární druh a žánr:}
\noindent xXx

% \subsection*{Literární směr:}

% \subsection*{Jazyk a slovní zásoba:}

\subsection*{Figury:}
\noindent 
\enquote{xXx}

\subsection*{Tropy:}
\noindent 
xXx

\subsection*{Stylistická charakteristika textu:}
\noindent 
xXx

\subsection*{Postavy:}
\noindent 
\textsc{xXx --} xXx \\

\subsection*{Děj:}
\noindent 
xXx

\subsection*{Kompozice, prostor a čas:}
\noindent 
xXx

\subsection*{Význam sdělení:}
\noindent 
xXx

\section*{LITERÁRNÍ HISTORIE}

% \subsection*{Politická situace:}

% \subsection*{Základní principy fungování společnosti:}

% \subsection*{Kontext dalších druhů umění:}

% \subsection*{Kontext literárního vývoje:}

\subsection*{Autor {\ssmall -- život autora:}}
\noindent 
xXx

% \subsection*{Vlivy na dané dílo:}

\subsection*{Další autorova tvorba:}
\noindent 
Mezi nejznámnější díla patří \textsc{xXx}

% \subsection*{Jak dílo inspirovalo další vývoj literatury:}

% \section*{LITERÁRNÍ KRITIKA}

% \subsection*{Dobové vnímání díla a jeho proměny:}

% \subsection*{Aktuálnost tématu a zpracování tématu:}


\section*{OSOBNÍ NÁZOR}
\noindent 
xXx

\vfill

\noindent\begin{minipage}{\textwidth}
    \textcolor{darkgray}{\rule{\linewidth}{0.4pt}
    \footnotesize
    \textbf{Placeholder --} xXx
    }
\end{minipage}

\newpage

% \coffeestainA{0.4}{1}{90}{30em}{40em}

\changefontsize{8pt}

\part*{OSCAR WILDE -- Obraz Doriana Graye {\hfill \normalfont\tiny\textsc{Libor Halík}}}

\noindent\begin{wrapfigure}{r}{0.35\textwidth}
\tiny

\subsection*{Ukázka z díla:}
\setlength{\parindent}{3pt}
\noindent
\enquote{Ty nechceš, abych se s ním setkal?} \\
\enquote{Ne.} \\
\enquote{Pan Dorian Gray je v ateliéru, pane,} pravil komorník, vstupující do zahrady. \\
\enquote{Teď mě představit musíš,} řekl lord Henry se smíchem. \\
Malíř se obrátil k sluhovi, který tu stál a pomrkával ve slunečním světle.
\enquote{Požádejte pana Graye, aby počkal, Parkere. Že přijdu za~okamžik.}
Sluha se uklonil a vracel se po pěšině.
Basil se zahleděl na lorda Henryho.
\enquote{Dorian Gray je můj nejdražší přítel,} řekl.
\enquote{Má prostou a krásnou povahu.
Tvoje teta ti o něm řekla pravdu.
Nezkaz ho.
Nesnaž se ho ovlivnit.
Tvůj vliv by byl špatný.
Svět~je~veliký a je v něm plno skvělých lidí.
Neodveď mi tedy jedinou bytost, které moje umění vděčí za veškeré kouzlo, co v něm je.
Závisí na něm můj život umělce.
Na to nezapomeň, Harry, důvěřuji ti.}
Mluvil velmi zvolna a zdálo se, že se ta slova z něho derou téměř proti jeho vůli. \\
\enquote{Co to říkáš za nesmysl!} řekl lord Henry s úsměvem, a uchopiv Hallwarda pod paží, téměř ho vlekl do domu. 
\end{wrapfigure}

\section*{LITERÁRNÍ TEORIE}

\subsection*{Literární druh a žánr:}
\noindent
Román, obsahuje hororové prvky

% \subsection*{Literární směr:}

% \subsection*{Jazyk a slovní zásoba:}

% \subsection*{Figury:}

% \subsection*{Tropy:}

\subsection*{Stylistická charakteristika textu:}
\noindent 
Text je rozdělen do dvaceti kapitol.
Paralérní kompozice děje, kdy se takto seznamujeme se Sibyl Vaneovou a její rodinou.
Autor na začátek ještě přidal předmluvu, která slouží jako jakýsi návod na pochopení jeho díla.

\subsection*{Postavy:}
\noindent 
\textsc{Dorian Gray --} hlavní postava, jehož příběh sledujeme. Začíná jako nádherný a nezkažený mladý muž, ještě adolescent. Díky vlivům svých, Basilových a Lorda Henryho se ale postupně kazí, až spáchá dokonce i vraždu.
Už od útlého věku se u něj projevuje narcisimus a během svého života se propadá stále hloubš za honbou po potešení.
Stává se z něj proutník, narkoman a jeho osobnost upadá. \\
\textsc{Basil Harvard --} idealistický malíř, Graye si všimne jako jeden z prvních a zamiluje se do něj.
I jeho nekonečné velebení Dorianovi osobnosti mohlo mít vliv na jeho budoucnost. \\
\textsc{Lord Henry --} manipulátor, miluje ovlivňovat okolí a sledovat jak se vyvine.
To Lord Henry byl první, který Dorianovi nabídl svět potěšení a i když by si ho narcistický Dorian dost možná našel i sám, tak mu to nikdy nepřestal vyčítat.
Během díla vystupuje jako filozof a inteligentní muž. \\
\textsc{Sibyl Vaneová --} první dívka, jež měla tu smůlu a zamilovala se do Doriana Graye. Také první život, který kdy Dorian vzal a který byl předzvěstí jeho dalšího počínání.

\subsection*{Děj:}
\noindent 
Příběh začíná pohledem na místo Basila Harvarda, ke kterému na návštěvu přišel jeho přítel Lord Henry.
Baví se o mladíkovi jménem Dorian Gray a~tím uvádějí jeho charakter ještě před uvedením jeho postavy do děje.
Po této první kapitole už se děj odehrává výhradně kolem Doriana Graye.
Zamiluje se do Sibyl Vaneové, jen aby podlehl potěšení a nepřímo zavinil její smrt.
Po této zkušenosti propadne požitkářství bohémského života a~postupně se propadá.
Tráví noci ve vykřičených domech a opiových doupatech, okolí se od něj distancuje a to celé zatímco na něj se nezmění ani vráska.
Jeho~modlitba vyřčena toho prvního dne u Basila Harvarda totiž zavinila, že všechny projevy, stáří i morálního úpadku, mění podobu jeho obrazu a~nikoliv podobu Doriana samotného.
Ten, co Doriana neviděl něco z toho dělat ani neuvěří, že ten mile vypadající adolescent je třicetiletý proutník závislý na opiu.
Během tohoto morálního úpadku zabije Basila Harvarda, který namaloval obraz a v jeho očích to celé zavinil.
I Lordu Henrymu vyčítá svojí situaci.
Ale děj nikdy neurčil, kdo ze tří mužů mohl za situaci Doriana Graye, zda on sám, jeho přítelé, či všichni dohromady.
A příběh končí, když se Dorian rozhodne zničit obraz, důkaz o jeho morálním úpadku a tím zabije sám sebe.

\subsection*{Kompozice, prostor a čas:}
\noindent 
Většina děje, až na pár výjimek, se odehrává v Londýně. Doba je circa konec 19. století.

\subsection*{Význam sdělení:}
\noindent 
Autor zde pojednává o muži, jenž se behlavě žene za potěšením a tím nezničí jen sám sebe, ale i lidi kolem sebe.
I když hovoří o umění a kráse potěšení, tak mluví i o trestu, který má být očistou a kterému se Dorian ve své nepřirozenosti vyhýbá.

\section*{LITERÁRNÍ HISTORIE}

% \subsection*{Politická situace:}

% \subsection*{Základní principy fungování společnosti:}

% \subsection*{Kontext dalších druhů umění:}

% \subsection*{Kontext literárního vývoje:}

\subsection*{Autor {\ssmall -- život autora:}}
\noindent 
Oscar Wilde byl jistě geniální muž, který rád dráždil své okolí nemístníma narážkama.
Rád nechával své okolí v nejistotě, zda není homosexuál a to ho přišlo draho, když byl odsouzen za homosexualitu ke dvěma rokům těžkých prací.
Po odsloužení trestu se už nikdy nedal zcela dohromady a umřel v~bídě jen pár let poté.

% \subsection*{Vlivy na dané dílo:}

\subsection*{Další autorova tvorba:}
\noindent 
Mezi nejznámnější díla patří \textsc{Jak je důležité míti Filipa}, právě \textsc{Obraz Doriana Graye}, \textsc{Strašidlo cantervillské}, či \textsc{Lidská duše za socialismu}.

% \subsection*{Jak dílo inspirovalo další vývoj literatury:}

\section*{LITERÁRNÍ KRITIKA}

\subsection*{Dobové vnímání díla a jeho proměny:}
\noindent
Když Oscar Wilde původně vydal Obraz Doriana Graye do novin, tak to jeho nakladateli přišlo až moc pohoršující a pozměnil či smazal asi 500 slov.
Tento muž, Stoddart, mazal jak homosexuální, tak i heterosexuální narážky.
Gray například o svých ženách dřív mluvil jako o milenkách, ale i jeho vztah s Basilem Harvardem byl hlubší.

\noindent
Tato cenzura však nebyla konečná.
Dílo se stejně setkalo s kritikou a sám Oscar Wilde, když se dílo rozhodl vydat jako knihu, tak provedl autocenzuru.
Pozměnil některé texty, přidal další tři kapitoly. Poslední kapitolu rozdělil na dvě. Přidal i předmluvu, která je jeho odpovědí všem kritikům původní, časopisecké, verze.

\noindent
Stejně byl odsouzen ke dvěma letům nucených prací a právě toto dílo bylo použito jako jeden z důkazů, že má homosexuální sklony.

\subsection*{Aktuálnost tématu a zpracování tématu:}
\noindent
Bohémský život se vyskytuje stále a i když už se homosexualita nevnímá jako problém, tak přehnané požitkářství ano.
Oscar Wilde představuje osobu Doriana Graye bravurně a je až hrůzostrašné sledovat, jak jednoduše se dají zničit životy lidí kolem sebe, když je člověk jen trochu neohleduplný.
Vlastně se člověk může do Doriana Graye poměrně snadno vcítit a to to horší dopad na čtenáře má, když najednou vidí dopady jeho chování.

\section*{OSOBNÍ NÁZOR}
\noindent 
Wildeovo dílo mě chytilo a až do konce mě nepustilo.
Nabídlo mi nový pohled na svět kolem sebe, i lekci, kterou by si měli poslechnout všichni.
Naprosto rozumím, proč je Obraz Doriana Graye jednou z nejdůležitějších literatur své doby.

\vfill

%\noindent\begin{minipage}{\textwidth}
%    \textcolor{darkgray}{\rule{\linewidth}{0.4pt}
%    \footnotesize
%    \textbf{ --} 
%    }
%\end{minipage}

\newpage

% \coffeestainA{0.4}{1}{90}{30em}{40em}

\changefontsize{8pt}

\part*{NIKOLAJ VASILJEVIČ GOGOL -- Revizor {\hfill \normalfont\tiny\textsc{Libor Halík}}}

\noindent\begin{wrapfigure}{r}{0.35\textwidth}
\tiny

\subsection*{Ukázka z díla:}
\setlength{\parindent}{3pt}
xXx
\end{wrapfigure}

\section*{LITERÁRNÍ TEORIE}

\subsection*{Literární druh a žánr:}
\noindent xXx

% \subsection*{Literární směr:}

% \subsection*{Jazyk a slovní zásoba:}

\subsection*{Figury:}
\noindent 
\enquote{xXx}

\subsection*{Tropy:}
\noindent 
xXx

\subsection*{Stylistická charakteristika textu:}
\noindent 
xXx

\subsection*{Postavy:}
\noindent 
\textsc{xXx --} xXx \\

\subsection*{Děj:}
\noindent 
xXx

\subsection*{Kompozice, prostor a čas:}
\noindent 
xXx

\subsection*{Význam sdělení:}
\noindent 
xXx

\section*{LITERÁRNÍ HISTORIE}

% \subsection*{Politická situace:}

% \subsection*{Základní principy fungování společnosti:}

% \subsection*{Kontext dalších druhů umění:}

% \subsection*{Kontext literárního vývoje:}

\subsection*{Autor {\ssmall -- život autora:}}
\noindent 
xXx

% \subsection*{Vlivy na dané dílo:}

\subsection*{Další autorova tvorba:}
\noindent 
Mezi nejznámnější díla patří \textsc{xXx}

% \subsection*{Jak dílo inspirovalo další vývoj literatury:}

% \section*{LITERÁRNÍ KRITIKA}

% \subsection*{Dobové vnímání díla a jeho proměny:}

% \subsection*{Aktuálnost tématu a zpracování tématu:}


\section*{OSOBNÍ NÁZOR}
\noindent 
xXx

\vfill

\noindent\begin{minipage}{\textwidth}
    \textcolor{darkgray}{\rule{\linewidth}{0.4pt}
    \footnotesize
    \textbf{Placeholder --} xXx
    }
\end{minipage}

% --------------------------------------------------
% Č Á S T   T Ř E T Í
% --------------------------------------------------

\newpage

% \coffeestainA{0.4}{1}{90}{30em}{40em}

\changefontsize{8pt}

\part*{DAN BROWN -- Inferno {\hfill \normalfont\tiny\textsc{Libor Halík}}}

\noindent\begin{wrapfigure}{r}{0.35\textwidth}
\tiny

\subsection*{Ukázka z díla:}
\setlength{\parindent}{3pt}
xXx
\end{wrapfigure}

\section*{LITERÁRNÍ TEORIE}

\subsection*{Literární druh a žánr:}
\noindent xXx

% \subsection*{Literární směr:}

% \subsection*{Jazyk a slovní zásoba:}

\subsection*{Figury:}
\noindent 
\enquote{xXx}

\subsection*{Tropy:}
\noindent 
xXx

\subsection*{Stylistická charakteristika textu:}
\noindent 
xXx

\subsection*{Postavy:}
\noindent 
\textsc{xXx --} xXx \\

\subsection*{Děj:}
\noindent 
xXx

\subsection*{Kompozice, prostor a čas:}
\noindent 
xXx

\subsection*{Význam sdělení:}
\noindent 
xXx

\section*{LITERÁRNÍ HISTORIE}

% \subsection*{Politická situace:}

% \subsection*{Základní principy fungování společnosti:}

% \subsection*{Kontext dalších druhů umění:}

% \subsection*{Kontext literárního vývoje:}

\subsection*{Autor {\ssmall -- život autora:}}
\noindent 
xXx

% \subsection*{Vlivy na dané dílo:}

\subsection*{Další autorova tvorba:}
\noindent 
Mezi nejznámnější díla patří \textsc{xXx}

% \subsection*{Jak dílo inspirovalo další vývoj literatury:}

% \section*{LITERÁRNÍ KRITIKA}

% \subsection*{Dobové vnímání díla a jeho proměny:}

% \subsection*{Aktuálnost tématu a zpracování tématu:}


\section*{OSOBNÍ NÁZOR}
\noindent 
xXx

\vfill

\noindent\begin{minipage}{\textwidth}
    \textcolor{darkgray}{\rule{\linewidth}{0.4pt}
    \footnotesize
    \textbf{Placeholder --} xXx
    }
\end{minipage}

\newpage

% \coffeestainA{0.4}{1}{90}{30em}{40em}

\changefontsize{8pt}

\part*{DAN BROWN -- Šifra Mistra Leonarda {\hfill \normalfont\tiny\textsc{Libor Halík}}}

\noindent\begin{wrapfigure}{r}{0.35\textwidth}
\tiny

\subsection*{Ukázka z díla:}
\setlength{\parindent}{3pt}
xXx
\end{wrapfigure}

\section*{LITERÁRNÍ TEORIE}

\subsection*{Literární druh a žánr:}
\noindent xXx

% \subsection*{Literární směr:}

% \subsection*{Jazyk a slovní zásoba:}

\subsection*{Figury:}
\noindent 
\enquote{xXx}

\subsection*{Tropy:}
\noindent 
xXx

\subsection*{Stylistická charakteristika textu:}
\noindent 
xXx

\subsection*{Postavy:}
\noindent 
\textsc{xXx --} xXx \\

\subsection*{Děj:}
\noindent 
xXx

\subsection*{Kompozice, prostor a čas:}
\noindent 
xXx

\subsection*{Význam sdělení:}
\noindent 
xXx

\section*{LITERÁRNÍ HISTORIE}

% \subsection*{Politická situace:}

% \subsection*{Základní principy fungování společnosti:}

% \subsection*{Kontext dalších druhů umění:}

% \subsection*{Kontext literárního vývoje:}

\subsection*{Autor {\ssmall -- život autora:}}
\noindent 
xXx

% \subsection*{Vlivy na dané dílo:}

\subsection*{Další autorova tvorba:}
\noindent 
Mezi nejznámnější díla patří \textsc{xXx}

% \subsection*{Jak dílo inspirovalo další vývoj literatury:}

% \section*{LITERÁRNÍ KRITIKA}

% \subsection*{Dobové vnímání díla a jeho proměny:}

% \subsection*{Aktuálnost tématu a zpracování tématu:}


\section*{OSOBNÍ NÁZOR}
\noindent 
xXx

\vfill

\noindent\begin{minipage}{\textwidth}
    \textcolor{darkgray}{\rule{\linewidth}{0.4pt}
    \footnotesize
    \textbf{Placeholder --} xXx
    }
\end{minipage}

\newpage

% \coffeestainA{0.4}{1}{90}{30em}{40em}

\changefontsize{8pt}

\part*{FRANCIS SCOTT FITZGERALD -- Velký Gatsby {\hfill \normalfont\tiny\textsc{Libor Halík}}}

\noindent\begin{wrapfigure}{r}{0.35\textwidth}
\tiny

\subsection*{Ukázka z díla:}
\setlength{\parindent}{3pt}
xXx
\end{wrapfigure}

\section*{LITERÁRNÍ TEORIE}

\subsection*{Literární druh a žánr:}
\noindent xXx

% \subsection*{Literární směr:}

% \subsection*{Jazyk a slovní zásoba:}

\subsection*{Figury:}
\noindent 
\enquote{xXx}

\subsection*{Tropy:}
\noindent 
xXx

\subsection*{Stylistická charakteristika textu:}
\noindent 
xXx

\subsection*{Postavy:}
\noindent 
\textsc{xXx --} xXx \\

\subsection*{Děj:}
\noindent 
xXx

\subsection*{Kompozice, prostor a čas:}
\noindent 
xXx

\subsection*{Význam sdělení:}
\noindent 
xXx

\section*{LITERÁRNÍ HISTORIE}

% \subsection*{Politická situace:}

% \subsection*{Základní principy fungování společnosti:}

% \subsection*{Kontext dalších druhů umění:}

% \subsection*{Kontext literárního vývoje:}

\subsection*{Autor {\ssmall -- život autora:}}
\noindent 
xXx

% \subsection*{Vlivy na dané dílo:}

\subsection*{Další autorova tvorba:}
\noindent 
Mezi nejznámnější díla patří \textsc{xXx}

% \subsection*{Jak dílo inspirovalo další vývoj literatury:}

% \section*{LITERÁRNÍ KRITIKA}

% \subsection*{Dobové vnímání díla a jeho proměny:}

% \subsection*{Aktuálnost tématu a zpracování tématu:}


\section*{OSOBNÍ NÁZOR}
\noindent 
xXx

\vfill

\noindent\begin{minipage}{\textwidth}
    \textcolor{darkgray}{\rule{\linewidth}{0.4pt}
    \footnotesize
    \textbf{Placeholder --} xXx
    }
\end{minipage}

\newpage

% \coffeestainA{0.4}{1}{90}{30em}{40em}

\changefontsize{7pt}

\part*{GEORGE ORWELL -- Devatenáct set osmdesát čtyři {\hfill \normalfont\tiny\textsc{Libor Halík}}}

\noindent\begin{wrapfigure}{r}{0.35\textwidth}
\tiny

\subsection*{Ukázka z díla:}
\setlength{\parindent}{3pt}\setlength{\parskip}{0.5em}
Winston viděl O’Briena možná tucetkrát během právě tolika let. Silně ho to k němu táhlo, nejen proto, že ho zaujal rozpor mezi O’Brienovým uhlazeným chováním a jeho zápasnickou konstitucí. Mnohem spíš to způsobovala skrytá víra – nebo možná ani ne víra, pouze naděje –, že O’Brienova politická uvědomělost není bez vady. Cosi v jeho tváři to neodbytně naznačovalo. A třeba se mu ve tváři neodrážela ani neuvědomělost, nýbrž jen inteligence. Ale tak či onak působil dojmem člověka, se kterým se dá mluvit, pokud se vám nějak podaří obelstít telestěnu a zastihnout ho o samotě. Winston si tuhle domněnku nikdy ani v nejmenším nezkoušel ověřit: nebylo totiž ani možno jak. V tu chvíli O’Brien pohlédl na hodinky, zjistil, že se blíží jedenáct nula nula, a zjevně se rozhodl zůstat v Archivním oddělení, dokud Dvouminutovka nenávisti neskončí. Usadil se na židli v téže řadě jako Winston, o pár míst dál. Mezi nimi seděla drobná, písková blondýna, která pracovala v kóji vedle Winstona. Tmavovláska se uvelebila přímo za ním. Vzápětí se z velké telestěny na konci místnosti vydral příšerný, řezavý jek, jako by se nějaký obludný stroj točil bez mazání.
\end{wrapfigure}

\section*{LITERÁRNÍ TEORIE}

\subsection*{Literární druh a žánr:}
\noindent 
Epický román, sci-fi a utopickými prvky, které ale kritizuje.

% \subsection*{Literární směr:}

% \subsection*{Jazyk a slovní zásoba:}

% \subsection*{Figury:}

% \subsection*{Tropy:}

\subsection*{Stylistická charakteristika textu:}
\noindent
Text je pevně rozdělen na tři části. Děj je vyprávěn er-formou se zaměřením na Winstona. Autor je dobrým esejistou, ne tak dobrým romanopiscem.

\subsection*{Postavy:}
\noindent
\textsc{Winston Smith} -- Hlavní postava, takzvaný every-man archetyp. Člověk bez nijak výrazných rysů, do kterého by se měl každý člověk být schopen vcítit. Prostě přežívá.\\
\textsc{Julia} -- Romantický interest pro Winstona. Prosazuje životní styl tichého odporu. Kdy umí jít proti systému v malých věcech, ale nevidí smysl v organizovaném odporu. \\
\textsc{O'Brian} -- Winstonův protipól. Winston z něj cítí, že ten jediný mu porozumí. A on mu skutečně rozumí. Ale opačně. A chce ho dostat na svojí stranu.

\subsection*{Děj:}
{\setlength{\parindent}{5pt}
Děj sleduje Winstonův běžný život a seznamuje nás s životem v Oceánii. První výraznou změnou je, když se Winston dá dohromady s Julií, do které se dle vlastních slov zamiloval. Druhou výraznou změnou je, když Winstona s Julií dostihne Psychopol a následně jsou mučeni na Ministerstvu Lásky. Děj končí kompletním obratem Winstonovi osobnosti. Kdy se kompletně podvolí Straně.
}
\subsection*{Kompozice, prostor a čas:}
\noindent 
Většina příběhu se odehrává v Londýně, ve státu zvaném Oceánie. Mohl by být rok 1984, i když to není jisté. Během příběhu je pár vzpomínkových sekvencí a informativní pasáž v podobě Knihy, kterou Winston čte.

\subsection*{Význam sdělení:}
\noindent
Autor varuje před technologií a čeho by byla schopna ve špatných rukou. Varuje před totalitním státem a jaké kontroly by byl schopen s pomocí moderních technologií. 

\section*{LITERÁRNÍ HISTORIE}

\subsection*{Politická situace:}
Autor napsal svůj román po dvou světových válkách, v období velkého technologického rozmachu. Svět byl nestálý, v chaosu a do toho zažili největší rozkvět techniky za posledních několik staletí. Pro románovou totalitní nadvládu ho inspiroval socialismus v SSSR.

% \subsection*{Základní principy fungování společnosti:}

% \subsection*{Kontext dalších druhů umění:}

% \subsection*{Kontext literárního vývoje:}

\subsection*{Eric Arthur Blair {\ssmall -- život autora:}}
\noindent 
Dle některých zdrojů nejvlivnější esejista své doby. Narodil se v chudém prostředí. Odtud to zaměření na nejspodnější vrstvu. Svůj pseudonym si vymyslel dle názvu řeky v místě, kde bydlel. Svojí tvorbou se snažil upozornit na stav, ve kterém se svět nachází a mohl by se nacházet, pokud bude pokračovat směrem, kam míří.

% \subsection*{Vlivy na dané dílo:}

\subsection*{Další autorova tvorba:}
\noindent 
\begin{itemize}
    \item Farma zvířat
    \item Cesta k Wigan Pier
    \item Hold Katalánsku
\end{itemize}

% \subsection*{Jak dílo inspirovalo další vývoj literatury:}

% \section*{LITERÁRNÍ KRITIKA}

% \subsection*{Dobové vnímání díla a jeho proměny:}

\subsection*{Aktuálnost tématu a zpracování tématu:}
\noindent
Témata jako cenzura, války za mocí, technologická nadvláda, sledování, ztráta soukromí a další, jsou stále aktuální, ne-li ještě více, než v minulosti.


\section*{OSOBNÍ NÁZOR}
\noindent 
Román mistra Orwella jsem měl problém ocenit. První dvě části byli natolik zmatečné a ploché, že můj názor byl spíše negativní. Ale třetí část byla schopná zodpovědět každý můj problém (či na mě zapůsobila O’Brienova převýchova a já si vyvinul dipsych) a během posledního sta stránek se můj postoj otočil o~180~stupňů.



\vfill

% \noindent\begin{minipage}{\textwidth}
%    \textcolor{darkgray}{\rule{\linewidth}{0.4pt}
%    \changefontsize{7pt}
%    \footnotesize
%    Fuck my life}
% \end{minipage}

\newpage

% \coffeestainA{0.4}{1}{90}{30em}{40em}

\changefontsize{8pt}

\part*{JOANNE ROWLINGOVÁ -- Harry Potter {\hfill \normalfont\tiny\textsc{Libor Halík}}}

\noindent\begin{wrapfigure}{r}{0.35\textwidth}
\tiny

\subsection*{Ukázka z díla:}
\setlength{\parindent}{3pt}
xXx
\end{wrapfigure}

\section*{LITERÁRNÍ TEORIE}

\subsection*{Literární druh a žánr:}
\noindent xXx

% \subsection*{Literární směr:}

% \subsection*{Jazyk a slovní zásoba:}

\subsection*{Figury:}
\noindent 
\enquote{xXx}

\subsection*{Tropy:}
\noindent 
xXx

\subsection*{Stylistická charakteristika textu:}
\noindent 
xXx

\subsection*{Postavy:}
\noindent 
\textsc{Harry Potter --} xXx \\
\textsc{Ron Weasley --} xXx \\
\textsc{Hermiona Grangerová --} xXx \\
\textsc{Albus Brumbál --} xXx \\
\textsc{Quirell --} xXx \\

\subsection*{Děj:}
\noindent 
xXx

\subsection*{Kompozice, prostor a čas:}
\noindent 
xXx

\subsection*{Význam sdělení:}
\noindent 
xXx

\section*{LITERÁRNÍ HISTORIE}

% \subsection*{Politická situace:}

% \subsection*{Základní principy fungování společnosti:}

% \subsection*{Kontext dalších druhů umění:}

% \subsection*{Kontext literárního vývoje:}

\subsection*{Autor {\ssmall -- život autora:}}
\noindent 
xXx

% \subsection*{Vlivy na dané dílo:}

\subsection*{Další autorova tvorba:}
\noindent 
Mezi nejznámnější díla patří \textsc{xXx}

% \subsection*{Jak dílo inspirovalo další vývoj literatury:}

% \section*{LITERÁRNÍ KRITIKA}

% \subsection*{Dobové vnímání díla a jeho proměny:}

% \subsection*{Aktuálnost tématu a zpracování tématu:}


\section*{OSOBNÍ NÁZOR}
\noindent 
xXx

\vfill

\noindent\begin{minipage}{\textwidth}
    \textcolor{darkgray}{\rule{\linewidth}{0.4pt}
    \footnotesize
    \textbf{Placeholder --} xXx
    }
\end{minipage}

\newpage

% \coffeestainA{0.4}{1}{90}{30em}{40em}

\changefontsize{8pt}

\part*{J. R. R. TOLKIEN -- Hobbit {\hfill \normalfont\tiny\textsc{Libor Halík}}}

\noindent\begin{wrapfigure}{r}{0.35\textwidth}
\tiny

\subsection*{Ukázka z díla:}
\setlength{\parindent}{3pt}
xXx
\end{wrapfigure}

\section*{LITERÁRNÍ TEORIE}

\subsection*{Literární druh a žánr:}
\noindent xXx

% \subsection*{Literární směr:}

% \subsection*{Jazyk a slovní zásoba:}

\subsection*{Figury:}
\noindent 
\enquote{xXx}

\subsection*{Tropy:}
\noindent 
xXx

\subsection*{Stylistická charakteristika textu:}
\noindent 
xXx

\subsection*{Postavy:}
\noindent 
\textsc{xXx --} xXx \\

\subsection*{Děj:}
\noindent 
xXx

\subsection*{Kompozice, prostor a čas:}
\noindent 
xXx

\subsection*{Význam sdělení:}
\noindent 
xXx

\section*{LITERÁRNÍ HISTORIE}

% \subsection*{Politická situace:}

% \subsection*{Základní principy fungování společnosti:}

% \subsection*{Kontext dalších druhů umění:}

% \subsection*{Kontext literárního vývoje:}

\subsection*{Autor {\ssmall -- život autora:}}
\noindent 
xXx

% \subsection*{Vlivy na dané dílo:}

\subsection*{Další autorova tvorba:}
\noindent 
Mezi nejznámnější díla patří \textsc{xXx}

% \subsection*{Jak dílo inspirovalo další vývoj literatury:}

% \section*{LITERÁRNÍ KRITIKA}

% \subsection*{Dobové vnímání díla a jeho proměny:}

% \subsection*{Aktuálnost tématu a zpracování tématu:}


\section*{OSOBNÍ NÁZOR}
\noindent 
xXx

\vfill

\noindent\begin{minipage}{\textwidth}
    \textcolor{darkgray}{\rule{\linewidth}{0.4pt}
    \footnotesize
    \textbf{Placeholder --} xXx
    }
\end{minipage}

% --------------------------------------------------
% Č Á S T   Č T V R T Á
% --------------------------------------------------

\newpage

% \coffeestainA{0.4}{1}{90}{30em}{40em}

\changefontsize{8pt}

\part*{FRANTIŠEK GELLNER -- Po nás ať přijde potopa {\hfill \normalfont\tiny\textsc{Libor Halík}}}

\noindent\begin{wrapfigure}{r}{0.35\textwidth}
\tiny

\subsection*{Ukázka z díla:}
\setlength{\parindent}{3pt}
xXx
\end{wrapfigure}

\section*{LITERÁRNÍ TEORIE}

\subsection*{Literární druh a žánr:}
\noindent xXx

% \subsection*{Literární směr:}

% \subsection*{Jazyk a slovní zásoba:}

\subsection*{Figury:}
\noindent 
\enquote{xXx}

\subsection*{Tropy:}
\noindent 
xXx

\subsection*{Stylistická charakteristika textu:}
\noindent 
xXx

\subsection*{Postavy:}
\noindent 
\textsc{xXx --} xXx \\

\subsection*{Děj:}
\noindent 
xXx

\subsection*{Kompozice, prostor a čas:}
\noindent 
xXx

\subsection*{Význam sdělení:}
\noindent 
xXx

\section*{LITERÁRNÍ HISTORIE}

% \subsection*{Politická situace:}

% \subsection*{Základní principy fungování společnosti:}

% \subsection*{Kontext dalších druhů umění:}

% \subsection*{Kontext literárního vývoje:}

\subsection*{Autor {\ssmall -- život autora:}}
\noindent 
xXx

% \subsection*{Vlivy na dané dílo:}

\subsection*{Další autorova tvorba:}
\noindent 
Mezi nejznámnější díla patří \textsc{xXx}

% \subsection*{Jak dílo inspirovalo další vývoj literatury:}

% \section*{LITERÁRNÍ KRITIKA}

% \subsection*{Dobové vnímání díla a jeho proměny:}

% \subsection*{Aktuálnost tématu a zpracování tématu:}


\section*{OSOBNÍ NÁZOR}
\noindent 
xXx

\vfill

\noindent\begin{minipage}{\textwidth}
    \textcolor{darkgray}{\rule{\linewidth}{0.4pt}
    \footnotesize
    \textbf{Placeholder --} xXx
    }
\end{minipage}

\newpage

% \coffeestainA{0.4}{1}{90}{30em}{40em}

\changefontsize{8pt}

\part*{FRANZ KAFKA -- Proměna {\hfill \normalfont\tiny\textsc{Libor Halík}}}

\noindent\begin{wrapfigure}{r}{0.35\textwidth}
\tiny

\subsection*{Ukázka z díla:}
\setlength{\parindent}{3pt}
xXx
\end{wrapfigure}

\section*{LITERÁRNÍ TEORIE}

\subsection*{Literární druh a žánr:}
\noindent xXx

% \subsection*{Literární směr:}

% \subsection*{Jazyk a slovní zásoba:}

\subsection*{Figury:}
\noindent 
\enquote{xXx}

\subsection*{Tropy:}
\noindent 
xXx

\subsection*{Stylistická charakteristika textu:}
\noindent 
xXx

\subsection*{Postavy:}
\noindent 
\textsc{xXx --} xXx \\

\subsection*{Děj:}
\noindent 
xXx

\subsection*{Kompozice, prostor a čas:}
\noindent 
xXx

\subsection*{Význam sdělení:}
\noindent 
xXx

\section*{LITERÁRNÍ HISTORIE}

% \subsection*{Politická situace:}

% \subsection*{Základní principy fungování společnosti:}

% \subsection*{Kontext dalších druhů umění:}

% \subsection*{Kontext literárního vývoje:}

\subsection*{Autor {\ssmall -- život autora:}}
\noindent 
xXx

% \subsection*{Vlivy na dané dílo:}

\subsection*{Další autorova tvorba:}
\noindent 
Mezi nejznámnější díla patří \textsc{xXx}

% \subsection*{Jak dílo inspirovalo další vývoj literatury:}

% \section*{LITERÁRNÍ KRITIKA}

% \subsection*{Dobové vnímání díla a jeho proměny:}

% \subsection*{Aktuálnost tématu a zpracování tématu:}


\section*{OSOBNÍ NÁZOR}
\noindent 
xXx

\vfill

\noindent\begin{minipage}{\textwidth}
    \textcolor{darkgray}{\rule{\linewidth}{0.4pt}
    \footnotesize
    \textbf{Placeholder --} xXx
    }
\end{minipage}

\newpage

% \coffeestainA{0.4}{1}{90}{30em}{40em}

\changefontsize{8pt}

\part*{VIKTOR DYK -- Krysař {\hfill \normalfont\tiny\textsc{Libor Halík}}}

\noindent\begin{wrapfigure}{r}{0.35\textwidth}
\tiny

\subsection*{Ukázka z díla:}
\setlength{\parindent}{3pt}
xXx
\end{wrapfigure}

\section*{LITERÁRNÍ TEORIE}

\subsection*{Literární druh a žánr:}
\noindent xXx

% \subsection*{Literární směr:}

% \subsection*{Jazyk a slovní zásoba:}

\subsection*{Figury:}
\noindent 
\enquote{xXx}

\subsection*{Tropy:}
\noindent 
xXx

\subsection*{Stylistická charakteristika textu:}
\noindent 
xXx

\subsection*{Postavy:}
\noindent 
\textsc{xXx --} xXx \\

\subsection*{Děj:}
\noindent 
xXx

\subsection*{Kompozice, prostor a čas:}
\noindent 
xXx

\subsection*{Význam sdělení:}
\noindent 
xXx

\section*{LITERÁRNÍ HISTORIE}

% \subsection*{Politická situace:}

% \subsection*{Základní principy fungování společnosti:}

% \subsection*{Kontext dalších druhů umění:}

% \subsection*{Kontext literárního vývoje:}

\subsection*{Autor {\ssmall -- život autora:}}
\noindent 
xXx

% \subsection*{Vlivy na dané dílo:}

\subsection*{Další autorova tvorba:}
\noindent 
Mezi nejznámnější díla patří \textsc{xXx}

% \subsection*{Jak dílo inspirovalo další vývoj literatury:}

% \section*{LITERÁRNÍ KRITIKA}

% \subsection*{Dobové vnímání díla a jeho proměny:}

% \subsection*{Aktuálnost tématu a zpracování tématu:}


\section*{OSOBNÍ NÁZOR}
\noindent 
xXx

\vfill

\noindent\begin{minipage}{\textwidth}
    \textcolor{darkgray}{\rule{\linewidth}{0.4pt}
    \footnotesize
    \textbf{Placeholder --} xXx
    }
\end{minipage}

\newpage

% \coffeestainA{0.4}{1}{90}{30em}{40em}

\changefontsize{8pt}

\part*{KAREL ČAPEK -- Bílá Nemoc {\hfill \normalfont\tiny\textsc{Libor Halík}}}

\noindent\begin{wrapfigure}{r}{0.35\textwidth}
\tiny

\subsection*{Ukázka z díla:}
\setlength{\parindent}{3pt}
xXx
\end{wrapfigure}

\section*{LITERÁRNÍ TEORIE}

\subsection*{Literární druh a žánr:}
\noindent xXx

% \subsection*{Literární směr:}

% \subsection*{Jazyk a slovní zásoba:}

\subsection*{Figury:}
\noindent 
\enquote{xXx}

\subsection*{Tropy:}
\noindent 
xXx

\subsection*{Stylistická charakteristika textu:}
\noindent 
xXx

\subsection*{Postavy:}
\noindent 
\textsc{xXx --} xXx \\

\subsection*{Děj:}
\noindent 
xXx

\subsection*{Kompozice, prostor a čas:}
\noindent 
xXx

\subsection*{Význam sdělení:}
\noindent 
xXx

\section*{LITERÁRNÍ HISTORIE}

% \subsection*{Politická situace:}

% \subsection*{Základní principy fungování společnosti:}

% \subsection*{Kontext dalších druhů umění:}

% \subsection*{Kontext literárního vývoje:}

\subsection*{Autor {\ssmall -- život autora:}}
\noindent 
xXx

% \subsection*{Vlivy na dané dílo:}

\subsection*{Další autorova tvorba:}
\noindent 
Mezi nejznámnější díla patří \textsc{xXx}

% \subsection*{Jak dílo inspirovalo další vývoj literatury:}

% \section*{LITERÁRNÍ KRITIKA}

% \subsection*{Dobové vnímání díla a jeho proměny:}

% \subsection*{Aktuálnost tématu a zpracování tématu:}


\section*{OSOBNÍ NÁZOR}
\noindent 
xXx

\vfill

\noindent\begin{minipage}{\textwidth}
    \textcolor{darkgray}{\rule{\linewidth}{0.4pt}
    \footnotesize
    \textbf{Placeholder --} xXx
    }
\end{minipage}

\newpage

% \coffeestainA{0.4}{1}{90}{30em}{40em}

\changefontsize{8pt}

\part*{KAREL ČAPEK -- R. U. R. {\hfill \normalfont\tiny\textsc{Libor Halík}}}

\noindent\begin{wrapfigure}{r}{0.35\textwidth}
\tiny

\subsection*{Ukázka z díla:}
\setlength{\parindent}{3pt}
xXx
\end{wrapfigure}

\section*{LITERÁRNÍ TEORIE}

\subsection*{Literární druh a žánr:}
\noindent xXx

% \subsection*{Literární směr:}

% \subsection*{Jazyk a slovní zásoba:}

\subsection*{Figury:}
\noindent 
\enquote{xXx}

\subsection*{Tropy:}
\noindent 
xXx

\subsection*{Stylistická charakteristika textu:}
\noindent 
xXx

\subsection*{Postavy:}
\noindent 
\textsc{xXx --} xXx \\

\subsection*{Děj:}
\noindent 
xXx

\subsection*{Kompozice, prostor a čas:}
\noindent 
xXx

\subsection*{Význam sdělení:}
\noindent 
xXx

\section*{LITERÁRNÍ HISTORIE}

% \subsection*{Politická situace:}

% \subsection*{Základní principy fungování společnosti:}

% \subsection*{Kontext dalších druhů umění:}

% \subsection*{Kontext literárního vývoje:}

\subsection*{Autor {\ssmall -- život autora:}}
\noindent 
xXx

% \subsection*{Vlivy na dané dílo:}

\subsection*{Další autorova tvorba:}
\noindent 
Mezi nejznámnější díla patří \textsc{xXx}

% \subsection*{Jak dílo inspirovalo další vývoj literatury:}

% \section*{LITERÁRNÍ KRITIKA}

% \subsection*{Dobové vnímání díla a jeho proměny:}

% \subsection*{Aktuálnost tématu a zpracování tématu:}


\section*{OSOBNÍ NÁZOR}
\noindent 
xXx

\vfill

\noindent\begin{minipage}{\textwidth}
    \textcolor{darkgray}{\rule{\linewidth}{0.4pt}
    \footnotesize
    \textbf{Placeholder --} xXx
    }
\end{minipage}

\end{document}
