\documentclass{article}
\usepackage[czech]{babel}
\usepackage{geometry}
\geometry{a4paper, margin=1.5cm}
\usepackage{xcolor}
\usepackage{pgfornament}
\usepackage{nopageno}
\usepackage[T1]{fontenc}
\usepackage{calligra}
\NewDocumentCommand{\setfont}{m m}{%
{\fontfamily{#1}\selectfont #2}%
}

\def\wloss{%
\begingroup
\def\i{\pgfusepath{clip}}%
\def\k{\pgfusepath{stroke}}%
\let\o\pgfpathclose
\let\s\pgfusepathqfillstroke
\def\p ##1##2{\pgfqpoint{##1bp}{##2bp}}%
\def\m ##1 ##2 {\pgfpathmoveto{\p{##1}{##2}}}%
\def\r ##1 ##2 ##3 ##4 {\pgfpathrectangle{\p{##1}{##2}}{%
\p{##3}{##4}}}%
\def\l ##1 ##2 {\pgfpathlineto{\p{##1}{##2}}}%
\def\c ##1 ##2 ##3 ##4 ##5 ##6 {%
\pgfpathcurveto{\p{##1}{##2}}{\p{##3}{##4}}{\p{##5}{##6}}}%
\begin{tikzpicture}
\pgftransformscale{.08}
\m 133.06666667 201.73333333
\l 132.66666667 138
\l 93.06666667 137.6
\l 53.33333333 137.33333333
\l 53.33333333 166.66666667
\l 53.33333333 196
\l 48.66666667 196
\l 44 196
\l 44 166.66666667
\l 44 137.33333333
\l 22 137.33333333
\l 0 137.33333333
\l 0 133.33333333
\l 0 129.33333333
\l 22 129.33333333
\l 44 129.33333333
\l 44 97.33333333
\l 44 65.33333333
\l 48.66666667 65.33333333
\l 53.33333333 65.33333333
\l 53.33333333 97.33333333
\l 53.33333333 129.33333333
\l 67.33333333 129.33333333
\l 81.33333333 129.33333333
\l 81.33333333 97.33333333
\l 81.33333333 65.33333333
\l 86 65.33333333
\l 90.66666667 65.33333333
\l 90.66666667 97.33333333
\l 90.66666667 129.33333333
\l 112 129.33333333
\l 133.33333333 129.33333333
\l 133.33333333 64.66666667
\l 133.33333333 -0
\l 137.33333333 -0
\l 141.33333333 -0
\l 141.33333333 64.66666667
\l 141.33333333 129.33333333
\l 166.66666667 129.33333333
\l 192 129.33333333
\l 192 107.33333333
\l 192 85.33333333
\l 177.86666667 85.33333333
\l 163.86666667 85.33333333
\l 164.26666667 80.93333333
\l 164.66666667 76.66666667
\l 215.06666667 76.26666667
\l 265.33333333 76
\l 265.33333333 80.66666667
\l 265.33333333 85.33333333
\l 232.66666667 85.33333333
\l 200 85.33333333
\l 200 107.33333333
\l 200 129.33333333
\l 232.66666667 129.33333333
\l 265.33333333 129.33333333
\l 265.33333333 133.33333333
\l 265.33333333 137.33333333
\l 250.66666667 137.33333333
\l 236 137.33333333
\l 236 152.66666667
\l 236 168
\l 231.33333333 168
\l 226.66666667 168
\l 226.66666667 152.66666667
\l 226.66666667 137.33333333
\l 213.33333333 137.33333333
\l 200 137.33333333
\l 200 166.66666667
\l 200 196
\l 196 196
\l 192 196
\l 191.73333333 167.06666667
\l 191.33333333 138
\l 166.66666667 138
\l 142 138
\l 141.6 201.73333333
\l 141.33333333 265.33333333
\l 137.33333333 265.33333333
\l 133.33333333 265.33333333
\pgfsetfillcolor{white}
\pgfusepath{fill}
\k
\end{tikzpicture}
\endgroup
}

\usepackage[
    hidelinks,                    % No visible link colors or borders
    pdftitle={Motivační dopis},       % PDF metadata: title
    pdfauthor={Libor Halík},         % PDF metadata: author
    pdfstartview=FitH,            % Fit page width to window
    pdfduplex=DuplexFlipShortEdge % Double-sided printing, flip on short edge
]{hyperref}

\usepackage{fontspec}
\setmainfont{Tex Gyre Termes}
\newfontfamily\signaturefont{MrDafoe}

\begin{document}\fontsize{10pt}{10pt}\selectfont\setlength\parindent{0em}\setlength{\parskip}{1em}
% ADRESA MOJE
Libor Halík\\
Nepovim 42\\
541 02{ }{ }Trutnov\\
+365 73 73 73 230\\
halik@ssoh.cz\\
\vspace{2em}

%ADRESA PRÁCE
PŘEDVÝBĚR.CZ a. s.\\
František Boudný\\
Na Kozačce 1289/7\\
120 00{ }{ }Praha 2 -- Vinohrady\\
\vspace{1em}

%KDE
\begin{flushright}
Janské Lázně 22. října 2025
\end{flushright}
\vspace{1em}

%NADPIS
\subsubsection*{Zájem o místo Specialisty technické dokumentace}
Vážený pane,

s velkým zájmem jsem narazil na Vaši nabídku pozice Specialista technické dokumentace – Zdravotnické prostředky na portálu Prace.cz. Jsem 20letý absolvent střední ekonomické školy s maturitou a právě teď hledám první skutečnou příležitost, kde bych mohl uplatnit svou pečlivost, technické myšlení a chuť učit se novým věcem. I když zatím nemám žádné pracovní zkušenosti, celý život mě baví rozkládat věci na součástky, chápat, jak fungují, a pak to umět jasně popsat – ať už v návodech, které jsem si psal sám pro sebe, nebo ve školních projektech.

Pracovat na dokumentaci pro zdravotnické prostředky mě láká proto, že vím, jak důležité je, aby všechno sedělo na milimetr a bylo srozumitelné pro lidi, kteří s tím zařízením zachraňují životy. Jsem připravený rychle se učit normy, postupy i software, který používáte, a vím, že moje čerstvá hlava bez zažitých stereotypů může být výhoda. Pokud mi dáte šanci, slibuju, že budu ten, kdo se ptá na detaily, kontroluje každý řádek a doručí práci včas.

Děkuji za zvážení mé přihlášky. Rád přijdu na pohovor a ukážu, že i bez zkušeností mám drive a odhodlání, které u vás najdou uplatnění. Těším se na příležitost se s Vámi setkat.

S pozdravem\par
\vspace{0.5em}
{\signaturefont\selectfont\color{cyan} Libor Halík}\par
\vfill
\begin{flushright}
\href{https://www.prace.cz/nabidka/2000803587}{\wloss}
\end{flushright}
\end{document}