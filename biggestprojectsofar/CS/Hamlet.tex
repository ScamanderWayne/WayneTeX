\documentclass[../main.tex]{subfiles}
\begin{document}\fontsize{10pt}{10pt}\selectfont\setlength\parindent{0em}
\literature{Hamlet}{WILLIAM SHAKESPEARE}

\noindent\begin{wrapfigure}{r}{0.35\textwidth}
\tiny\subwection{Ukázka z díla:}\setlength{\parindent}{3pt}\noindent\textsc{Hamlet:} Být nebo nebýt -- to je otázka:
je důstojnější zapřít se~a~snášet surovost osudu a~jeho rány, anebo se vzepřít moři trápení a~skoncovat to navždy?
Zemřít, spát -- a~je to.
Spát -- a~navždy ukončit úzkost a~věčné útrapy a~strázně, co údělem jsou těla -- co~si~můžeme přát víc, po čem toužit?
-- Zemřít, spát -- spát, možná snít -- a~právě v tom je zrada.
Až ztichne vřava~pozemského bytí, ve spánku smrti můžeme mít sny -- to proto váháme a~snášíme tu dlouhou bídu, již se říká život.
Neboť kdo vydržel by kopance a~výsměch doby, aroganci mocných, průtahy soudů, znesvěcenou lásku, nadutost úřadů a~ústrky, co slušnost věčně sklízí od lumpů, když pouhá dýka~srovnala~by účty, a~byl by klid?
Kdo chtěl by nést to břímě, úpět a~plahočit se životem, nemít strach z toho, co je za~smrtí, z neznámé krajiny, z níž poutníci se nevracejí.
To nám láme vůli -- snášíme radši hrůzy, které známe, než abychom šli vstříc těm neznámým.
Tak svědomí z nás dělá zbabělce a~zdravá barva~rozhodného činu se roznemůže zbledlou meditací, záměry velké významem a~vahou se odvracejí z vytčeného směru a neuzrají v čin.
\end{wrapfigure}

\wection{LITERÁRNÍ TEORIE}

\subwection{Literární druh a žánr:}
\noindent drama, tragédie

% \subwection{Literární směr:}

\subwection{Jazyk a slovní zásoba: }
\noindent Hra je psána spisovným až vytříbeným jazykem.
Postavy od dvora mluví vznešeně.
Autor dodává postavám charakter skrz užitý jazyk. \\
\textsc{Hamlet} je velmi mluvnicky zdatný.
V jeho řeči se vyskytuje ironie až sarkasmus.
Metaforami, slovními hříčkami a~dvojsmysly dodává na \texthl{důvěře své předstírané šílenosti}.
Užívá i mnohých básnických figur, kupříkladu anafory, inverzi slovosledu a asyndoty.
\textsc{Laertes}, \textsc{Ofélie} a hlavně \textsc{Claudius} za ním \texthl{nezaostávají a mluví vznešeně též}.
\textsc{Polonius} je skvělou ukázkou charakteru promítnutého skrz řeč.
Využívá \texthl{nadbytek} vznešených slov pro ukázku svého obsáhlého repertoáru a to \texthl{na úkor samotné myšlenky} svého sdělení.
Dokazuje tím svou snahu obhájit si své místo u dvoru.
Výjimku z tohoto pravidla tvoří prostý lid, které \textit{Shakespeare} nechává mluvit prostě až~hrubě.  

% \subwection{Figury:}

% \subwection{Tropy:}

\subwection{Stylistická charakteristika textu: }
\noindent\textit{Hamlet}, typicky pro \textit{Shakespearova} díla, je psán \textbfhl{blankversem}.
Pro rozvinutí postav a jejich motivů slouží monology.
Dialog posouvá děj.
Důraz je na myšlenku a charaktery, děj je druhořadý a podtrhuje charakter postav.
Autor od děje kličkuje a viditelně oddaluje posun.
Což~dává postavám čas na rozvinutí.

\subwection{Postavy:}
\noindent 
\textschl{Hamlet --} princ, syn královny Gertrudy a již zesnulého krále,
následník trůnu, není schopen se vypořádat se ztrátou svého otce o které se později dozvídá,
že~nebyla z~přirozených důvodů. \\
\textschl{Horacio --} věrný přítel Hamleta, filosof a rozvážný člověk,
s Hamletem diskutuje o jeho názorech, dotazuje se. \\
\textschl{Polonius --} předseda státní rady, věrný koruně. Až moc horlivý,
snaží se králi pomoci se šíleným Hamletem. Na svou horlivost a sebevědomost doplatí. \\
\textschl{Ofelie --} dcera Poloniova, zamilovaná do Hamleta, láskou zmítaná až utrápena je. \\
\textschl{Laertes --} syn Poloniův, bratr Ofelie, který si vyčítá ztrátu otce i sestry
-- kvůli intrikám nového krále nakonec zabíjí Hamleta. \\
\textschl{Claudius --} nový král, manžel Gertrudy, naplánoval smrt starého krále,
svého bratra, pro svoje obohacení a svými intrikami se pokusil zbavit i Hamleta. \\
\textschl{Gertruda --} dánská královna, snaží se uklidnit manžela a ochraňovat syna,
pro svou nerozhodnost zemře. \\
\textschl{Duch starého krále --} jménem také Hamlet,
zapřísahá syna aby vykonal pomstu ale aby nijak neublížil své matce.

\subwection{Děj:}
\noindent 
Příběh se odehrává v poměrně krátkém časovém úseku, z pohledu historie,
od smrti jednoho Hamleta do smrti Hamleta druhého.
Duch starého krále zjevuje, že~byl otráven, Hamlet začne předstírat šílenství
a během tohoto období usiluje o usvědčení nového krále, svého strýce.
Nakonec ho usvědčí za pomoci kočovných herců, kteří sehrají na~královském dvoře onu bratrovraždu
a král se sám usvědčí svojí reakcí na tuto scénu.
Hamlet se setká se svojí matkou a probodne muže, co je odposlouchával za~tapisérií v~domnění,
že~jde o jeho strýce, zrádného krále Claudia.
Místo toho \texthl{k zemi padá Polonius}, otec jeho milované Ofelie a jejího bratra Laerta,
který byl tou dobou na~cestách.
To~dává králi příležitost se Hamleta zbavit a vykázat pryč, do Anglie.
Tam už Hamletovi domluvil popravu, aniž by o tom věděla jeho matka.
Hamlet ale lstí podstrčí na~popravu muže, kteří naopak měli dohlédnout na jeho popravu.
Po čase se vrací zpátky, zrovna na~pohřeb Ofelie.
Co sama, bez otce, bratra a milého, zhroutila se~do~šílenství a~nakonec utonula.
Laertes se vrátil ještě před Hamletem, akorát aby našel svou sestru šílenou a viděl její poslední chvíle.
Tohoto využije Claudius, který \texthl{Laerta poštve proti \textit{Hamletovi}}.
Oba muži souhlasí s šermířským soubojem a běsem zaslepený Laertes přijímá otrávenou čepel od Claudia,
aby zaručil, že Hamlet toho dne zemře.
Claudius si~chce výsledek pojistit ještě tím, že připraví otrávenou číši,
ze které by nechal Hamleta napít, kdyby souboj vyhrál bez jediného škrábance.
Z číše se napije královna Gertruda, Hamlet~je~opravdu zasažen otrávenou čepelí,
muži~si~čepele vymění a jedem je zasažen i Laertes, ten zjeví králův hrůzný plán Hamletovi
a ten otrávenou čepelí probodne krále a~do~krku mu~nalije jeho otrávené víno.
Všichni tedy umírají a Hamlet svými \textithl{posledními slovy žádá},
aby se jeho příběh nadále \textithl{tradoval po pravdě} a bez Claudiových lží.

\subwection{Kompozice, prostor a čas:}
\noindent 
Chronologický děj v pěti dějstvích vypráví příběh o tragédii prince z Denmarku, ze hradu Elsinor v Dánsku, v počátku 17. století.

% \subwection{Význam sdělení:}

\wection{LITERÁRNÍ HISTORIE}

\subwection{Politická situace:}
\noindent
\textit{Hamlet} byl napsán a hrán v období, kdy královně Elizabetě I. bylo kolem šedesáti let.
Šlo o období nepokojů a nejistoty ohledně koruny, protože trůn neměl dědice a královna odmítala jednoho jmenovat.
Poddaní si o svého času oblíbené královně začali myslet, že se na stará kolena stala slobymyslnou až nebezpečnou.
\textit{Hamlet} přesně takového monarchu popisuje, králi se nedá věřit ohledně jeho úmyslů a princ, následník trůnu, hraje blázna a aktivně jde proti králi.
Stejně jako v \textit{Shakespearově} realitě, i v jeho hře jde o nejisté období, kdy nikdo neví, jak bude královský rod pokračovat.

\subwection{Vlivy na dané dílo:}
\noindent
\textit{Shakespeare} se pro svůj příběh inspiroval starou norskou legendou.
Stejně jako u jiných svých významných děl tedy vzal příběh a ve svém podání z něj udělal hit.
\textit{Gesta Danorum}, či \textit{Činy Dánů}, jak je šestnáctidílná série napsaná \textit{Saxo Grammaticus} známá u nás, popisuje historii Dánska a knihy třetí a čtvrtá pojednávají právě o princi, který se stal předlohou pro \textit{Hamleta}.

\subwection{Kontext dalších druhů umění:}
\noindent
Drama náleží k podžánru, pro který tou dobou anglické obecenstvo ještě nemělo název a tím je \enquote{tragédie o pomstě} (v angličtině \textit{\enquote{Revenge tragedy}}).
Jde o~podžánr, který sdílí se známou \textithl{Španělskou tragédií}, která tomuto podžánru dala vzniknout.
Stejně jako moderní \textit{John Wick}, o něco starší \textit{Kill Bill}, či mnoho nezmíněných dalších děl, jde o příběh, kdy hrdina, který není zářnou ukázkou mravních zásad, přijde o milovaného člověka a celý příběh je o nalezení a zabití vraha - kdy hrdina většinou při daném činu sám umírá.
\textit{Hamlet}, i když vyšel jen asi dekádu po \textithl{Španělské tragédii}, tak subžánr podrobil mnoha otázkám.
Hrdina zde zná vraha už od začátku a neměl žádnou překážku v~tom ho zabít.
Příběh mohl ukončit už během prvních pár minut.
Místo toho přes monology a hluboké dialogy řeší, zda je v právu, zda to může provést, zda to smí provést.

% \subwection{Základní principy fungování společnosti:}

% \subwection{Kontext literárního vývoje:}

\subwection{Autor {\ssmall -- život autora:}}
\noindent 
\textit{William Shakespeare} je dodnes považován za jednu z nejvýznamnějších postav anglické historie.
Jeho dramata jsou světoznámá, do svých děl vymyslel spoustu slov a slovních spojení, která se později staly základem anglického jazyka. \\
\textit{Shakespearova} tvorba se dá zjednodušeně dělit do tří období jeho života.
První, což obsahuje díla s relativně nadějným námětem, patří sem i právě \textit{Romeo~a~Julie}.
V dalším období nahrazuje vtip za ironii a optimismus za rozpor světa, sem patří i \textit{Hamlet}.
A poslední období obsahuje příběhy s pohádkovými náměty, komedie i tragédie, z tohoto období vzešla kupříkladu \textit{Zimní pohádka}.

\subwection{Další autorova tvorba:}
\noindent 
Mezi další významná díla patří \textit{Romeo a Julie}, \textit{Othello}, \textit{Sen noci svatojánské} a \textit{Zimní pohádka}.

%\vspace{-0.4cm}
%\begin{multicols}{2}
\wection{LITERÁRNÍ KRITIKA}

\subwection{Dobové vnímání díla a jeho proměny:}
\noindent
Při prvních představeních byl \textit{Hamlet} přijat velmi pozitivně, dle dostupných zdrojů šlo o krvavější verzi moderních převyprávění, ale diváci hru milovali.
Během 18. století došlo ke změně, kritika narůstala a mnozí začali hru odsuzovat jako vulgární a barbarskou, že působí jak kdyby byla napsána opilcem.
V začátcích 20. století se náhled začal zase měnit.
Lidé se na něj začali dívat více psychologicky a oceňovali propracovanost postav.

\subwection{Jak dílo inspirovalo další vývoj literatury:}
\noindent
\textit{Shakespeare} zde porušil mnoho pravidel \enquote{správného dramatu},
které tou dobou autoři uznávali.
Nejvýznamnějších z nich je nadvláda akce nad myšlenkou, které porušil soustředěním se na monology.
Další autoři se právě v této změně narativu inspirovali
a viděli v tom možnost dát vlastním postavám větší hloubku.
\textit{Hamlet}~i~jako~příběh vyvolal odezvu, jde o díla psychologická,
která hledají hloubku v jeho činech, či o díla adaptační až epigonická,
která si dala za úkol dát hloubku vedlejším postavám, která v~originálu pouze rozšiřují Hamletovi motivy. \\
Za zmínku stojí \textit{Ophellia} jako dílo od \textithl{Lisi Kleinové},
která si vzala za úkol převyprávět příběh z pohledu hrdinky, co podle ní neměla tolik šancí vyniknout.

% \subwection{Aktuálnost tématu a zpracování tématu:}

\wection{OSOBNÍ NÁZOR}
\noindent 
Příběh samotný je poměrně jednoduchý, ale právě osobnost a motiv jednotlivých postav je tím,
v čem u mě příběh vede.
Postavy mají uvěřitelné důvody ke~svým činům a~pochybují nad jejich správností.
I samotný divák tak dostává zrcadlo, když je mu přednesen rozumný názor, který by sám v úvahu nevzal.
    
%\end{multicols}
\vfill

\noindent\begin{minipage}{\textwidth}
    {\textcolor{MPC}{\rule{\linewidth}{0.4pt}}
    \footnotesize
    \textbfhl{Blankvers --} nerýmovaný \textithl{sylabotónický}
    (záleží na počtu a typu slabik) \textithl{jambický}
    (dvojslabičná stopa sestávající z krátké a dlouhé slabiky) verš o pěti
    \textithl{stopách} (základní jednotka metra).
    \textbfhl{Anafora --} opakování shodného slova na začátku veršů.
    \textbfhl{Asyndota --} hromadění slov bez užití spojek
    (jeden kámen dva domy tři zříceniny čtyři hrobníci).
    \textbfhl{Epigon --} slepé následování originálu, dle určitých zdrojů se od adaptace
    a převyprávění liší jen menším množstvím originality.
    }
\end{minipage}

\ispageodd
\newpage
\end{document}
